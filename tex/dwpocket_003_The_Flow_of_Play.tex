\chapter{The Flow of Play}
 \index{The Flow of Play} \index{Flow} \index{Play}
       

Playing Dungeon World is a conversation of sorts; I say something, then you reply, maybe someone else chimes in. We talk about the fiction, what's happening to the characters we imagine and the world around them. We also talk about the rules, how they come from and lead back to the fiction. There are no turns or rounds in Dungeon World, no forced order of when people talk, but a conversation means taking turns. Dungeon World is never a monologue, always a conversation.

       

The rules shape the conversation. While the GM and the players are having a conversation the rules and the fiction are having a conversation too. The rules affect the game when the fiction triggers them and a rule will always tell you when it's meant to trigger.

       
\section{Moves}  \index{Moves} \index{Moves}
       

The basic unit of rules in Dungeon World is the {\bf Move} . A move looks like this:

       
\startExample
When you {\bf attack an enemy in melee} , roll+Str. On a 10+ you deal your damage to the enemy and avoid their attack. At your option, you may choose to do +1d6 damage but expose yourself to the enemy's attack. On a 7–9, you deal your damage to the enemy and the enemy makes an attack against you.
\stopExample
       

Moves are rules that describe when they trigger and what effect they have. A move always depends on a fictional action and always has some fictional effect. "Fictional" means that the action and effect come from the shared imaginative space we're describing, not from us directly. In the move above, the trigger is "when you attack an enemy in melee." The effect is what follows: a roll to be made and differing fictional effects based on the outcome of the roll. Most moves rely on one of a character's {\bf stats}  which represent the character's innate abilities.

       

When a player says their character does something that triggers a move that move happens and its rules apply. Moves and dealing damage are the only times dice are rolled. The move will tell you what dice to roll.

       

The basic rule of moves is: {\bf take the action to gain the effect} . To make the mechanical aspect of a move happen the character has to do something that triggers that move. Likewise, if the character does something that triggers a move the mechanical portion happens.

       
\subsection{Moves Are Indivisible}  \index{Moves Are Indivisible} \index{Moves} \index{Indivisible}
       

A character can't take the fictional action that triggers a move without that move occurring. For example, if Isaac tells the GM that his character dashes past a crazed, axe-wielding orc to the open door he makes the Defy Danger move because its trigger is "when you act despite an imminent threat". Isaac can't just have his character run past the orc without making the Defy Danger move and he can't make the Defy Danger move without acting despite an imminent threat. The moves and the fiction go hand-in-hand to make up the game. When a move is made it falls to the GM and players to make sure that both of these things (fiction and rules) happen.

       

Taking a fictional action that should trigger a move and not applying it looks like this: Ben says "I run past the orc to the door," but doesn't make the Defy Danger move. In this case, the GM should suggest that the move applies: "So you're Defying the Danger of the crazed orc as he swings at you?" Ben then has to be a real adventurer and Defy Danger or back off and do something else; he can't take action that triggers the move without making the move. He can't "just" run past the orc without making the move that applies.

       

Trying to apply a move without taking the action that makes the move occur happens when a player jumps straight to the effects of the move. The Hack and Slash move has damage as one of its effects. Dan can't just say "I'm Hacking and Slashing! I rolled +Str and got a 10, I do 1d8 damage." That doesn't work because his character hasn't taken any fictional action. "Hack and Slash" isn't something a character does—it's a rule that happens when the character fulfills its trigger. The GM's response should be "okay, how do you do that?" or "what does that look like?"

       

The GM's questions are there to refine the action, not to deny it. It's still a conversation. The GM asks to make sure the everyone understands what's happening and the moves involved.

       

Some moves work a little differently—they just provide a bonus all the time. These moves are still saying something fictionally and mechanically. They're saying something the character is or has. For example, the Thief move Cautious gives a constant bonus. That's still a move, it's just one that's always happening; the Thief is particularly careful when looking for signs of traps. Maybe they've learned their lesson from being caught in a trap before.

       
\section{Rolls and Results}  \index{Rolls and Results} \index{Rolls} \index{Results}
       

Once a move applies, it's time to look at the effects. Most moves tell you to roll+something. The {\bf roll}  part means to take two d6s, roll them, and add them together. The {\bf +something}  part means to add the modifier associated with that stat. So, a character with Dex modifier of +2 who launches a Volley rolls two d6s, adds them together, and adds two. Easy.

       

The result of the roll falls into three categories: a 10+ is a {\bf strong hit} . A 7–9 is a {\bf weak hit} . A 6- is a {\bf miss} .

       

Strong hits and weak hits are both {\bf hits} . A hit means the character does what they set out to, more or less. A strong hit means they do it without much trouble or complications. A weak hit means complications and unpleasantries. Sometimes, a weak hit will mean you need to make a hard decision about what to do next. The move will always say what to do for a strong and weak hit.

       

A miss means that the character's action is unsuccessful or carries major consequences. Unless the move tells you what to do, all moves work the same on a miss—the GM takes action, doing something dangerous to the characters.

       
\section{Terminology}  \index{Terminology} \index{Terminology}
       

Some moves use the phrase "{\bf deal damage} ." Dealing damage means you roll the damage dice for your class and modify it based on the weapon you were using for that move. You have to be wielding a weapon to use your class's damage dice. Default damage without a weapon is 1.

       

Some moves say "take +1 {\bf forward} ." That means to take +1 to your next move roll (not damage). The bonus can be greater than +1, or even a penalty, like -1. There also might be a condition, such as "take +1 forward to Hack and Slash," in which case the bonus applies only to the next time you roll Hack and Slash, not any other move.

       

Some moves say "take +1 {\bf ongoing} ." That means to take +1 to all move rolls (not damage). The bonus can be larger than +1, or it can be a penalty, like -1. There also might be a condition, such as "take +1 ongoing to Volley." An ongoing bonus also says what causes it to end, like "until you dismiss the spell" or "until you atone to your deity."

       

Some moves give you {\bf hold} . Hold is currency that allows you to make some choices later on by spending the hold as the move describes. Hold is always saved up for the move that generated it; you can't spend your hold from Defend on Trap Sense or vice versa.

       

There are some moves that all the players have access to. These are the {\bf Basic}  and {\bf Special}  moves. Basic moves are the things that happen often—players will roll these a lot. Special moves are moves that come up less frequently, but everyone has access to them.

       

Each class also has some of its own moves. Some of these moves are {\bf starting moves}  that the class starts with. Others are {\bf advanced moves}  that the player may choose as their character grows.

       
\section{Stats}  \index{Stats} \index{Stats}
       

The basic stats are:

       

         {\bf Strength}  (Str). The character's physical force and muscle. Used for moves in melee combat and breaking things.

       

         {\bf Dexterity}  (Dex). The character's precision and aim. Used for moves in ranged combat and avoiding things.

       

         {\bf Constitution}  (Con). The character's health and ability to take a beating. Used for moves that endure things and surviving dangers.

       

         {\bf Intelligence}  (Int). The character's accumulated knowledge and logical thinking. Used for moves that rely of remembered facts and casting some kinds of spells.

       

         {\bf Wisdom}  (Wis). The character's keen senses and mental defenses. Used for moves that rely on noticing things and casting some kinds of spells.

       

         {\bf Charisma}  (Cha). The character's force of personality and charm. Used for social moves.

       

Each basic stat has a {\bf score}  from 3 to 18 and a {\bf modifier}  from -3 to +3. When a stat is spelled out (like "Strength") that refers to the score, when the three letter abbreviation (like "Str") is used it refers to the modifier. The stat's modifier depends on the stat's score:

\bTABLE
 		
\bTR
\bTH
 			Score 			
\eTH
\bTH
Modifier
\eTH
 		
\eTR
 		
\bTR

\bTD
1-3
\eTD

\bTD
-3
\eTD

\eTR
 		
\bTR

\bTD
4-5
\eTD

\bTD
-2
\eTD

\eTR
 		
\bTR

\bTD
6-8
\eTD

\bTD
-1
\eTD

\eTR
 		
\bTR

\bTD
9-11
\eTD

\bTD
0
\eTD

\eTR
 		
\bTR

\bTD
12-15
\eTD

\bTD
+1
\eTD

\eTR
 		
\bTR

\bTD
16-17
\eTD

\bTD
+2
\eTD

\eTR
 		
\bTR

\bTD
18
\eTD

\bTD
+3
\eTD

\eTR
 	  
\eTABLE
        

There are also a few special stats:

       

         {\bf Bond}  is how well your character knows another character. You use Bond to aid another character or interfere with their actions. Bond is about knowledge and not about how well you get along or how similar you are. Bond may also be asymmetrical: the Fighter might know the Wizard very well, but the Wizard doesn't pay much attention to the Fighter. Your Bond with someone starts based on your history with them. Each class has starting bonds with blanks to fill in names. When you roll+Bond, count the number of Bonds you have with that person and add that to the roll.

       

         {\bf Level}  reflects how your character has grown. Your character starts at level 1, and may advance all the way to level 10. Your level tracks how far you've grown. As you advance in level you gain new moves.

       
\section{Equipment}  \index{Equipment} \index{Equipment}
       

Every adventurer needs stuff: weapons, spellbooks, armor, holy symbols, and the like. Each item says what it does. In general, weapons define the way a character deals damage with it and at what range they can do that damage. Armor and shields reduce damage taken. Other items have various effects.

       

Items say what they do through their {\bf tags} . A tag is a word or phrase that indicates some common ability the item possess. The Messy tag, for example, means the weapon does damage in a particularly devastating way.

       

All items, unless otherwise noted, are {\bf mundane} . They're not magical in any way. Some items are enchanted—they work through arcane or divine tricks. These magic items are tougher to get, tougher to destroy, and more powerful to use.

       

Characters are limited in how much they can carry by their {\bf Load} . Load is determined by class. Carrying items whose total weight is more than your load causes problems.

       

Some classes have other specific tools at their disposal like custom gear or ties to powerful entities or organizations. The rules for these are detailed with each class.

       
\section{Damage and HP}  \index{Damage and HP} \index{Damage} \index{Hp}
       

Dungeon World is a dangerous place in many ways, not least of which are the physical perils that await in the forgotten halls where adventurers explore. Each character has {\bf HP} . HP is short for hit points; it's a number which reflects a character's condition. The character's HP value is determined by their class and Constitution score. Your HP doesn't automatically go up as you level, but if your Constitution score changes you update your HP as well.

       

When a character takes {\bf damage}  they reduce their current HP by that amount. If their current HP falls to zero it means they're dying and must immediately make the Last Breath move. HP never goes negative; if damage would take a character's HP below zero set it to zero instead.

       

Depending on the outcome of the Last Breath move a character may be {\bf stable}  at 0 HP. Stable means the character won't get worse on their own but they won't get better without care or time. If a stable character takes damage they stay at 0 HP but must make the Last Breath move again immediately.

       

Armor prevents damage. When you take damage you subtract your armor from the damage dealt.

       

The Cleric is all about healing HP. Without the Cleric's healing magics, adventurers are left at the mercies of bandages, poultices, and other crude medicines, plus the odd healing potion.

       

HP tracks the assorted bruises and cuts that accumulate but some wounds go deeper. These are {\bf debilities} . Debilities give you a -1 to your modifier for one stat. They don't effect the base score (so being Weak won't effect your Load, just your Str modifier). They're tougher to heal than HP, your best bet is to get somewhere safe and spend a few days resting to get rid of them.

       
\section{Advancement}  \index{Advancement} \index{Advancement}
       

Adventurers in Dungeon World grow and learn from their experiences. Eventually, with time and luck, they survive to level up.

       

Experience is tracked via {\bf XP} . Players mark XP by keeping a tally on their character sheet. When they have XP marks equal to their current level + 7 they are ready to level up, but they do not actually level up until they have some downtime (usually in camp or in a nearby village). Gaining a new level means choosing a new move from your class. If your new level is 3rd, 6th, or 9th you also get to increase one stat by 2, adjusting the modifier to reflect the new score.

       

There are two times when you mark XP: when you roll a miss (6-) and when you make the End of Session move.

       

Whenever you roll a miss (a 6 or lower) when making a move you mark XP. These are the tough lessons of the adventuring life.

       

At the end of each session one of your Bonds may resolve. When a Bond is no longer applicable it can be resolved if the person you share that Bond with agrees. When a Bond is resolved you mark XP and write a new bond.

       

You also look back at your alignment over the session you just concluded. If you fulfilled your alignment at least once in the session you'll get XP.

       

The End of Session move also has three questions that all the players answer as a group. For each "yes" answer everyone gains XP.

       
\section{Sessions}  \index{Sessions} \index{Sessions}
       

A session of Dungeon World is one time you sit down to play. A session usually runs a few hours and may be a single adventure or part of a larger campaign.

       

The first session of a game starts of a little different. First you'll need to choose a GM. Once the GM is settled, everyone else will need to make characters. See the character creation chapter for more on that.

       

During character creation, the GM will be asking questions and making plans for how to start the game. A game of Dungeon World always starts with action, either action already underway or impending.

       

Such a situation will lead to the characters making moves, which will cause further moves. Moves lead to more moves. This snowballing action from move to move is what builds an exciting game. Once a few moves have been made, you'll find it easy to keep going. The moves will keep giving you ideas and prompts for further action which leads to more moves.

       
\section{Why?}  \index{Why}
       

Why play Dungeon World?

       

First, to see the characters do {\bf amazing things} . To see them explore the unexplored, slay the undying, and go from the deepest bowels of the world to the highest peaks of the heavens. To see them caught up in momentous events and grand tragedies.

       

Second, to see them {\bf play off each other} : to stand together as a united front against their foes or to bicker and argue over treasure. To unite and fall apart and reunite again.

       

Third, because {\bf the world still has so many places to explore} . There are unlooted tombs and dragon hordes dotting the countryside just waiting for quick-fingered and strong-armed adventurers to discover them. That unexplored world has plans of its own. We play to see what they are and how they'll change the lives of our characters.

                
