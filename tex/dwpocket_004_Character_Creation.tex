\chapter{Character Creation}
 \index{Character Creation} \index{Character} \index{Creation}
 



Making Dungeon World characters is quick and easy. You should all create your first characters together at the beginning of your first session. Character creation is, just like play, a kind of conversation—everyone should be there for it. It's somewhat likely your character may die along the way. if they do, no worries, the character creation process helps you make a new character that fits into the group in just a few minutes.

 

If you're the GM, your role during character creation is to help everyone, ask questions, and take notes. When a player makes a choice—particularly for their Bonds—ask them about it. Get more detail. Think about what these details mean.

 

The GM should also set expectations: the players are to play their characters as people. Skilled adventurers delving into dangerous places, but real people. The GM's role is to play the rest of the world as a dynamic, changing place.

 

Some questions commonly come up during character creation:

 

{\em Are the characters friends?}  No, not necessarily, but they do work together as a team for common goals. Their reasons for pursuing those goals may be different, but they generally manage to work together.

 

{\em Are there other Wizards?}  Not really. There are other workers of arcane magic, and the common folk may call them wizards, but they're not like you. They don't have the same abilities, though they may be similar. Same goes for any class: there's only one Cleric, though there are many with similar powers of divine servitude. There's only one Thief, but there are others that fight from the shadows and steal things.

 

{\em What's coin?}  Coin's the currency of the realm. It's good pretty much everywhere. It'll buy you mundane stuff, like steel swords and wooden staves, but the special stuff, like magic weapons, isn't for sale.

 

{\em Is the GM trying to kill us?}  Nope. The GM represents the world. It's a dangerous place, and yeah, you might die. But she's not trying to kill you.

 

Most everything you need to create a character you'll find on the character sheets. These steps will walk you through filling out a character sheet.

 
\section{1. Choose a Class}  \index{1 Choose a Class} \index{1} \index{Choose} \index{Class}
 

Look over the character classes and choose one that interests you. Everyone chooses a different class; there aren't two Wizards. If two people want the same class, talk it over like adults and compromise.

 
\startExample
I sit down with Paul and Shannon to play a game run by John. I've got some cool ideas for a Wizard, so I mention that would be my first choice. No one else was thinking of playing one, so I take the playbook.
\stopExample
 
\section{2. Choose a Race}  \index{2 Choose a Race} \index{2} \index{Choose} \index{Race}
 

Every class has a few race options. Choose one. Your race gives you a special move.

 
\startExample
I like the idea of summoning up Things From Beyond, so I choose Human, since that gives me a bonus to Summoning spells. I thought about being an Elf, but Shannon's playing the Cleric so I don't think we'll need more Cleric spells.
\stopExample
 
\section{3. Choose a Name}  \index{3 Choose a Name} \index{3} \index{Choose}
 

Choose your character’s name from the list.

 
\startExample
Avon sounds good.
\stopExample
 
\section{4. Choose Look}  \index{4 Choose Look} \index{4} \index{Choose}
 

Your look is your physical appearance. Choose one item from each list.

 
\startExample
Haunted eyes sound good, since I've seen Things From Beyond. No good Wizard has time for hair styling, wild hair it is. My robes are strange, and I mention to everyone that I think maybe they came from Beyond as part of one of my summonings. No time to eat with all that magic: thin body.
\stopExample
 
\section{5. Choose Stats}  \index{5 Choose Stats} \index{5} \index{Choose} \index{Stats}
 

Look over the basic moves and the starting moves for your class. Pick out the move that interests you the most: something you'll be doing a lot, or something that you excel at. Put your 17 in the stat for that move. Look over the list again and pick out the next most important move to your character, maybe something that supports your first choice. Put your 15 in the stat for that move. Repeat this process for your remaining scores: 13, 11, 9, 8.

 

Alternatively, if everyone wants a little more randomness then you can roll stats. Roll 3d6 and assign the total to a stat—repeat this until you have all your stats.

 

If you want something really random you can roll for stats in order (Str, Dex, Con, Int, Wis, Cha). If you choose this method you get to roll before you choose your class.

 
\startExample
It looks like I need Intelligence to cast spells, which are my thing, so my 17 goes there. The Defy Danger option for Dexterity looks like something I might be doing to dive out of the way of a spell, so that gets my 15. A 13 Wisdom will help me notice important details (and maybe keep my sanity, based on the Defy Danger move). Charisma might be useful is dealing with summoned creatures so I'll put my 11 there. Living is always nice, so I put my 9 in Constitution for some extra HP. Strength gets the 8.
\stopExample
 
\section{6. Figure Out Modifiers}  \index{6 Figure Out Modifiers} \index{6} \index{Figure} \index{Modifiers}
 

Next you need to figure out the modifiers for your stats. The modifiers are what you use when a move says +Dex or +Cha. You won’t actually use the raw scores much.

\bTABLE
 		
\bTR
\bTH
 			Score 			
\eTH
\bTH
Modifier
\eTH
 		
\eTR
 		
\bTR

\bTD
1-3
\eTD

\bTD
-3
\eTD

\eTR
 		
\bTR

\bTD
4-5
\eTD

\bTD
-2
\eTD

\eTR
 		
\bTR

\bTD
6-8
\eTD

\bTD
-1
\eTD

\eTR
 		
\bTR

\bTD
9-11
\eTD

\bTD
0
\eTD

\eTR
 		
\bTR

\bTD
12-15
\eTD

\bTD
+1
\eTD

\eTR
 		
\bTR

\bTD
16-17
\eTD

\bTD
+2
\eTD

\eTR
 		
\bTR

\bTD
18
\eTD

\bTD
+3
\eTD

\eTR
 	  
\eTABLE
        


\section{7. Set Starting HP}  \index{7 Set Starting HP} \index{7} \index{Set} \index{Hp}
 

Your starting HP is equal to your class's base HP+Constitution score.

 
\startExample
Base 4 plus 9 con gives me a whopping 13 HP. I guess Summoning takes a toll on the body.
\stopExample
 
\section{8. Choose Starting Moves}  \index{8 Choose Starting Moves} \index{8} \index{Choose} \index{Moves}
 

Some classes, like the Fighter, have choices to make as part of one of their moves. Make these choices now. The Wizard will need to choose spells for their spellbook. Both the Cleric and the Wizard will need to choose which spells they have prepared to start with.

 
\startExample
A Summoning spell is an easy choice, so I take Contact Spirits. Magic Missile will allow me to deal more damage than my pitiful d4 damage dice, so that's in too. I choose Alarm for my last spell, since I can think of some interesting uses for it.
\stopExample
 
\section{9. Choose Alignment}  \index{9 Choose Alignment} \index{9} \index{Choose} \index{Alignment}
 

Your alignment is a few words that describe your character's moral outlook. These are general and tend to guide your character's outlook rather than dictate their actions. Usually alignment is a single term declaring the character's allegiance to the forces of good, the hordes of evil, or the path of neutrality between. The alignments are Good, Evil, and Neutral. Some classes may only be certain alignments. Choose your alignment—it gives you more ways to earn XP.

 
\startExample
Avon is all about the magical mysteries, which makes the Neutral alignment stand out. I'll go with that one.
\stopExample
 
\section{10. Choose Gear}  \index{10 Choose Gear} \index{10} \index{Choose} \index{Gear}
 

Each class has choices to make for starting gear. Keep your Load in mind—it limits how much you can easily carry. Make sure to total up your armor and note it on your character sheet.

 
\startExample
I'm worried about my HP, so I take armor over books. A dagger sounds about right for rituals, I choose that over a staff. It's a toss up between the healing potion and the antitoxin, but healing wins out. I also end up with some rations.
\stopExample
 
\section{11. Introduce Your Character}  \index{11 Introduce Your Character} \index{11} \index{Introduce} \index{Character}
 

Now that you know who your character is, it's time to introduce them to everyone else. Wait until everyone's finished choosing their name. Then go around the table; each player gets to share their look, class and anything else about their character. You can share your alignment now or keep it a secret if you prefer.

 

This is also the time for the GM to ask questions. The GM's questions should help establish the relationships between characters ("What do you think about that?") and draw the group into the adventure ("Does that mean you've met Grundloch before?"). The GM should listen to everything in the description and ask about anything that stands out. Establish where they're from, who they are, how they came together, or anything else that seems relevant or interesting.

 
\startExample
"This is Avon, summoner of Things From Beyond! He's a human wizard with haunted eyes, wild hair, strange robes, and a thin body. Like I mentioned before his robes are strange because they're literally not of this world: they came through as part of a summoning ritual."
\stopExample
 
\section{12. Choose Bonds}  \index{12 Choose Bonds} \index{12} \index{Choose} \index{Bonds}
 

Once everyone has described their characters you can choose your Bonds. You must fill in one bond but it's in your best interest to fill in more. For each blank fill in the name of one character. You can use the same character for more than one statement.

 

Once everyone’s filled in their bonds read them out to the group. When a move has you roll+bonds you'll count the number of Bonds you have with the character in question and add that to the roll.

 
\startExample
With everyone introduced I choose which character to list in each Bond, I have Paul's Fighter Gregor and Shannon's Cleric Brinton to choose from. The Bond about prophecy sounds fun, so I choose Gregor for it and end up with "Gregor will play an important role in the events to come. I have foreseen it!" It seems like The Wizard who contacts Things From Beyond and the Cleric might not see eye to eye, so I add Shannon's character and get "Brinton is woefully misinformed about the world; I will teach them all that I can." I leave my last Bond blank, I'll deal with it later. Once everyone is done I read my Bonds aloud and we all discuss what this means about why we're together and where we're going.
\stopExample






 
