\chapter{Basic Moves}
 \index{Basic Moves} \index{Basic} \index{Moves}
 


\section{Hack and Slash}  \index{Hack and Slash} \index{Hack} \index{Slash}
 

When you {\bf attack an enemy in melee} , roll+Str. On a 10+ you deal your damage to the enemy and avoid their attack. At your option, you may choose to do +1d6 damage but expose yourself to the enemy's attack. On a 7–9, you deal your damage to the enemy and the enemy makes an attack against you.

 
\section{Volley}  \index{Volley} \index{Volley}
 

When you {\bf take aim and shoot at an enemy at range} , roll+Dex. On a 10+ you have a clear shot—deal your damage. On a 7–9, choose one (whichever you choose you deal your damage):

 
\startitemize[1,packed]

\item You have to move to get the shot placing you in danger of the GM's choice

 
\item You have to take what you can get: -1d6 damage

 
\item You have to take several shots, reducing your ammo by one.


\stopitemize
 
\section{Defy Danger}  \index{Defy Danger} \index{Defy} \index{Danger}
 

When you {\bf act despite an imminent threat}  or {\bf suffer a calamity} , say how you deal with it and roll. If you do it…

 
\startitemize[1,packed]

\item …by powering through, +Str

 
\item …by getting out of the way or acting fast, +Dex

 
\item …by enduring, +Con

 
\item …with quick thinking, +Int

 
\item …through mental fortitude, +Wis

 
\item …using charm and social grace, +Cha


\stopitemize
 

On a 10+, you do what you set out to, the threat doesn't come to bear. On a 7–9, you stumble, hesitate, or flinch: the GM will offer you a worse outcome, hard bargain, or ugly choice.

 
\section{Defend}  \index{Defend} \index{Defend}
 

When you {\bf stand in defense of a person, item, or location}  under attack, roll+Con. On a 10+, hold 3. On a 7–9, hold 1. So long as you stand in defense, when you or the thing you defend is attacked you may spend hold, 1 for 1, to choose an option:

 
\startitemize[1,packed]

\item Redirect an attack from the thing you defend to yourself

  
\item Halve the attack's effect or damage

 
\item Open up the attacker to an ally giving that ally +1 forward against the attacker

 
\item Deal damage to the attacker equal to your level


\stopitemize
 
\section{Spout Lore}  \index{Spout Lore} \index{Spout} \index{Lore}
 

When you {\bf consult your accumulated knowledge about something} , roll+Int. On a 10+ the GM will tell you something interesting and useful about the subject relevant to your situation. On a 7–9 the GM will only tell you something interesting—it's on you to make it useful. The GM might ask you "How do you know this?" Tell them the truth, now.

 
\section{Discern Realities}  \index{Discern Realities} \index{Discern} \index{Realities}
 

When you {\bf closely study a situation or person} , roll+Wis. On a 10+ ask the GM 3 questions from the list below. On a 7–9 ask 1. Take +1 forward when acting on the answers.

 
\startitemize[1,packed]

\item What happened here recently?

 
\item What is about to happen?

 
\item What should I be on the lookout for?

 
\item What here is useful or valuable to me?

 
\item Who’s really in control here?

 
\item What here is not what it appears to be?


\stopitemize
 
\section{Parley}  \index{Parley} \index{Parley}
 

When you {\bf you have leverage on a GM character and manipulate them} , roll+Cha. Leverage is something they need or want. On a hit they ask you for something and do it if you make them a promise first. On a 7–9, they need some concrete assurance of your promise, right now.

 
\section{Aid or Interfere}  \index{Aid or Interfere} \index{Aid} \index{Interfere}
 

When you {\bf help or hinder someone you have a Bond with} , roll+Bond with them. On a 10+ they take +1 or -2, your choice. On a 7–9 you also expose yourself to danger, retribution, or cost.



 
