\chapter{First Session}
 \index{First Session} \index{Session}
 



The first session of a game of Dungeon World begins with character creation. Character creation is also world creation, the details on the character sheets and the questions the GM asks establish what Dungeon World is like—who lives in it and what's going on.

 

This section is for the GM so it's addressed to you—the GM. For the players, the first session is just like every other. They just have to play their characters like real people and explore Dungeon World. The GM has to do a little more in the first session. They establish the world and the threats the players will face. Don't worry, it's fun.

 
\section{Prep}  \index{Prep} \index{Prep}
 

Before the first session, you'll need to print some stuff. Print off:

 
\startitemize[1,packed]

\item A few copies of the basic moves and special moves (double sided, basic on one side and special on the other). You'll want about one per player.

 
\item One copy of each class sheet, double sided.

 
\item One copy each of the cleric and wizard spell sheets, double sided.

 
\item A few copies of the equipment sheet.

 
\item The GM moves.

 
\item The GM worksheets.


\stopitemize
 

You'll also need to read this whole book, especially the sections on GMing (GM moves) and the basic moves. It's a good idea to be acquainted with the class moves too, so you can be prepared for them. Be especially sure to read the rules for Fronts, but don't create any yet.

 

Think about fantastic worlds, strange magic, and foul beasts. If you've played other fantasy games like Dungeons and Dragons think back to what made your old games so much fun. Remember the games you played and the stories you told. They can all provide inspiration for your Dungeon World game. Watch some movies, read some comics; get heroic fantasy into your brain.

 

What you bring to the first session, ideas-wise, is up to you. At the very least bring your head full of ideas. That's the bare minimum.

 

If you like you can plan a little more. Maybe think of an evil plot, or who's behind it, or some monsters you'd like to use.

 

If you've got some spare time on your hands you can even draw some maps (but remember, from your Principles: leave blanks) and imagine specific locations. Flexibility is key when planning: what happens during character creation trumps anything you wrote ahead of time.

 

The one thing you absolutely can't bring to the table is a planned storyline or plot. You don't know the heroes or the world before you sit down to play so planning anything concrete is just going to frustrate you. It also conflicts with your Agenda: play to find out what happens.

 

Don't use the Fronts rules (in the next chapter) for the first session either. Those will come with time but in the first session you need to be able to focus more on getting the game rolling. The big picture doesn't matter so much, yet. Instead focus on getting the players into action, interacting with each other, and using the rules.

 
\subsection{Getting Started}  \index{Getting Started} \index{Started}
 

When everyone shows up for the first session briefly introduce anyone who hasn't played before to Dungeon World. Cover the mechanical basis of moves. Introduce the character classes, help players pick their classes, and walk them through character creation.

 

During this entire process, especially character creation, ask questions. Look for interesting facts established by the characters' Bonds, moves, classes, and descriptions and ask about those things. Be curious! When someone mentions the demons that slaughtered their village find out more about them. After all, you don't have anything (except maybe a dungeon) and everything they give you is fuel for future adventures.

 

Also pay attention to the players' questions. When mechanical questions come up answer them. When questions of setting or fiction come up your best bet is to turn those questions around. When a player says "Who is the King of Torsea" say "I don't know. Who is it? What is he like?" Collaborate with your players. Asking a question means it's something that interests them so work with them to make the answers interesting. Don't be afraid to say "I don't know" and ask them the same questions; work together to find a fantastic and interesting answer.

 

Share the ideas you've brought to the table (either general ones or even a specific dungeon). If you're interested in starting with the players hunting for a lost wizard, tell them that. Until the players agree, it's just your idea. Once they nod their heads, it's part of the game.

 

Once everyone has their characters created you can take a deep breath. Look back over the questions you've asked and answered so far. You should have some notes that will point you towards what the game might look like. Look at what the players have brought to the table. Look at the ideas that've been stewing away in your head. It's time for the adventure to begin!

 
\subsection{The First Adventure}  \index{The First Adventure} \index{Adventure}
 

The first adventure is really about finding out what future sessions will deal with. Throughout the first adventure keep your eye out for unresolved threats; note dangerous things that are mentioned but not dealt with. These will be fuel for future sessions.

 

Start the session with a group of player characters (maybe all of them) in a tense situation. Use anything that demands action: outside the entrance to a dungeon, ambushed in a fetid swamp, peeking through the crack in a door at the orc guards, or being sentenced before the King. If the situation stems directly from the characters and your questions, all the better.

 

Here's where the game starts. The players will start saying and doing things, which means they'll start making moves. For the first session you should watch especially carefully for when moves apply, until the players get the hang of it. Often, in the early sessions, the players will be most comfortable just narrating their actions—this is fine. When a move triggers let them know. Say "It sounds like you're trying to…" and then walk them through the move. Players looking for direction will look to their character sheet. Be quick to ask "so what are you actually doing?" when a player just says "I Hack and Slash him." Ask, too, "how?" or "with what?".

 

For the first session, you have a few specific goals:

 
\startitemize[1,packed]

\item Establish details, describe

 
\item Use what they give you

 
\item Ask questions

 
\item Leave blanks

 
\item Look for interesting facts

 
\item Help the players understand the moves

 
\item Give each character a chance to shine

 
\item Introduce NPCs


\stopitemize
 
\subsubsection{Establish details, describe}  \index{Establish details describe} \index{Establish} \index{Details} \index{Describe}
 

All the ideas and visions in your head don't really exist in the fiction of the game until you share them, describe them and detail them. The first session is the time to establish the basics of what things look like, who's in charge, what they wear, what the world is like, what the immediate location is like. Describe everything but keep it brief enough to expand on later. Use a detail or two to make a description really stand out as real.

 
\subsubsection{Use what they give you}  \index{Use what they give you} \index{Give}
 

The best part of the first session is you don't have to come with anything concrete. You might have a dungeon sketched out but the players provide the real meat—use it. They'll emerge from the darkness of that first dungeon and when they do and their eyes adjust to the light, you'll have built up an exciting world to explore with their help. Look at their Bonds, their moves, how they answer your questions and use those to fill in the world around the characters.

 
\subsubsection{Ask questions}  \index{Ask questions} \index{Questions}
 

You're using what they give you, right? What if you need more? That's when you draw it out by asking questions. Poke and prod about specific things. Ask for reactions "what does Lux think about that?" "is Avon doing something about it?"

 

If you ever find yourself at a loss, pause for a second and ask a question. Ask one character a question about another. When a character does something, ask how a different character feels or reacts. Questions will power your game and make it feel real and exciting. Use the answers you find to fill in what might happen next.

 
\subsubsection{Leave blanks}  \index{Leave blanks} \index{Leave} \index{Blanks}
 

This is one of your Principles, but it's especially true during the first session. Every blank is another cool thing waiting to happen, leave yourself a stock of them.

 
\subsubsection{Look for interesting facts}  \index{Look for interesting facts} \index{Facts}
 

There are some ideas that, when you hear them, just jump out at you. When you hear one of those ideas, just write it down. When a player mentions the Duke of Sorrows being the demon he bargained with, note it. That little fact is the seed for a whole world.

 
\subsubsection{Help the players understand the moves}  \index{Help the players understand the moves} \index{Players} \index{Understand} \index{Moves}
 

You've already read the game, the players may not have, so it's up to you to help them if they need it. The fact is, they likely won't need it much. All they have to do is describe what their character does, the rules take care of the rest.

 

The one place they may need some help is remembering the triggers for the moves. Keep an ear out for actions that trigger moves, like attacking in melee or consulting their knowledge. After a few moves the players will likely remember them on their own.

 
\subsubsection{Give each character a chance to shine}  \index{Give each character a chance to shine} \index{Give} \index{Character} \index{Chance} \index{Shine}
 

As a fan of the heroes (remember your Agenda?) you want to see them do what they do best. Give them a chance at this, not by tailoring every room to their skills, but by portraying a fantastic world (Agenda again) where there isn't one solution to everything.

 
\subsubsection{Introduce NPCs}  \index{Introduce NPCs} \index{Introduce} \index{Npcs}
 

NPCs bring the world to life. If every monster does nothing more than attack and every blacksmith sets out their wares for simple payment the world is dead. Instead give your characters, especially those that the players show an interest in, life (Principles, remember?). Introduce NPCs but don't protect them. The recently-deceased Goblin King is just as useful for future adventures as the one who's still alive.

 






