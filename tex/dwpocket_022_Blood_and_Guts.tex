\chapter{Blood and Guts}
 \index{Blood and Guts} \index{Blood} \index{Guts}
 



Death and dismemberment are common dangers for adventurers to face in Dungeon World. In the course of play, characters will take damage, heal, and maybe even die. A character's health is measure by their HP (HP being short for hit points). Damage subtracts from HP, which may lead to death. In the right conditions, or with medical or magical help, damage is healed and HP is restored.

 
\section{HP}  \index{HP} \index{Hp}
 

A character's HP is a measure of their stamina, endurance, and health. More HP means the character can fight longer and endure more before facing Death's cold stare. Think of HP in the abstract—a character with high HP can't be hit in the head any more times than one with low HP—they just have greater stores of energy to expend before it comes to blows-to-the-head.

 

Your class tells you how many HP you get. Your Constitution (the score, not the modifier) always comes into play as well so more Constitution means more HP. If your Constitution permanently changes during play you adjust your HP to reflect your new Constitution score. Unless your Constitution changes your maximum HP stays the same.

 
\subsection{Damage}  \index{Damage} \index{Damage}
 

When a character takes damage they subtract the damage dealt from their current HP. Armor mitigates damage; if a character has armor they subtract their armor's value from the damage dealt. Damage can never take a character below 0 HP.

 

Damage is decided by the attacker. Each class has a base damage die, which may be modified by the weapon used. No matter the implement, the Fighter will always deal more damage than a Wizard—it's about training and skill. Monsters and other non-player characters have a static damage instead of a dice to roll.

 

Player characters deal damage according to their class, the weapon used and the move they've made. When a character is armed, they deal their class's damage. If a character is unarmed, they probably can't deal damage, or they might do 1 stun damage.

 

If a move just says "deal damage" the character rolls their class's damage dice plus any bonuses or penalties from moves or weapons. If a move specifies an amount of damage, use that in place of the class's damage roll.

 

Monsters' damage is listed in their description. Use this damage any time the monster takes direct action to hurt someone, even if they use a method other than their normal attack.

 

Other sources of damage—like being struck by a chunk of a collapsing tower, or falling into a pit—are left to the GM based on these options:

 
\startitemize[1,packed]

\item It threatens bruises and scrapes at worst: d4 damage

 
\item It's likely to spill some blood, but nothing horrendous: d6 damage

 
\item It might break some bones: d8 damage

 
\item It could kill a common person: d10 damage


\stopitemize
 

Add the Ignores Armor tag if the source of the damage is particularly large or if the damage comes from magic or poison.

 

Temporary or circumstantial armor works the same way: 1 armor for partial cover, 2 armor for major cover.

 

Remember that damage is both prescriptive and descriptive: if a move says someone takes damage—they have been struck by the weapon or ability causing the damage. If a character is struck by a weapon, they take damage. This means that you can deal damage without making a move. Think of it like an implied move: if you hurt someone and no other move applies, you just deal your damage.

 

Damage only applies when the injury is general. Falling into a pit trap is general, it could cause any sort of injury, so it's represented by HP loss. When the harm is specific, like an orc pulling your arm from its socket, HP should be part of the effect but not the entirety of it. The bigger issue is dealing with the newly disjointed arm: how do you swing a sword or cast a spell? Likewise having your head chopped off is not HP damage, it's just death.

 
\subsubsection{Damage From Multiple Creatures}  \index{Damage From Multiple Creatures} \index{Damage} \index{Multiple} \index{Creatures}
 

It's a brave monster that goes into battle alone. Most creatures fight with someone at their side, and maybe another at their back, and possibly an archer covering the rear, and so on. This can lead to multiple monsters dealing their damage at once.

 

If multiple creatures attack at once roll the damage die for each of them and take the highest result. If some of the creatures deal a different amount of damage roll the damage with the highest potential for each creature involved in the attack and take the highest result.

 
\startExample
A goblin orkaster (d10+1 damage ignores armor) and three goblins (d6 damage) all throw their respective weapons—a magical acid orb for the orkaster, spears for the rest—at Lux as she assaults their barricade. I roll the highest damage, d10+1 ignores armor, four times: once for the orkaster, and once for each of the other goblins. I take the highest result, a roll of 8, and tell Lux she takes 9 damage ignoring armor as the acid leaks into the scratches left by the spears.
\stopExample
 
\subsection{Stun Damage}  \index{Stun Damage} \index{Stun} \index{Damage}
 

Stun damage is non-lethal damage. A PC who takes stun damage is Defying Danger to do anything at all, the danger being "you're stunned." A GM character that takes stun damage counts it against their HP as usual, but when they are out of HP they are knocked out, not at the GM's mercy.

 
\subsection{Healing}  \index{Healing}
 

There are two sources of healing in Dungeon World: the passage of time and medical aid.

 

Whenever a character spends some time resting without aggravating their wounds they heal. The amount of healing is described in the move (Make Camp for a night in a makeshift bed, Recover for says in civilization).

 

Medical aid, both magical and mundane, also provides healing. The amount of damage healed is dependent on the move or item used. Some moves may fully replenish HP while others are just enough to keep someone standing through a fight.

 

No matter the source of the healing a character's HP can never increase over their maximum.

 
\subsection{Death}  \index{Death} \index{Death}
 

Death walks the edges of every battle. It waits silently to claim those that fall. A characters who is reduced to 0 HP immediately takes his Last Breath. Death comes for commoner and king alike—no stat is added to the Last Breath roll.

 

What lies beyond the Black Gates of Death is unknown but it is said that many secrets of the mortal plane are laid bare in what lies beyond.

 

Death's bargains range from the simple to the costly. Death is capricious. One life may be traded for two more dead while for another Death may demand eternal servitude.

 

Depending on the outcome of the Last Breath the character may become stable. A stable character stays at 0 HP but is alive and unconscious. If they receive healing they regain consciousness and may return to battle or seek safety. If a stable character takes damage again they face Death and draw their Last Breath once more.

 
\subsection{After Death}  \index{After Death} \index{Death}
 

Being an adventurer isn't easy—it's cold nights in the wild and sharp swords and monsters. Sooner or later, you're going to make that long walk to the Black Gates and give up the ghost. In Dungeon World, Death is always watching and waiting for an adventurer to slip up and visit the other side. That doesn't mean you have to give it the satisfaction of sticking around. Death, in its way, is just another challenge to conquer. Even dead adventurers can rise again.

 

If your character dies, you can ask the GM and the other players to try and resurrect you. The GM will tell them what it will cost to return your poor, dead character to life. If you're all willing to pay that cost and succeed at the goal set before you then your character can cross back over to the land of the living. The Resurrection spell is a special case of this: the magic of the spell gives you an easier way to get a companion back, but the GM still has a say.

 

While this quest is underway you can play a new character. Maybe a hireling becomes a full-fledged adventurer worthy of a whole share and a part in the real action. Maybe the characters in the party find a new friend in a steading, willing to join them. Maybe your character had a vengeful family member who now seeks to take up their blades and spells to make right what happened. In any case, make your new character as you normally would at level 1. Add Bonds with the other player characters and join in the quest to resurrect the fallen. When the price has been paid and the quest is done, you can choose which character to play. You can then retire your new character to safety or simply have them vanish into the background. At the start of any given session, choose which character you'll be playing that time around and set the other aside. Make sure this change makes sense in the story you've created—characters can't just appear out of nowhere without a good excuse.

 

GM, when you tell the players what needs to be done to bring their comrade back, don't feel like it has to derail the flow of the current game. Weave it in to your fronts, steadings and prep. This is a great opportunity to change focus or introduce an element you've been waiting to show off. Don't feel, either, that it has to be some great and epic quest. If the character died at the end of a goblin pike, maybe all it takes is an awkward walk home and a few thousand gold pieces donated to a local temple. Think about the ramifications of such a charitable act and how it might affect the world, give the character back his sweet, sweet life and remember; Death never forgets a soul stolen from his realm.

 
\section{Debilities}  \index{Debilities} \index{Debilities}
 

Losing HP is a general thing, it's getting tired, bruised, cut, and so on. Some wounds are deeper though. These are debilities.

 
\startitemize[1,packed]

\item Weak (Str): You can't exert much force. Maybe some important muscles were slashed, or maybe the strength was pulled out of you by magic.

 
\item Shaky (Dex): You're a little unsteady on your feet and you've got a shake in your hands.

 
\item Sick (Con): Something just isn't right deep inside. It could be a disease or it might be an organ swollen to bursting.

 
\item Stunned (Int): You're having trouble… remembering? Is that how that sentence ends?

 
\item Confused (Wis): Ears ringing. Vision blurred. You're more than a little out of it.

 
\item Scarred (Cha): It may not be permanent, but for now you don't look so good. Your voice is probably weak too.


\stopitemize
 

Debilities are inflicted by certain monsters. Not every attack inflicts a debility—they're most often associated with magic, poison, or stranger things like a vampire sucking your blood. Each debility is tied to a stat and gives you -1 to that stat's modifier. The stat's score is unaffected so you don't have to worry about changing your Load when you're Weak.

 

You can only have each debility once. If you're already Sick and something makes you Sick you just ignore it.

 

Debilities are harder to heal than HP. Some high level magic can do it, sure, but your best bet is getting somewhere safe and spending a few days in a soft warm bed. Of course debilities are both descriptive and prescriptive: if something happens that would remove a debility, that debility is gone.

 

Debilities don't replace descriptions and using the established fiction. When someone loses an arm that isn't Weak, that's losing an arm. They can't hold a shield, to begin with. Don't let debilities limit you. A specific disease can have whatever effects you can dream up, Sick is just a convenient shorthand for some anonymous fever picked up from a filthy rat.

 
\section{Advancement}  \index{Advancement} \index{Advancement}
 

Dungeon World is ever-changing. The characters change, too. As their adventures progress, player characters gain experience or XP, which lets them level up. This prepares them for greater danger, bigger adventures, and mightier deeds.

 

Advancement, like everything else in Dungeon World, is both prescriptive and descriptive. Prescriptive means that when a player changes their character sheet the character changes. Descriptive means that when the character changes the player should change the character sheet to reflect that.

 

This isn't a benefit or detriment to the players or the GM; it's not an excuse to gain more powers or take them away. It's just a reflection of life in Dungeon World.

 
\startExample
Gregor offers his signature weapon, an axe permanently dyed green in orc blood, as a desperate bargain to save the King from eternal damnation. Without his axe he gets none of the benefits of his signature weapon. Should he recover it he'll have access to its benefits again.
\stopExample
 
\startExample
Avon, despite being a Wizard, has risen to the notice of Lenoral, the deity of arcane knowledge. After being blessed by an avatar of Lenoral, Avon is under the deity's watch. He can fulfill Petitions and gain boons like a Cleric.
\stopExample
 

Descriptive changes only happen when the character has clearly gained access to an ability. Befriending a stray dog does not have the same benefits as an animal companion.

 
\subsection{Level Up}  \index{Level Up} \index{Level}
 

When you {\bf have a safe moment and XP equal to (or greater than) your current level + 7} , reset your XP to 0 and choose a new advanced move from your class. If you are the Wizard, you also get to add a new spell to your spellbook.

 

New moves are chosen based on the character's new level. If a move requires 6th level, it's available as the character advances from 5th to 6th level.

 

If your new level is 3, 6, or 9, you also get to increase a stat by 2. Increase the base score of the stat of your choice by 2, adjust the modifier to reflect the new score. Changing your Constitution increases your maximum and current HP. Ability scores can't go higher than 18.

 
\subsection{Requires \& Replaces}  \index{Requires \& Replaces} \index{Requires} \index{\&} \index{Replaces}
 

Some moves depend on other moves. If another move is listed along with the word {\bf Requires}  or {\bf Replaces}  you can only gain the new move if you have the move it requires or replaces.

 

A move the requires another move can only be taken if you have the move it requires already. You then have both moves and they both apply.

 

A move that replaces another move can only be taken if you have the move it replaces already. You lose access to the replaced move and just have the new one. The new move will usually include all the benefits of the replaced one: maybe you replace a move that gives you 1 armor with one that gives you 2 armor instead.

 
\section{Bonds}  \index{Bonds} \index{Bonds}
 

Bonds are what make you a party of adventurers, not just a random assortment of people. Seeing your bonds evolve and play off each other is one of the best parts of the game.

 

That said, this isn't high drama. How you feel about Titanius doesn't matter so much when you're both fighting for life and limb against a horde of demons who would happily end the world if they could. Bonds are the icing on the cake: they make your adventures (and your adventurers) more interesting.

 
\subsection{Resolving Bonds}  \index{Resolving Bonds} \index{Bonds}
 

At the end of each session you may resolve one bond. Resolution of a bond depends on both you and the player of the character you share the bond with: you suggest that the bond has been resolved and, if they agree, it is.

 

A bond is resolved when it no longer describes how you relate to that person. That may be because circumstances have changed—Thelian used to have your back but after he didn't rush to save you from the ankheg you're not so sure. Or it could be because that's no longer a question—you guided Wesley before and he owed you, but he paid that debt when he saved your life with a well-timed spell. Any time you look at a Bond and think "that's not a big factor in how we relate anymore" the bond is at a good place to resolve.

 

If a character has blank Bonds left over from character creation they can resolve that Bond without asking anyone and write a new one or they can add a character's name to the Bond instead of writing a new Bond. Ignoring a Bond at character creation does not reduce the total Bonds available to the character.

 
\subsection{Writing New Bonds}  \index{Writing New Bonds} \index{Bonds}
 

You write a new bond whenever you resolve an old one. Your new bond may be with the same character, but it doesn't have to be.

 

When you write a new bond first choose another character. Then pick something relevant to the last session you've just finished—maybe a place you traveled together or a treasure you discovered. Lastly, choose a thought or belief your character holds that ties the two together and an action, something you're going to do about it. You'll end up with something like this:

 
\startitemize[1,packed]

\item Avon proved himself a coward in the dungeons of Xax'takar, he is a dangerous liability to the party and must be watched.

 
\item Mouse's quick thinking saved me from the white dragon we faced. I owe her a boon.

 
\item Xotoq won the Bone-and-Whispers Axe through trickery! It will be mine, I swear it.

 
\item Valeria's kindness to the Gnomes of the Vale has swayed my heart. I will find a way to prove to her my love.


\stopitemize
 

These new bonds act just like the old ones. They are still resolved and still grant XP when resolved.

 

If you chose not to use a starting Bond you can replace it with a new Bond at the end of any session. This does not count as resolving a Bond, you don't get XP for it.

 
\section{Alignment}  \index{Alignment} \index{Alignment}
 

Alignment is your characters' way of thinking and moral compass. For the character, this is reflected as an ethical ideal, religious strictures, or maybe just a gut instinct. It reflects the things your character might aspire to be and can guide you when you're not sure what to do next. Some characters might proudly proclaim their alignment while others might hide it away. A character might not say "I'm an evil person" but may instead say "I put myself first." That's all well and good for a character, but the world knows otherwise. Buried deep down inside is the ideal self a person wants to become—it is this mystic core that certain spells and abilities tap into when detecting someone's alignment. Every sentient creature in Dungeon World bears an alignment, be she an elf, a human, or some other, stranger thing.

 

The alignments are Good, Lawful, Neutral, Chaotic, and Evil. Each one shows an aspiration to be a different type of person.

 

Lawful creatures aspire to impose order on the world, either for their own benefit or for that of others. Chaotic creatures embrace change and idealize the messy reality of the world, prizing freedom above all else. Good creatures seek to put others before themselves. Evil creatures put themselves first at the expense of others.

 

A Neutral creature looks out for itself so long as that doesn't hurt anyone else much. Neutral characters are content to live their lives and pursue their own goals and let others do the same.

 

Most creatures are Neutral. They take no particular pleasure in harming others, but will do it if it is justified by their situation. Those that put an ideal, be it Law, Chaos, Good, or Evil, above themselves are rarer.

 

Even two creatures of the same alignment can come into conflict. Aspiring to help others does not grant infallibility, two Good creatures may fight and die over two different views of how to help others. A great king may wage war on a free city despite his good alignment since he sees (justly, perhaps) that the peoples of the free city will live a better life under his enlightened rule.

 
\subsection{Changing Alignment}  \index{Changing Alignment} \index{Alignment}
 

Alignment can, and will, change. Usually such a change comes about as a gradual slide to a decisive moment when the pain caused becomes too great or the benefit too others too small.

 

Any time a player would earn XP from their current alignment they can, instead of taking the XP, change the alignment. The player must have a reason for the change which they can explain to the other players. If they can't explain why their character has had a change of heart they can't change alignment. Don't abuse the privilege.

 

The first time a player character changes alignment it must be to another alignment listed for their class, though they can choose any of the alignment moves below for a listed alignment. After that they can go to any alignment they like.

 

In some cases a player character may switch alignment moves while still keeping the same alignment. This reflects a smaller shift, one of priority instead of a wholesale shift in thinking. They simply choose a new move for the same alignment from below and mention why their character now sees this as important.

 

GM characters can change alignment as well, even if the players have already discerned the character's alignment. Since NPCs do not earn XP the GM can change the NPC's alignment any time it's warranted. The GM is subject to the same justification requirement: if an NPC with a known alignment is no longer that alignment the players may ask the GM for a reason why.

 
\subsubsection{Lawful}  \index{Lawful} \index{Lawful}
 
\startitemize[1,packed]

\item Uphold the letter of the law over the spirit

 
\item Fulfill a promise of import

 
\item Bring someone to justice

 
\item Choose honor over personal gain

 
\item Put power in its rightful hands

 
\item Return treasure to its rightful owner


\stopitemize
 
\subsubsection{Good}  \index{Good} \index{Good}
 
\startitemize[1,packed]

\item Ignore danger to aid another

 
\item Lead others into righteous battle

 
\item Give up powers or riches for the greater good

 
\item Reveal a dangerous lie

 
\item Show mercy


\stopitemize
 
\subsubsection{Neutral}  \index{Neutral} \index{Neutral}
 
\subsubsection{Chaotic}  \index{Chaotic} \index{Chaotic}
 
\startitemize[1,packed]

\item Reveal corruption

 
\item Break an unjust law to benefit another

 
\item Defeat a tyrant

 
\item Reveal hypocrisy


\stopitemize
 
\subsubsection{Evil}  \index{Evil} \index{Evil}
 
\startitemize[1,packed]

\item Take advantage of someone's trust

 
\item Cause suffering for its own sake

 
\item Destroy something beautiful

 
\item Upset the rightful order

 
\item Harm an innocent


\stopitemize
 






