\chapter{Monsters}
 \index{Monsters} \index{Monsters}
            

Great heroes need horrendous antagonists. This section is about how to create and play as those antagonists—from the lowly goblin warrior to the hellish demon.

       
\section{Using Monsters}  \index{Using Monsters} \index{Monsters}
       

A monster is any living (or undead) thing that stands in the players' way.

       

How you use these monsters follows directly from your Agenda and Principles. Stay true to your principles, use your moves and pursue your agenda—you can't go wrong. 

       

Your first agenda is to "Make the world fantastic". This shines through strongly based on how you think about monsters. Everyone and everything who comes up against the players is a monster but that doesn't mean you have to write their stats out ahead of time. In a fantastic world, every goblin might end up in a fight but you don't have to know their HP before that happens. A monster is so simple to make you can jump right into the fiction, describing whatever you want and back it up with stats as you need them. Make the world fantastic: describe your monsters first and worry about their stats later.

       

The player characters are the heroes. You shouldn't be rooting for the monsters, per se. Monsters exist to illustrate what a dangerous awful place Dungeon World can be—how it will remain if the players don't step in. If you feel like your monsters are being beaten too quickly, don't worry. Let the players revel in their victory and prepare a bigger, badder follow-up monster for next time.

       

The principle of "Think dangerous" sums up that philosophy—the world is just as dangerous for the monsters as for the characters. An evil overlord doesn't care about his every golem, demon, and harpy. Until proven otherwise, consider every monster an arrow fired at the characters. The monsters are ammunition of the Danger you're presenting. Some may be smarter, faster, or more dangerous than others but until a monster warrants a name, a personality, or some other special consideration, it's an arrow. Take aim and shoot. Don't worry if you miss.

       

A monster stops being an arrow when it is given a chance to shine by the players' actions. When the players are forced to run away from something it gains weight. When a monster somehow survives the players' assault it becomes interesting to the players and to the world at large. The players are the heroes. Your monsters are only important when they become important to the heroes and, thus, important to the world.

       

One thing that your Agenda and Principles don't say anything about is setting up a fair fight. Heroes are often out numbered or faced with ridiculous odds—sometimes they have to retreat and make a new plan. Sometimes they suffer loss. When adding a monster to a front, placing them in a dungeon or making them up on the fly your first responsibility is to the fiction (Make the world fantastic) and to give the characters a real threat (Make the characters heroes), not to make a balanced fight. Dungeon World isn't about balancing encounter levels or counting experience points; it's about telling stories about adventure and death-defying feats!

       
\section{Elements of a Monster}  \index{Elements of a Monster} \index{Elements} \index{Monster}
       

The most important part of a monster is what it does. These are it's {\bf moves} . Just like the normal GM moves, they're things that you do when there's a lull in the action or when the players give you a golden opportunity. Just like the normal GM moves they can be hard or soft depending on the circumstances and the move: a move that's irreversible and immediate is hard, a move that's impending or mitigable is soft.

       

Each monster's {\em raison d'être}  is summed up in its {\bf instinct} . Much like Dangers, monsters have instincts that describe their goals at a high level. Some monsters live for conquest, or treasure, or simply for blood. The monster's instinct is the guide to how to use the monster

       

The monster's {\bf description}  is where all its other features come from. The description is how you know what the monster really is, the other elements just reflect the description.

       

         {\bf Damage}  is a measure of how much pain the monster can inflict at once. Just like player damage it's a dice to roll, maybe with some modifiers. A monster deals its damage to another monster or a player when it uses its standard weapons and tactics to hurt them, or when a move says so.

       

Just like a weapon, monsters have {\bf tags}  that describe how it deals damage, including what range(s) it can do damage at. When trying to attack something it out of it's range (to close or too far) the monster's out of luck, no damage. Any tag that can go on a weapon (like Messy or Slow) can also go on a monster.

       

There are also monster tags that apply only to monsters. These tags, listed below, describe the monster's key attributes. Every monster has a tag for its scope: where it falls in the bigger picture of Dungeon World. The scope tag lets the GM know how to portray the monster in a way that its stats back up, for example if an army of gnolls can take a defended village (hint: most likely). Some monsters also have a size tag, which notes their physical size. Monsters without a size tag are just about human size, give or take.

       

A monster's {\bf HP}  is a measure of how much damage it can take before it dies. Just like players, when a monster takes damage it subtracts that amount from it's HP. At 0 HP it's dead, no Last Breath.

       

Some monsters are lucky enough to enjoy {\bf Armor} . Just like player armor: when a monster with armor takes damage it subtracts its armor from the damage done.

       

         {\bf Special qualities}  describe innate aspects of the monster that are important to play. These are a guide to the fiction, and therefore the moves. There is no master list of special qualities, they're just plain-english descriptions of the qualities of a monster that aren't part of an attack. A quality like "Intangible" means just what it says: mundane stuff just passes through it. That means swinging a mundane sword at it isn't Hack and Slash, for a start.

       
\subsection{Monsters Without Stats}  \index{Monsters Without Stats} \index{Monsters} \index{Stats}
       

Some creatures operate on a scale so far beyond the mortal that concepts like HP, Armor, and Damage just do not hold. These creatures may still cause problems for the players and may even be defeated with clever thinking and enough preparation, they just won't be trading blows.

       

If a creature is of such a scale far beyond the players, or if it simply doesn't live or die like a mortal creature, don't assign it HP, Damage, or Armor. You can still use the monster creation rules to give it tags. The core of a stat-less monster is its instinct and moves; the GM can still make its moves and act according to its instinct.

       
\section{Monster Tags}  \index{Monster Tags} \index{Monster} \index{Tags}
       

         {\em Magical} : It is by nature magical through and through.

       

         {\em Devious} : Its main danger lies beyond the simple clash of battle.

       

         {\em Gibbous} : Its anatomy and organs are bizarre and unnatural.

       

         {\em Organized} : It has a group structure that aids it in survival. Defeating one may cause the wrath of others. One may sound an alarm.

       

         {\em Intelligent} : Its smart enough that some individuals pick up other skills. The GM can adapt the monster by adding tags to reflect specific training, like a mage or warrior.

       

         {\em Hoarder} : It almost certainly has treasure.

       

         {\em Stealthy} : It can avoid detection and prefers to attack with the element of surprise.

       

         {\em Terrifying} : Its presence and appearance evoke fear.

       

         {\em Cautious} : It prizes survival over aggression.

       

         {\em Construct} : It was made, not born

       

         {\em Planar} : Its from beyond this world

       
\subsection{Organization Tags}  \index{Organization Tags} \index{Organization} \index{Tags}
       

         {\em Horde} : Where there's one, theres more. A lot more.

       

         {\em Group} : Usually seen in small numbers, 3–6 or so.

       

         {\em Solitary} : It lives and fights alone.

       
\subsection{Size Tags}  \index{Size Tags} \index{Size} \index{Tags}
       

         {\em Tiny} : It's much smaller than a halfling.

       

         {\em Small} : It's about halfling size.

       

         {\em Big} : It's much bigger than a human, about as big as a cart.

       

         {\em Huge} : It's as big as a small house or larger.

       
\section{Making Monsters}  \index{Making Monsters} \index{Monsters}
       

Monsters start with your description of them. No matter if you're making the monster before play or just as the players come face-to-face with it a monster starts with a clear vision of what it is and what it does.

       

If you're making a monster between sessions start be imagining it. Imagine what it looks like, what it does, why it stands out. Imagine the stories told about it and what effects it has had on the world.

       

If you're making a monster on the fly during a session start by describing it to the players. Your description starts before the characters even lay eyes on it: describe where it lives, what marks it has made on the environment around it. Your description is the key to the monster.

       

When you find you need stats for the monster you use this series of questions to establish them. Answer every question based on the facts established and imagined. Don't answer them aloud to anyone else, just note down the answers and the stats listed with each answer.

       

If two questions would grant the same tag don't worry about it. If you like you can adjust damage or HP by 2 to reflect the tag that would be repeated, but it's not necessary. If a combination of answers would reduce HP or damage below 1 they stay at 1.

       

When you're finished your monster may have only one move. If this is the case and you plan on using the monster often, give it another 2–3 moves of your choice. These moves often describe secondary modes of attack, other uses for a primary mode of attack, or connections to a certain place in the world.

       
\subsubsection{What is it known to do?}  \index{What is it known to do}
       

Write a monster move describing what it does.

       
\subsubsection{What does it want that causes problems for others?}  \index{What does it want that causes problems for others} \index{Problems}
       

This is its instinct. Write it as an intended action.

       
\subsubsection{How does it usually hunt or fight?}  \index{How does it usually hunt or fight} \index{Hunt} \index{Fight}
       
\startitemize[1,packed]
         
\item In large disorganized groups: Horde, d6 damage, 3 HP

         
\item In small groups, about 2–5: Group, d8 damage, 6 HP

         
\item All by its lonesome: Solitary, d10 damage, 12 HP

       
\stopitemize
       
\subsubsection{How big is it?}  \index{How big is it} \index{Big}
       
\startitemize[1,packed]
         
\item Smaller than a house cat: Tiny, Hand, -2 damage

         
\item Halfling-esque: Small, Close

         
\item About human size: Close

         
\item As big as a cart: Large, Close, Reach, +4 HP, +1 damage

         
\item Much larger than a cart: Huge, Reach, +8 HP, +3 damage

       
\stopitemize
       
\subsubsection{What is its most important defense?}  \index{What is its most important defense} \index{Important} \index{Defense}
       
\startitemize[1,packed]
         
\item Cloth or flesh: 0 armor

         
\item Leathers or think hide: 1 armor

         
\item Mail or scales: 2 armor

         
\item Plate or bone: 3 armor

         
\item Permanent magical protection: 4 armor, Magical

       
\stopitemize
       
\subsubsection{What is it known for? (Choose all that apply)}  \index{What is it known for (Choose all that apply)} \index{(choose} \index{Apply)}
       
\startitemize[1,packed]
         
\item Unrelenting strength: +2 damage, forceful

         
\item Skill in offense: roll damage twice and take the best

         
\item Skill in defense: +1 armor

         
\item Deft strikes: +1 piercing

         
\item Uncanny endurance: +4 HP

         
\item Deceit and trickery: Stealthy, write a move about dirty tricks

         
\item A useful adaptation like being amphibious or having wings: Add a special quality for the adaptation

         
\item The favor of the gods: Divine, +2 damage or +2 HP or both (your call)

         
\item Spells and magic: Magical, write a move about its spells

       
\stopitemize
       
\subsubsection{What is its most common form of attack?}  \index{What is its most common form of attack} \index{Common} \index{Form} \index{Attack}
       

Note it along with the creature's damage. Common answers include: a type of weapon, claws, a specific spell. Then answer these questions about it.

       
\startitemize[1,packed]
         
\item Its armaments are vicious and obvious: +2 damage

         
\item It lets the monster keep others at bay: Reach

         
\item Its armaments are small and weak: -damage dice size

         
\item Its armaments can slice or pierce metal: Messy, +1 Piercing or +3 Piercing if it can just tear metal apart

         
\item Armor doesn't help with the damage it deals (due to magic, size, etc.): Ignores Armor

         
\item It usually attacks at range (with arrows, spells, or other projectiles): Near or Far or both (your call)

       
\stopitemize
       
\subsubsection{Which of these describe it? (Choose all that apply)}  \index{Which of these describe it (Choose all that apply)} \index{Describe} \index{(choose} \index{Apply)}
       
\startitemize[1,packed]
         
\item It isn't dangerous because of the wounds it inflicts, but for other reasons: Devious, -damage dice size, write a move about why it's dangerous

         
\item It organizes into larger groups that it can call on for support: Organized, write a move about calling on others for help

         
\item It's as smart as a human or thereabouts: Intelligent

         
\item It actively defends itself with a shield or similar: Cautious, +1 Armor

         
\item It collects trinkets that humans would consider valuable (gold, gems, secrets): Hoarder

         
\item It uses magic: Magical, write a move about its style of magic and the effects it can invoke

         
\item It's from beyond this world: Planar, write a move about using it's otherworldly knowledge and power

         
\item It's kept alive by something beyond simple biology: +4 HP

         
\item It was made by someone: Construct, give it a special quality or two about its construction or purpose

         
\item Its appearance is disturbing, terrible, or horrible: Terrifying, write a special quality about why it's so horrendous

         
\item It doesn't have organs or discernible anatomy: Gibbous, +1 Armor, +3 HP

       
\stopitemize
       
\section{Treasure}  \index{Treasure} \index{Treasure}
       

Monsters, much like adventurers, collect shiny useful things. When the players search the belongings of a monster (be they on their person or tucked away somewhere) describe them honestly.

       

If the monster has accumulated some wealth you can roll that randomly. Start with the monster's damage die, modified if the monster is:

       
\startitemize[1,packed]
         
\item Hoarder: roll damage die twice, take higher result

         
\item Far from home: +at least one ration (usable by anyone with similar taste)

         
\item Magical: some strange item, possibly magical

         
\item Divine: a sign of a deity (or deities)

         
\item Planar: something not of this earth

         
\item Lord over others: +1d4 to the roll

         
\item Ancient and noteworthy: +1d4 to the roll

       
\stopitemize
       

Roll the monster's damage die plus any added dice to fine the monster's treasure:

       
\startitemize[n,packed]
         
\item A few coins, 2d8 or so

         
\item An item useful to the current situation

         
\item Several coins, about 4d10

         
\item A small item (gem, art) of considerable value, worth as much as 2d10×10 coins, 0 weight

         
\item Some minor magical trinket

         
\item Useful information (in the form of clues, notes, etc.)

         
\item A bag of coins, 1d4×100 or thereabouts. 1 weight per 100.

         
\item A very valuable small item (gem, art) worth 2d6×100, 0 weight

         
\item A chest of coins and other small valuables. 1 weight but worth 3d6×100 coins

         
\item A magical item or magical effect

         
\item Many bags of coins for a total of 2d4×100 or so

         
\item A sign of office (crown, banner) worth at least 3d4×100 coins

         
\item A large art item worth 4d4×100 coins, 1 weight

         
\item A unique item worth at least 5d4×100 coins

         
\item All the information needed to learn a new spell and roll again

         
\item A portal or secret path (or directions to one) and roll again

         
\item Something relating to one of the characters and roll again

         
\item A hoard: 1d10×1000 coins and 1d10×10 gems worth 2d6×100 each

       
\stopitemize
       
\section{Monster Settings}  \index{Monster Settings} \index{Monster} \index{Settings}
       

The monsters in this book are presented in {\bf monster settings} . A monster setting is a location (or type of location) and the monsters that inhabit it. It's a way of grouping monsters by where they fit in the world. A monster setting tells you what kind of monsters might inhabit an area while your Fronts tell you what monsters are working together or have ongoing plots.

       

When creating your own monster settings, they can be more specific. You could create a monster setting for the Great Western Steppes or the Domains of the Horse Lords.

       

Consult a monster setting to populate a Front or when you want a threat that is only tangentially related to one of your Fronts. For example, if the heroes are battling against the Dungeon Front "The Cult of Khul-ka-ra" by exploring the ancient ruins that the cult has made its home then you might use monsters from the Legions of the Undead as a related threat—not truly part of the front but still a block in the heroes' path.

       

Monsters within a given setting will tend to be about as powerful. This is a product of their ecology—they're in competition for space and resources, after all. Cavern Dwellers or Denizens of the Murky Swamp are likely to be faced by fresh adventurers as they are the creatures who most often encroach on civilization. The Gnarled Woods, Ravenous Hordes, and Twisted Experiments settings hold more powerful monsters, monsters that can threaten the safety of whole cities. Planar Powers and Creatures of the Lower Depths are the most dangerous enemies the heroes can face—often endangering entire kingdoms. The Legions of the Undead are everywhere and can appear in just about any setting or situation.

       

The monster stat blocks within these settings describe not only, HP, damage, and all the other aspects of the monster, but also the reasons those stats were assigned. These monsters were created with the same process listed above and the reasons for their stats are just as important as the stats themselves. Looking at the reasoning behind the stats will allow you to present the monsters honestly, answering questions that arise in Dungeon World like "can a warband of gnolls sack an entire village?"

       
\subsection{Cavern Dwellers}  \index{Cavern Dwellers} \index{Cavern} \index{Dwellers}
       

At the edges of civilization in the caves and tunnels below the old mountains of the world dwell all sorts of scheming, dangerous monsters. Some are wily and old, like the race of goblins scheming to burn villages and make off with livestock. Others are strange aberrations of nature like the stinking, trash-eating Otyugh. A word of caution, then, to those brave adventurers whose first foray into danger leads them into these dank and shadowy places; bad things live in the dark. Bad things with sharp teeth.

       
\subsection{Denizens of the Swamp}  \index{Denizens of the Swamp} \index{Denizens} \index{Swamp}
       

All things give way to rot, in the end. Food spoils on the table, men's minds go mad with age and disease. Even the world itself, when left untended and uncared for, can turn to black muck and stinking air. Things dwell in these parts of Dungeon World. Things gone just as a bad as the swirling filth that fills the swamps. In these cesspit lowlands adventurers will find such creatures as the deadly-eyed basilisk or the famed, unkillable troll. You'll need more than a dry pair of boots to survive these putrid fens. A sword would be a good start.

       
\subsection{Legions of the Undead}  \index{Legions of the Undead} \index{Legions} \index{Undead}
       

The sermons of mannish and dwarven gods would tell you that Death is the end of all. They say that once the mortal coil is unwound and a person takes their final breath that all is warmth and song and the white wings of angels. Not so. Not for all. For some, after life's embrace loses its strength a darker power can take hold. Black magic rips the dead from the ground and gives them shambling unlife full of hate and hunger. Sorcery and witchcraft lend an ancient spell-smith the power to live forever in the husk of a Lich. There are bleak enchantments at play in shadowy corners all throughout Dungeon World. These creatures are the spawn of that fell magic.

       
\subsection{The Dark Woods}  \index{The Dark Woods} \index{Dark} \index{Woods}
       

It would not be a lie to say that there are trees that stand in the deepest groves of Dungeon World that have stood since before man or elf walked amidst their roots. It would be true, too, to say that these ancient trees have long lost the green leaves of spring. In the strands of the dark woods one finds, if one looks in the right place, sylvan monsters of old and powerful nature. Here live the race of savage Centaurs and the fey soul-stealing creatures of olde. Under the shadow of the ancient trees, wolf-men howl for blood. Hurry along the old forest road and light no fire for food or warmth for it's said that flames offend the woods themselves. You wouldn't want that, would you?

       
\subsection{Ravenous Hordes}  \index{Ravenous Hordes} \index{Ravenous} \index{Hordes}
       

"I've bested an orc in single combat" they crow. "I've fought a gnoll and lived to the tell the tale." Which is no small feat and yet, you know the truth of these boasts. Like vermin, spotting but one of these creatures speaks to a greater doom on the horizon. No orc travels alone. No slavering gnoll moves without his pack. You know that soon, the wardrums will sound and the walls will be besieged by the full fury of the warchief and his tusked berserkers. These are the monsters that will bring civilization, screaming and weeping, to its knees. Unless you can stop them. Best of luck.

       
\subsection{Twisted Experiments}  \index{Twisted Experiments} \index{Twisted} \index{Experiments}
       

For some who learn the arcane arts it's not merely enough to be able to live for a thousand years or throw lightning bolts that can fry a man. Some aren't quite satisfied with the power to speak to the dead or draw the angels down from heaven. Hubris calls on those cloaked-and-hooded "scientists" to make a strange and unholy life of their own. No mortal children, these. These are the wages of a mind gone foul with strange magic. In this setting you will find such nightmares as the chimera, dripping poison. Here, too, are the protector golems and mutant apes. All sorts of bad ideas await you in the fallen towers of the mad magicians of Dungeon World.

       
\subsection{The Lower Depths}  \index{The Lower Depths} \index{Depths}
       

Ruins dot the countryside of Dungeon World. Old bastions of long-forgotten civilization fallen to decay, to monsters or to the whim of a vengeful god. These ruins often cover a much more dangerous truth – catacombs and underground complexes lousy with traps and monsters. Gold, too. Which is why you're here. Why you're locked in mortal combat with a tribe of spiteful dark elves. Battling stone giants in caverns the size of whole countries. Maybe, though, you're the noble souls who've travelled to the world's heart to put an end to the Apocalypse Dragon—the beast who, it is said, will one day swallow the sun and kill us all. We appreciate it, really. We'll all pray for you.

       
\subsection{Planar Powers}  \index{Planar Powers} \index{Planar} \index{Powers}
       

Sometimes, monsters do not come from Dungeon World at all. Beyond the mountains at the edge of the world or below the deepest seas the sages and wise old priests say that there are gateways to the lands beyond. They speak of elysian fields; rivers of sweet wine and maidens dancing in fields of gold. They tell tales of the paradise of heavens to be found past the Planar Door. Tales tell, too, of the Thousandfold Hell. Of the swirling Elemental Vortex and the devils that wait for the stars to align so they can enter Dungeon World and wreak their bloody havoc. You must be curious to know if these tales are true? What will you see when the passage to the beyond is opened?

                
\section{Cavern Dwellers}  \index{Cavern Dwellers} \index{Cavern} \index{Dwellers}
            
\startMonsterName
Ankheg	\CMTags{Group, Large} 
\stopMonsterName
       

Bite (d8+1 damage)	10 HP	3 Armor

       

         \CMTags{Close, Reach}        

       
\startMonsterQualities
         {\bf Special Qualities:}  Burrowing
\stopMonsterQualities
       
\startMonsterDescription
A hide like plate armor and great crushing mandibles are problematic. A stomach full of acid that can burn a hole through a stone wall makes them all the worse. They’d be bad enough if they were proper insect-sized but these things have the gall to be as long as any given horse. It’s just not natural! Good thing they tend to stick to one place? Easy for you to say—you don’t have an Ankheg living under your corn field. {\em Instinct} : To undermine
\stopMonsterDescription
       
\startitemize[1,packed]
         
\item Undermine the ground

         
\item Burst from the earth

         
\item Spray forth acid, eating away at metal and flesh

       
\stopitemize
       
\startMonsterName
Cave Rat	\CMTags{Horde, Small} 
\stopMonsterName
       

Gnaw (d6 damage 1 piercing)	7 HP	1 Armor

       

         \CMTags{Close, Messy}        

       
\startMonsterDescription
Who hasn’t seen a rat before? It’s like that, but nasty and big and not afraid of you anymore. Maybe this one was a cousin to that one you caught in a trap or the one you killed with a knife in that filthy tavern in Darrow. Maybe he’s looking for a little ratty revenge. {\em Instinct} : To devour
\stopMonsterDescription
       
\startitemize[1,packed]
         
\item Swarm

         
\item Rip apart something (or someone)

       
\stopitemize
       
\startMonsterName
Choker	\CMTags{Solitary, Stealthy, Intelligent} 
\stopMonsterName
       

Choke (d10 damage)	15 HP	2 Armor

       

         \CMTags{Close, Reach}        

       
\startMonsterQualities
         {\bf Special Qualities:}  Flexible
\stopMonsterQualities
       
\startMonsterDescription
Some say these things descended from the family of a cruel wizard who forced them to live out their lives underground. Say his experiments led him to fear the sun and ages passed while he descended into unlife, dragging his folk along with him. These things resemble men, in a way. Head, four limbs and all that. Only their skin is wet and rubbery and their arms long and fingers grasping. They hate all life that bears the stink of the sun’s touch, as one might expect. Jealousy, long instilled, is hard to shake. {\em Instinct} : To deny light
\stopMonsterDescription
       
\startitemize[1,packed]
         
\item Hold someone, wringing the breath from them

         
\item Fling a held creature

       
\stopitemize
       
\startMonsterName
Cloaker	\CMTags{Solitary, Stealthy} 
\stopMonsterName
       

Constrict (d10 damage ignores armor)	12 HP	1 Armor

       

         \CMTags{Close}        

       
\startMonsterQualities
         {\bf Special Qualities:}  Looks like a cloak
\stopMonsterQualities
       
\startMonsterDescription
Don’t put on that cloak, Gareth. Don’t. You don’t know where it’s been. I tell you, it’s no good. See! It moved! I’m not mad, Gareth, it moved! Don’t do it! No! GARETH! {\em Instinct} : To engulf
\stopMonsterDescription
       
\startitemize[1,packed]
         
\item Engulf the unsuspecting

       
\stopitemize
       
\startMonsterName
Dwarven Warrior	\CMTags{Horde, Organized} 
\stopMonsterName
       

Axe (d6 damage)	7 HP	2 Armor

       

         \CMTags{Close}        

       
\startMonsterDescription
For ages, men believed all dwarves were men and all were of this ilk—stoic and proud warriors. Axe-wielding and plate-wearing. Stout bearded battle-hungry men who would push them, time and time again, back up out of their mines and tunnels with ferocity. It just goes to show how little men know about the elder races. These folk are merely a vanguard, and they bravely do their duty to protect the riches of the Dwarven realm. Earn their trust and you’ve an ally for life. Earn their ire and you’re not like to regret it very long. {\em Instinct} : To defend
\stopMonsterDescription
       
\startitemize[1,packed]
         
\item Drive them back

         
\item Call up reinforcements

       
\stopitemize
       
\startMonsterName
Earth Elemental	\CMTags{Solitary, Huge} 
\stopMonsterName
       

Smash (d10+5 damage)	27 HP	4 Armor

       

         \CMTags{Reach, Forceful}        

       
\startMonsterQualities
         {\bf Special Qualities:}  Made of stone
\stopMonsterQualities
       
\startMonsterDescription
Our shaman says that all the things of the world have a spirit. Stones, trees, a stream. Now that I’ve seen the earth roil under my feet and fists of stone beat my friends half to death I’m like to believe that crazy old man. The one I saw was huge—big as a house! It came boiling up from a rockslide out of nowhere and had a voice like an avalanche. I pay my respects, now. Rightly so. {\em Instinct} : To show the strength of earth
\stopMonsterDescription
       
\startitemize[1,packed]
         
\item Turn the ground into a weapon

         
\item Meld into stone

       
\stopitemize
       
\startMonsterName
Goliath	\CMTags{Group, Huge, Organized, Intelligent} 
\stopMonsterName
       

Mace (d8+7 damage)	14 HP	1 Armor

       

         \CMTags{Reach, Forceful}        

       
\startMonsterDescription
They dwell beneath the earth because they do not belong above it any longer. An undying race of mighty titans fled the plains and mountains in ages past—driven out by men and their heroes. Left to bide their time in the dark, hate and anger warmed by the pools of lava deep below. It’s said that an earthquake is a goliath’s birthing cry. Some day they’ll take back what’s theirs. {\em Instinct} : To retake
\stopMonsterDescription
       
\startitemize[1,packed]
         
\item Shake the earth

         
\item Retreat, only to come back stronger

       
\stopitemize
       
\startMonsterName
Fire Beetle	\CMTags{Horde, Small} 
\stopMonsterName
       

Flames (d6 damage ignores armor)	3 HP	3 Armor

       

         \CMTags{Near}        

       
\startMonsterQualities
         {\bf Special Qualities:}  Full of flames
\stopMonsterQualities
       
\startMonsterDescription
Scarabaeus Pyractomena! What a delightful creature—see how its carapace glitters in the light of our torches? Not too close now, they’re temperamental, you see. The fire in their belly isn’t just metaphorical, no. Watch as I goad the beast. Aha! A spout of flame! Unexpected, isn’t it? One of these creatures alone, if it comes up from below, can be a hellish nuisance to a farmstead or village. A whole swarm? There’s a reason they call it a conflagration of fire beetles. I’ll say that much. {\em Instinct} : To enflame
\stopMonsterDescription
       
\startitemize[1,packed]
         
\item Undermine the ground

         
\item Burst from the earth

         
\item Spray forth acid, eating away at metal and flesh

       
\stopitemize
       
\startMonsterName
Gargoyle	\CMTags{Horde, Stealthy, Hoarder} 
\stopMonsterName
       

Claw (d6 damage)	3 HP	2 Armor

       

         \CMTags{Close}        

       
\startMonsterQualities
         {\bf Special Qualities:}  Wings
\stopMonsterQualities
       
\startMonsterDescription
It’s a sad thing, really. Guardians bred by Magi of the past with no more castles to guard. Their ancestors’ sacred task bred into their blood leads them to find a place—ruins, mostly but sometimes a cave or hill or mountain cliff—and guard it as though their masters yet lived below. They’re notoriously good at finding valuables buried below the earth, though. Find one of these winged reptiles and you’ll find yourself a treasure nearby. Just be careful, they’re hard to spot and tend to move in packs. {\em Instinct} : To guard
\stopMonsterDescription
       
\startitemize[1,packed]
         
\item Attack with the element of surprise

         
\item Take to the air

         
\item Blend into stonework

       
\stopitemize
       
\startMonsterName
Gelatinous Cube	\CMTags{Solitary, Large, Stealthy, Gibbous} 
\stopMonsterName
       

Engulf (d10+1 damage ignores armor)	20 HP	1 Armor

       

         \CMTags{Hand}        

       
\startMonsterQualities
         {\bf Special Qualities:}  Transparent
\stopMonsterQualities
       
\startMonsterDescription
How many adventurers’ last thought was "strange, this tunnel seems cleaner than most?" Too many and all because of this transparent menace. A great acidic blob that expands to fill a small chamber or corridor and then slides, ever so slowly along, eating everything in its path. It cannot eat stone or metal and will often have them floating in its jelly mass. Blech. {\em Instinct} : To clean
\stopMonsterDescription
       
\startitemize[1,packed]
         
\item Fill an apparently empty space

         
\item Dissolve

       
\stopitemize
       
\startMonsterName
Goblin	\CMTags{Horde, Small, Intelligent, Organized} 
\stopMonsterName
       

Spear (d6 damage)	3 HP	1 Armor

       

         \CMTags{Close, Reach}        

       
\startMonsterDescription
Nobody seems to know where these things came from. Elves say they’re the Dwarves’ fault—dredged up from a hidden place beneath the earth. Dwarves say they’re bad Elvish children, taken away at birth and raised in the dark. The truth of the matter is that goblins have always been here and they’ll be here once all the civilized races have fallen and gone away. Goblins never die out. There’s just too damn many of them. {\em Instinct} : To multiply
\stopMonsterDescription
       
\startitemize[1,packed]
         
\item Charge!

         
\item Call more goblins

         
\item Retreat and return with (many) more

       
\stopitemize
       
\startMonsterName
Goblin Orkaster	\CMTags{Solitary, Small, Magical, Intelligent, Organized} 
\stopMonsterName
       

Acid orb (d10+1 damage ignores armor)	12 HP	0 Armor

       

         \CMTags{Near, Far}        

       
\startMonsterDescription
Oh lord, who taught them magic? {\em Instinct} : To tap power beyond their stature
\stopMonsterDescription
       
\startitemize[1,packed]
         
\item Unleash a poorly understood spell

         
\item Pour forth magical chaos

         
\item Use other goblins for shields

       
\stopitemize
       
\startMonsterName
Otyugh	\CMTags{Solitary, Large} 
\stopMonsterName
       

Tentacles (d10+3 damage)	20 HP	1 Armor

       

         \CMTags{Close, Reach, Forceful}        

       
\startMonsterQualities
         {\bf Special Qualities:}  Filth Fever
\stopMonsterQualities
       
\startMonsterDescription
The mating call of the otyugh is a horrible, blaring cry that sounds like a cross between an elephant dying and an over-eager vulture. The otyugh spends much of its time partly submerged in filthy water and prefers eating garbage over any other food. As a result, it often grows fat and strong on the offal of orcs, goblins and other cave-dwelling sub-humans. Get too close, however, and you’ll have one of its barbed tentacles dragging you into that soggy, razor-toothed maw. If you get away with your life, best get to a doctor, or your victory may be short lived. {\em Instinct} : To foul
\stopMonsterDescription
       
\startitemize[1,packed]
         
\item Infect someone with filth fever

         
\item Fling someone or something

       
\stopitemize
       
\startMonsterName
Maggot-Squid	\CMTags{Horde, Small} 
\stopMonsterName
       

Chew (d6 damage)	3 HP	1 Armor

       

         \CMTags{Close}        

       
\startMonsterQualities
         {\bf Special Qualities:}  Amphibious, Paralyzing Tentacles
\stopMonsterQualities
       
\startMonsterDescription
The gods that made this thing were playing some sick joke on the civilized folk of the world. The maggot-squid wields a face full of horrible squirming tentacles that, if they touch you, feel like being struck by lightning. They’ll paralyze you and chew you up slowly while you’re helpless. Best to not let it get to that. {\em Instinct} : To eat
\stopMonsterDescription
       
\startitemize[1,packed]
         
\item Paralyze with a touch

       
\stopitemize
       
\startMonsterName
Purple Worm	\CMTags{Solitary, Huge} 
\stopMonsterName
       

Bite (d10+5 damage)	20 HP	2 Armor

       

         \CMTags{Reach, Forceful}        

       
\startMonsterQualities
         {\bf Special Qualities:}  Burrowing
\stopMonsterQualities
       
\startMonsterDescription
Iä! Iä! The Purple Worm! Blessed is its holy slime! We walk, unworthy, in its miles of massive tunnels. We are but shadows under its violet and all-consuming glory. Mere acolytes, we who hope someday to return to the great embrace of its tooth-ringed maw. Let it consume us! Let it eat our homes and villages so that we might be taken! Iä! Iä! The Purple Worm! {\em Instinct} : To consume
\stopMonsterDescription
       
\startitemize[1,packed]
         
\item Swallow whole

         
\item Tunnel through stone and earth

       
\stopitemize
       
\startMonsterName
Roper	\CMTags{Solitary, Large, Stealthy, Intelligent} 
\stopMonsterName
       

Bite (d10+1 damage)	16 HP	1 Armor

       

         \CMTags{Close, Reach}        

       
\startMonsterQualities
         {\bf Special Qualities:}  Rock-like Flesh
\stopMonsterQualities
       
\startMonsterDescription
Evolutionary happenstance has created a clever underground predator. Disguised as a rocky formation—most often a stalactite or stalagmite—the roper waits for its prey to wander by. When it does, whether it’s a rat, a goblin or a foolhardy adventurer, a mass of thin, whipping tentacles erupts from the thing’s hide. A hundred lashes in the blink of an eye and the stunned prey is being dragged into the roper’s mouth. Surprisingly effective for a thing that looks like a rock. {\em Instinct} : To ambush
\stopMonsterDescription
       
\startitemize[1,packed]
         
\item Ensnare the unsuspecting

         
\item Burst from the earth

         
\item Spray forth acid, eating away at metal and flesh

       
\stopitemize
       
\startMonsterName
Rot Grub	\CMTags{Horde, Tiny} 
\stopMonsterName
       

Burrow (d6-2 damage)	3 HP	0 Armor

       

         \CMTags{Hand}        

       
\startMonsterQualities
         {\bf Special Qualities:}  Burrow into flesh
\stopMonsterQualities
       
\startMonsterDescription
They live in your skin. Or your organ meat. Or your eyeballs. They grow there and then, in a bloody and horrific display, burrow their way out. Disgusting. {\em Instinct} : To infect
\stopMonsterDescription
       
\startitemize[1,packed]
         
\item Burrow under flesh

         
\item Lay eggs

         
\item Burst forth from an infected creature

       
\stopitemize
       
\startMonsterName
Spiderlord	\CMTags{Solitary, Large, Devious, Intelligent} 
\stopMonsterName
       

Mandibles (d8+4 damage)	16 HP	3 Armor

       

         \CMTags{Close, Reach}        

       
\startMonsterQualities
         {\bf Special Qualities:}  Burrowing
\stopMonsterQualities
       
\startMonsterDescription
Even spiders have their gods, whispered to in webs with little praying arms. {\em Instinct} : To weave webs (literal and metaphorical)
\stopMonsterDescription
       
\startitemize[1,packed]
         
\item Enmesh in webbing

         
\item Put a plot into motion

       
\stopitemize
       
\startMonsterName
Troglodyte	\CMTags{Group, Organized} 
\stopMonsterName
       

Club (d8 damage)	10 HP	1 Armor

       

         \CMTags{Close}        

       
\startMonsterDescription
Long-forgotten, our last remaining ancestors dwell in caves in the wild parts of the world. Driven away by our cities and villages, our iron swords and our fire, these ape-men eat their meat raw with sharp-nailed hands and jagged teeth. They strike out at frontier villages wielding clubs and in overwhelming numbers to seize cattle, tools, and poor prisoners to drag into the hills. Known for their viciousness and their stink, they’re an old and dying race we’d all sooner forget existed. {\em Instinct} : To prey on civilization
\stopMonsterDescription
       
\startitemize[1,packed]
         
\item Raid and retreat

         
\item Use scavenged weapons or magic

       
\stopitemize
                
\section{Denizens of the Murky Swamp}  \index{Denizens of the Murky Swamp} \index{Denizens} \index{Murky} \index{Swamp}
            
\startMonsterName
Bakunawa	\CMTags{Solitary, Large, Intelligent, Messy, Forceful} 
\stopMonsterName
       

Bite (d10+3 damage 1 piercing)	16 HP	2 Armor

       

         \CMTags{Close, Reach}        

       
\startMonsterQualities
         {\bf Special Qualities:}  Amphibious
\stopMonsterQualities
       
\startMonsterDescription
Dragon-turtle’s sister is a mighty serpent queen. Ten yards of scales and muscle, they say she wakes with a hunger when the sun disappears from the sky. She is drawn in by bright light in the darkness and like any snake, the Bakunawa is sneaky. She will seek first to beguile and mislead and will only strike out with violence when no other option is available. When she does, though, her jaws are strong enough to crack the hull of any swamp-boat and certainly enough to slice through a steel breastplate or two. {\em Instinct} : To devour
\stopMonsterDescription
       
\startitemize[1,packed]
         
\item Lure prey with lies and illusions

         
\item Lash out at light

         
\item Devour

       
\stopitemize
       
\startMonsterName
Basilisk	\CMTags{Solitary, Hoarder} 
\stopMonsterName
       

Bite (d10 damage)	12 HP	2 Armor

       

         \CMTags{Close}        

       
\startMonsterDescription
Few have seen a basilisk and lived to tell the tale. Get it? Seen a basilisk? Little bit of basilisk humor there. Sorry, I know you’re looking for something helpful, sirs. Serious stuff, I understand. The basilisk, even without its ability to turn your flesh to stone with a gaze, is a dangerous creature. A bit like a frog, bulbous eyes and six leaping-muscle legs. A bit like an alligator, with snapping jaws and sawing teeth. Covered in stoney scales and very hard to kill. Best avoided, if possible. {\em Instinct} : To collect stone
\stopMonsterDescription
       
\startitemize[1,packed]
         
\item Turn flesh to stone with a gaze

         
\item Retreat into a maze of stone

       
\stopitemize
       
\startMonsterName
Black Pudding	\CMTags{Solitary, Gibbous} 
\stopMonsterName
       

Corrode (d10 damage ignores armor)	15 HP	1 Armor

       

         \CMTags{Close}        

       
\startMonsterQualities
         {\bf Special Qualities:}  Amorphous
\stopMonsterQualities
       
\startMonsterDescription
How do you kill a pile of goo? A great squishy, pile of goo that happens also to want to dissolve you and slurp you up? That is a good question to which I have no answer. Do let us know when you find out. {\em Instinct} : To dissolve
\stopMonsterDescription
       
\startitemize[1,packed]
         
\item Eat away metal, flesh, or wood

         
\item Ooze into a troubling place: food, armor, stomach

       
\stopitemize
       
\startMonsterName
Coutal	\CMTags{Solitary, Intelligent Devious} 
\stopMonsterName
       

Light ray (d8 damage ignores armor)	12 HP	2 Armor

       

         \CMTags{Close}        

       
\startMonsterQualities
         {\bf Special Qualities:}  Wings, Halo
\stopMonsterQualities
       
\startMonsterDescription
As if in direct affront to the decay and filth of the world, the gods granted us the Coutal. As if to say “there is beauty, even in this grim place” have we been gifted the Coutal. A serpent in flight on jeweled wings, these beautiful creatures glow with a soft light, as the sun does through stained glass. Bright, wise and calm, a Coutal often knows many things and sees many more. You might be able to make a trade with it in exchange for some favor. They seek to cleanse and to purge and to make of this dark world a better one. Shame we have so few. The gods are cruel. {\em Instinct} : To cleanse
\stopMonsterDescription
       
\startitemize[1,packed]
         
\item Pass judgement on a person or place

         
\item Summon divine forces to cleanse

         
\item Offer information in exchange for service

       
\stopitemize
       
\startMonsterName
crocodilian	\CMTags{Group, Large} 
\stopMonsterName
       

Bite (d8+3 damage)	10 HP	2 Armor

       

         \CMTags{Close, Reach}        

       
\startMonsterQualities
         {\bf Special Qualities:}  Amphibious, Camouflage
\stopMonsterQualities
       
\startMonsterDescription
It’s a really, really big crocodile. Seriously. So big. {\em Instinct} : To eat
\stopMonsterDescription
       
\startitemize[1,packed]
         
\item Attack an unsuspecting victim

         
\item Escape into the water

         
\item Hold something tight in its jaws

       
\stopitemize
       
\startMonsterName
Doppelganger	\CMTags{Solitary, Devious, Intelligent} 
\stopMonsterName
       

Dagger (d6 damage)	12 HP	0 Armor

       

         \CMTags{Close}        

       
\startMonsterQualities
         {\bf Special Qualities:}  Shapeshifting
\stopMonsterQualities
       
\startMonsterDescription
Their natural form, if you ever see it, is a hideous thing. Like a creature who stopped growing part-way, before it decided it was elf or man or dwarf. Then again, maybe that’s how you get to be the way a doppleganger is—without form, without shape to call their own, they seek little more than a place to fit in. If you go out into the world, when you come back home, make sure your friends are who you think they are. They might, instead, be a doppleganger and your friend might be dead at the bottom of a well somewhere. Then again, depending on your friends, that might be an improvement. {\em Instinct} : To infiltrate
\stopMonsterDescription
       
\startitemize[1,packed]
         
\item Assume the shape of a person who's flesh it's tasted

         
\item Use another's identity to advantage

         
\item Leave someone's reputation shattered

       
\stopitemize
       
\startMonsterName
Dragon Turtle	\CMTags{Solitary, Huge, Cautious} 
\stopMonsterName
       

Bite (d10+3 damage)	20 HP	4 Armor

       

         \CMTags{Reach}        

       
\startMonsterQualities
         {\bf Special Qualities:}  Shell, Amphibious
\stopMonsterQualities
       
\startMonsterDescription
Bakunawa has a brother. Where she is quick to anger and hungry for gold, he is slow and sturdy. She is a knife and he is a shield. A great turtle that lies in the muck and mire for ages as they pass, mud piled upon his back—sometimes trees and shrubs. Sometimes a whole misguided clan of goblins will build their huts and cook their ratty meals on the shell of the dragon turtle. His snapping jaws, glacier-slow they may be, can rend a castle wall. Careful where you tread. {\em Instinct} : To resist change
\stopMonsterDescription
       
\startitemize[1,packed]
         
\item Move forward implacably

         
\item Bring its full bulk to bear

         
\item Destroy structures and buildings

       
\stopitemize
       
\startMonsterName
Dragon Whelp	\CMTags{Solitary, Small, Intelligent, Cautious, Hoarder} 
\stopMonsterName
       

Elemental breath (d10+2 damage)	16 HP	3 Armor

       

         \CMTags{Close, Near}        

       
\startMonsterQualities
         {\bf Special Qualities:}  Wings, Elemental Blood
\stopMonsterQualities
       
\startMonsterDescription
What? Did you think they were all a mile long? Did you think they didn’t come smaller than that? Sure, they may be no bigger than a dog and no smarter than an ape, but a Dragon Whelp can still belch up a hellish ball of fire that’ll melt your armor shut and drop you screaming into the mud. Their scales, too, are softer than their bigger kin but still turn aside an arrow or sword not perfectly aimed. Size is not the only measure of might. {\em Instinct} : To grow in power
\stopMonsterDescription
       
\startitemize[1,packed]
         
\item Start a lair, form a base of power

         
\item Call on family ties

         
\item Demand oaths of servitude

       
\stopitemize
       
\startMonsterName
Ekek	\CMTags{Horde} 
\stopMonsterName
       

Talons (d6 damage)	3 HP	1 Armor

       

         \CMTags{Close}        

       
\startMonsterQualities
         {\bf Special Qualities:}  Wing-arms
\stopMonsterQualities
       
\startMonsterDescription
Ugly, wrinkled bird-folk, these. Once, maybe, in some ancient past, they were a race of angelic men from on high but now they eat rats they fish from the murk with talon-feet and devour with needle-teeth. They understand the tongues of men and dwarves but speak in little more than gibbering tongues, mimicking the words they hear with mocking laughter. It’s a chilling thing to see a beast so close to man or bird but not quite either one. {\em Instinct} : To lash out
\stopMonsterDescription
       
\startitemize[1,packed]
         
\item Attack from the air

         
\item Carry out the bidding of a more powerful creature

       
\stopitemize
       
\startMonsterName
Fire Eels	\CMTags{Horde, Tiny} 
\stopMonsterName
       

Burning touch (d6-2 damage ignores armor)	3 HP	0 Armor

       

         \CMTags{Hand}        

       
\startMonsterQualities
         {\bf Special Qualities:}  Flammable oil, aquatic
\stopMonsterQualities
       
\startMonsterDescription
These strange creatures are no bigger or smarter then their mundane kin. They have the same vicious nature. Over their relations they have one advantage—an oily secretion that drips from their skin. It makes them hard to catch. On top of that, with a twist of their body they ignite the stuff, leaving pools of burning oil atop the surface of the water and roasting prey and predator alike. I hear the slimy things make good ingredients for fire-resistant gear, but you have to get your hands on one, first. {\em Instinct} : To ignite
\stopMonsterDescription
       
\startitemize[1,packed]
         
\item Catch someone or something on fire (even underwater)

         
\item Consume burning prey

       
\stopitemize
       
\startMonsterName
Frogman	\CMTags{Horde, Small, Intelligent} 
\stopMonsterName
       

Spear (d6 damage)	7 HP	1 Armor

       

         \CMTags{Close}        

       
\startMonsterQualities
         {\bf Special Qualities:}  Amphibious
\stopMonsterQualities
       
\startMonsterDescription
Croak croak croak. Little warty munchkins. Some wizard or godling’s idea of a bad joke, these creatures are. They stand as men, dress in scavenged cloth and hold court in their froggy villages. They speak a rumbling pidgin form of the tongue of man and war constantly with their neighbors. They’re greedy and stupid but clever enough when they need to defend themselves. Some say, too, their priests have a remarkable skill at healing. Or maybe they’re just really, really hard to kill. {\em Instinct} : To war
\stopMonsterDescription
       
\startitemize[1,packed]
         
\item Launch an amphibious assault

         
\item Heal at a prodigious rate

       
\stopitemize
       
\startMonsterName
Hydra	\CMTags{Solitary, Large} 
\stopMonsterName
       

Bite (d10+3 damage)	16 HP	2 Armor

       

         \CMTags{Close, Reach}        

       
\startMonsterQualities
         {\bf Special Qualities:}  Many heads, Only killed by a blow to the heart
\stopMonsterQualities
       
\startMonsterDescription
A bit like a dragon, wingless though it may be. Heads, nine in number at birth, spring from a muscled trunk and weave a sinuous pattern in the air. A hydra is to be feared—a scaled terror of the marsh. The older ones, though, they have more heads for every failed attempt to murder it just makes it stronger. Cut off a head and two more grow in its place. Only a strike, true and strong, to the heart can end a Hydra’s life. Not time or tide or any other thing but this. {\em Instinct} : To grow
\stopMonsterDescription
       
\startitemize[1,packed]
         
\item Attack many enemies at once

         
\item Regenerate a body part (especially a head)

       
\stopitemize
       
\startMonsterName
Kobold	\CMTags{Horde, Small, Stealthy, Intelligent, Organized} 
\stopMonsterName
       

Short spear (d6 damage)	3 HP	1 Armor

       

         \CMTags{Close, Reach}        

       
\startMonsterQualities
         {\bf Special Qualities:}  Dragon connection
\stopMonsterQualities
       
\startMonsterDescription
Some are wont to lump these rat-like little dragon-men in with goblins and orcs, bugbears and hobgoblins. They are smarter and wiser than their kin, however. The Kobolds, beholden slaves to dragons and—in ancient times—their lorekeepers and sorcerer-servants. Their clans, named in fashion like Ironscale and Whitewing, form around a dragon master and live to serve and do its bidding. Spotting a Kobold means more are near—if more are near then a mighty dragon cannot be far, either. {\em Instinct} : To serve dragons
\stopMonsterDescription
       
\startitemize[1,packed]
         
\item Lay a trap

         
\item Call on dragons or draconic allies

         
\item Retreat and regroup

       
\stopitemize
       
\startMonsterName
Lizardman	\CMTags{Group, Stealthy, Intelligent, Organized} 
\stopMonsterName
       

Spear (d8 damage)	6 HP	2 Armor

       

         \CMTags{Close}        

       
\startMonsterQualities
         {\bf Special Qualities:}  Amphibious
\stopMonsterQualities
       
\startMonsterDescription
A traveling sorcerer once told me that Lizardmen came before we did. That before elves and dwarves and men built even the first of their wattle huts a race of proud lizard kings strode the land. That they lived in palaces of crystal and worshipped their own scaly gods. Maybe that’s true and maybe it ain’t—now they dwell in places men long forgot or abandoned, crafting tools from volcano-glass and lashing against the works of the civilized world. Maybe they just want back what they lost. {\em Instinct} : To destroy civilization
\stopMonsterDescription
       
\startitemize[1,packed]
         
\item Ambush the unsuspecting

         
\item Launch an amphibious assault

       
\stopitemize
       
\startMonsterName
Medusa	\CMTags{Solitary, Devious, Intelligent, Hoarder} 
\stopMonsterName
       

Claws (d6 damage)	12 HP	0 Armor

       

         \CMTags{Close}        

       
\startMonsterQualities
         {\bf Special Qualities:}  Look turns you to stone
\stopMonsterQualities
       
\startMonsterDescription
The Medusa are children of a serpent-haired mother, birthing them in ancient times to bear her name across the ages. They dwell near places of civilization—luring folks to their caves with promises of beauty untold or riches. Fine appreciators of art, the medusa curate strange collections of their victims, terror or ecstasy frozen forever in stone. It satisfies their vanity to know they were the last thing seen in so many lives. Arrogant, proud, and spiteful, in their way, they seek what so many do—endless company. {\em Instinct} : To collect
\stopMonsterDescription
       
\startitemize[1,packed]
         
\item Turn a body part to stone with a look

         
\item Draw someone's gaze

         
\item Show hidden terrible beauty

       
\stopitemize
       
\startMonsterName
Sahuagin	\CMTags{Horde, Intelligent} 
\stopMonsterName
       

Endless teeth (d6+4 damage 1 piercing)	3 HP	2 Armor

       

         \CMTags{Close, Forceful, Messy}        

       
\startMonsterQualities
         {\bf Special Qualities:}  Amphibious
\stopMonsterQualities
       
\startMonsterDescription
The shape and craft of men wedded to the hunger and the endless teeth of a shark. Voracious and filled only with hate, these creatures will not stop until all life has been consumed. They cannot be reasoned with, they cannot be controlled or sated. They are hunger and bloodlust, driven up from the depths of the sea to ravage coastal towns and swallow island villages. {\em Instinct} : To spill blood
\stopMonsterDescription
       
\startitemize[1,packed]
         
\item Turn a body part to stone with a look

         
\item Draw someone's gaze

         
\item Show hidden terrible beauty

       
\stopitemize
       
\startMonsterName
Sauropod	\CMTags{Group, Huge, Cautious} 
\stopMonsterName
       

Trample (d8+5 damage)	18 HP	4 Armor

       

         \CMTags{Reach}        

       
\startMonsterQualities
         {\bf Special Qualities:}  Armor plated body
\stopMonsterQualities
       
\startMonsterDescription
Great lumbering beasts, they live in places long since forgotten by the thinking races of the world. Gentle if unprovoked but mighty if their ire is raised, they trample smaller creatures with the care we might give to crushing an any beneath our boots. If you see one, drift by and gaze in awe, but do not wake the giant. {\em Instinct} : To endure
\stopMonsterDescription
       
\startitemize[1,packed]
         
\item Stampede

         
\item Panic

       
\stopitemize
       
\startMonsterName
Swamp Shambler	\CMTags{Solitary, Large, Magical} 
\stopMonsterName
       

Lash (d10+3 damage)	23 HP	5 Armor

       

         \CMTags{Close, Reach, Forceful}        

       
\startMonsterQualities
         {\bf Special Qualities:}  Swamp form
\stopMonsterQualities
       
\startMonsterDescription
Some elementals are conjured up in sacred circles etched in chalk. Most, in fact. There’s a sort of science to it. Others, though, aren’t so orderly—they don’t fall under the carefully controlled assignments of fire or air or earth. Some are a natural confluence of vine and mire and fungus. They do not think the way a man might think. They cannot be understood as an elf might be. They simply are. Spirits of the swamp. Shamblers in the mud. {\em Instinct} : To preserve and create swamps
\stopMonsterDescription
       
\startitemize[1,packed]
         
\item Call on the swamp itself for aid

         
\item Meld into the swamp

         
\item Reassemble into a new form

       
\stopitemize
       
\startMonsterName
Troll	\CMTags{Solitary, Large} 
\stopMonsterName
       

Club (d10+3 damage)	20 HP	1 Armor

       

         \CMTags{Close, Reach, Forceful}        

       
\startMonsterQualities
         {\bf Special Qualities:}  Regeneration
\stopMonsterQualities
       
\startMonsterDescription
Tall. Real tall. Eight or nine feet when they’re young or weak. Covered all over in warty, tough skin, too. Big teeth, stringy hair like swamp moss and long, dirty nails. Some are green, some grey, some black. They’re clannish and hateful of each other, not to mention all the rest of us. Near impossible to kill, too, unless you’ve fire or acid to spare—cut a limb off and watch. In a few days, you’ve got two trolls where you once had one. A real serious problem, as you can imagine. {\em Instinct} : To smash
\stopMonsterDescription
       
\startitemize[1,packed]
         
\item Undo the effects of an attack (unless caused by a weakness, your call)

         
\item Hurl something or someone

       
\stopitemize
       
\startMonsterName
Wll-o-wisp	\CMTags{Solitary, Large} 
\stopMonsterName
       

Club (d10+3 damage)	20 HP	1 Armor

       

         \CMTags{Close, Reach, Forceful}        

       
\startMonsterQualities
         {\bf Special Qualities:}  Body of light
\stopMonsterQualities
       
\startMonsterDescription
Spot a lantern floating in the darkness, lost traveller in the swamp. Hope—a beacon of shimmering light. You call out to it, but there’s no answer. It begins to fade and so you follow, sloshing through the muck, tiring at the chase, hoping you’re being led to safety. Such a sad tale that always ends in doom. These creatures are a mystery—some say they’re ghosts, others beacons of faerie light. Nobody knows the truth. They are cruel, however. All can agree on that. {\em Instinct} : To misguide
\stopMonsterDescription
       
\startitemize[1,packed]
         
\item Lead someone astray

         
\item Clear a path to the worst place possible

       
\stopitemize
                
\section{Legions of the Undead}  \index{Legions of the Undead} \index{Legions} \index{Undead}
            
\startMonsterName
Abomination	\CMTags{Solitary, Large, Construct, Terrifying} 
\stopMonsterName
       

Slam (d10+3 damage)	20 HP	1 Armor

       

         \CMTags{Close, Reach, Forceful}        

       
\startMonsterQualities
         {\bf Special Qualities:}  Many limbs, heads, and so on
\stopMonsterQualities
       
\startMonsterDescription
Corpses sewn unto corpses make up the bulk of these shambling masses of dark magic.  Most undead are crafted to be controlled—made to serve some purpose like building a tower or serving as guardians.  Not so the abomination.  The last aspect of the ritual used to grant fire to their hellish limbs invokes a hatred so severe that the abomination knows but one task: to tear and rend at the very thing it cannot have—life.  Many students of the black arts learn to their mortal dismay the most important fact about these hulks. An abomination knows no master. {\em Instinct} : To end life
\stopMonsterDescription
       
\startitemize[1,packed]
         
\item Tear flesh apart

         
\item Spill forth putrid guts

       
\stopitemize
       
\startMonsterName
Banshee	\CMTags{Solitary, Magical, Intelligent} 
\stopMonsterName
       

Scream (d10 damage)	16 HP	0 Armor

       

         \CMTags{Near}        

       
\startMonsterQualities
         {\bf Special Qualities:}  Insubstantial
\stopMonsterQualities
       
\startMonsterDescription
Come away from an encounter with one of these vengeful spirits merely deaf and count yourself lucky for the rest of your peaceful, silent days.  Often mistaken at first glance for a ghost or wandering spirit, the Banshee reveals a far more deadly talent for sonic assault when angered.  And angered they are. A victim of betrayal (often by a loved one) the Banshee makes known their displeasure with a roar or scream that can putrefy flesh and rend the senses.  If you can, help them get their vengeance and they’ve been known to grant rewards.  Whether the affection of a spurned spirit is a thing you’d want, well, that’s another question. {\em Instinct} : To avenge
\stopMonsterDescription
       
\startitemize[1,packed]
         
\item Drown out all other sound with a ceaseless scream

         
\item Unleash a skull-splitting noise

         
\item Disappear into the mists

       
\stopitemize
       
\startMonsterName
Devourer	\CMTags{Solitary, Large, Intelligent, Hoarder} 
\stopMonsterName
       

Smash (d10+3 damage)	16 HP	1 Armor

       

         \CMTags{Close, Reach, Forceful}        

       
\startMonsterDescription
Most folk know that the undead feed on flesh.  The warmth, blood and living tissue continue their unholy existence.  This is true for most of the mindless dead, animated by black sorcery.  Not so the Devourer.  When a particularly wicked person (often a manipulator of men, an apostate priest or the like) dies in a gruesome way, the dark powers of Dungeon World might bring them back to a kind of life.  The Devourer, however, does not feed on the flesh of men or elves.  The Devourer eats souls.  It kills with a pleasure only the sentient can enjoy and in the moments of its victims’ expiry, draws breath like a drowning man and swallows a soul.  What does it mean to have your soul eaten by such a creature?  None dare ask for fear of finding out. {\em Instinct} : To eat souls
\stopMonsterDescription
       
\startitemize[1,packed]
         
\item Devour or trap dying soul

         
\item Bargain for a soul

       
\stopitemize
       
\startMonsterName
Dragonbone	\CMTags{Solitary, Huge} 
\stopMonsterName
       

Bite (d10+3 damage 3 piercing)	20 HP	2 Armor

       

         \CMTags{Reach, Messy}        

       
\startMonsterDescription
Mystical sorcerers debate: is this creature truly undead or is it a golem made of a particularly rare and blasphemous material?  The bones, sinews and scales of a dead dragon make up this bleak automaton.  Winged but flightless, dragon-shaped but without the mighty fire of such a noble thing, the Dragonbone serves its master with a twisted devotion and is often set to assault the keeps and towers of rival necromancers.  It would take a being of some considerable evil to twist the remains of a dragon thus. {\em Instinct} : To serve
\stopMonsterDescription
       
\startitemize[1,packed]
         
\item Attack unrelentingly

       
\stopitemize
       
\startMonsterName
Draugr	\CMTags{Horde, Organized} 
\stopMonsterName
       

Rusty sword (d6+1 damage)	7 HP	2 Armor

       

         \CMTags{Close, Reach}        

       
\startMonsterQualities
         {\bf Special Qualities:}  Icy touch
\stopMonsterQualities
       
\startMonsterDescription
In the Nordemark, the men and women tell tales in their wooden halls of a place where the noble dead go.  A mead hall atop their heavenly mountain where men of valor go to await the final battle for the world.  It is a goodly place.  It is a place where one hopes to go when they die.  And the inglorious dead?  Those who fall to poison or in an act of cowardice, warriors though they may be?  Well, those mead halls aren’t open to all and sundry. Some come back, frozen and twisted and empowered by jealous rage and wage their eternal war not on the forces of giants or trolls but on the towns of the men they once knew. {\em Instinct} : To take from the living
\stopMonsterDescription
       
\startitemize[1,packed]
         
\item Freeze flesh

         
\item Call on the unworthy dead

       
\stopitemize
       
\startMonsterName
Ghost	\CMTags{Solitary, Devious, Terrifying} 
\stopMonsterName
       

Phantom touch (d6 damage)	16 HP	0 Armor

       

         \CMTags{Close, Reach}        

       
\startMonsterQualities
         {\bf Special Qualities:}  Insubstantial
\stopMonsterQualities
       
\startMonsterDescription
Every culture tells the story the same way.  You live, you love or hate, you win or you lose, you die somehow you’re not too fond of and here you are, ghostly and full of disappointment and what have you.  Some people take it upon themselves, brave and kindly folks, to seek out the dead and help them pass to their rightful rest.  You can find them, most times, down at the tavern drinking away the terrors they’ve seen or babbling to themselves in the madhouse.  Death takes a toll on the living, no matter how you come by it. {\em Instinct} : To haunt
\stopMonsterDescription
       
\startitemize[1,packed]
         
\item Reveal the terrifying nature of death

         
\item Offer information from the other side, at a price

       
\stopitemize
       
\startMonsterName
Ghoul	\CMTags{Group} 
\stopMonsterName
       

Talons (d8 damage 1 piercing)	10 HP	1 Armor

       

         \CMTags{Close, Messy}        

       
\startMonsterDescription
Hunger.  Hunger hunger hunger.  Desperate clinging void-stomach-emptiness hunger.  Sharp talons to rend flesh and teeth to tear and crack bones and suck out the soft marrow inside.  Vomit up hate and screaming jealous anger and charge on twisted legs—scare the living flesh and sweeten it ever more with the stink of fear.  Feast.  Peasant or knight, wizard, sage, prince, or priest all make for such delicious meat. {\em Instinct} : To eat
\stopMonsterDescription
       
\startitemize[1,packed]
         
\item Gnaw off a body part

         
\item Gain the memories of their meal

       
\stopitemize
       
\startMonsterName
Lich	\CMTags{Solitary, Magical, Intelligent, Cautious, Hoarder, Construct} 
\stopMonsterName
       

Enervate (d10+3 damage ignores armor)	16 HP	5 Armor

       

         \CMTags{Near, Far}        

       
\startMonsterDescription
“At the end, they give you a scroll and a jeweled medallion to commemorate your achievements.  Grand Master of Abjuration, I was called, then.  Old man.  Weak and wizened and just a bit too senile for them—hose jealous halfwits.  Barely apprentices, and they called themselves The New Council.  It makes me sick, or would, if I still could be.  They told me it was an honor and I would be remembered forever.  It was like listening to my own eulogy.  Fitting, in a way, don’t you think?  It took me another ten years to learn the rituals and another four to collect the material and you see before you the fruits of my labour.  I endure.  I live.  I will see the death of this age and the dawn of the next.  It pains me to have to do this, but, you see, you cannot be permitted to endanger my research.  When you meet Death, say hello for me, would you?” {\em Instinct} : To un-live
\stopMonsterDescription
       
\startitemize[1,packed]
         
\item Cast a perfected spell of death or destruction

         
\item Set a ritual or great working into motion

         
\item Reveal a preparation or plan already completed

       
\stopitemize
       
\startMonsterName
Mohrg	\CMTags{Group} 
\stopMonsterName
       

Bite (d8 damage)	10 HP	0 Armor

       

         \CMTags{Close}        

       
\startMonsterDescription
You never get away with murder.  Not really.  You might evade the law, might escape your own conscience in the end and die, fat and happy in a mansion somewhere.  When the gods themselves notice your misdeeds, though, that’s where your luck runs out and a Mohrg is born.  The Mohrg is a skeleton—flesh and skin and hair all rotted away.  All but their guts—their twisted, knotted guts still spill from their bellies, magically preserved and often wrapped, noose-like, about their necks.  They do not think, exactly, but they suffer.  They kill and wreak havoc and their souls do not rest.  Such is the punishment, both on them for the crime and on all mankind for daring to murder one another.  The gods are just and they are harsh. {\em Instinct} : To wreak havoc
\stopMonsterDescription
       
\startitemize[1,packed]
         
\item Rage

         
\item Add to their collection of guts

       
\stopitemize
       
\startMonsterName
Mummy	\CMTags{Solitary, Divine, Hoarder} 
\stopMonsterName
       

Smash (d10+2 damage)	16 HP	1 Armor

       

         \CMTags{Close}        

       
\startMonsterDescription
There are cultures who revere the dead.  They do not bury them in the cold earth and mourn their passing. These people spend weeks preparing the sacred corpse for its eternal rest.  Temples, pyramids, and great vaults of stone are built to house them and are populated with slaves, pets and gold.  The better to live in luxury beyond the Black Gates, no?  Do not be tempted by these vaults—oh, I know that greedy look!  Heed my warnings or risk a terrible fate, for the honored dead do not wish to be disturbed.  Thievery will only raise their ire—do not say I did not warn you! {\em Instinct} : To enjoy eternal rest
\stopMonsterDescription
       
\startitemize[1,packed]
         
\item Curse them

         
\item Wrap them up

         
\item Rise again

       
\stopitemize
       
\startMonsterName
Nightwing	\CMTags{Horde, Stealthy} 
\stopMonsterName
       

Rend (d6 damage)	7 HP	1 Armor

       

         \CMTags{Close}        

       
\startMonsterQualities
         {\bf Special Qualities:}  Wings
\stopMonsterQualities
       
\startMonsterDescription
Scholars of the necromantic arts will tell you that the appellation “undead” applies not only to those who have lived, died, and been returned to a sort of partway living state.  It is the proper name of any creature whose energy originates beyond the Black Gates.  The creature men call the nightwing is one such creature—empowered by the negative light of Death’s domain.  Taking the shape of massive, shadowy, winged creatures (some more batlike, some like vultures, others like some ancient, leathery things) nightwings travel in predatory flocks, swooping down to strip the flesh from cattle, horses and unlucky peasants out past curfew.  Watch the night sky for their red eyes and listen for their screeching call and hope to the gods you have something to hide under until they pass. {\em Instinct} : To hunt
\stopMonsterDescription
       
\startitemize[1,packed]
         
\item Attack from the night sky

         
\item Fly away with prey

       
\stopitemize
       
\startMonsterName
Shadow	\CMTags{Horde, Large, Magical, Construct} 
\stopMonsterName
       

Shadow Touch (d6+1 damage)	11 HP	4 Armor

       

         \CMTags{Close, Reach}        

       
\startMonsterQualities
         {\bf Special Qualities:}  Shadow Form
\stopMonsterQualities
       
\startMonsterDescription
We call to the elements.  We call on fire, ever-burning.  We summon water, life-giving.  We beseech the earth, stable-standing.  We cry to the air, forever-changing.  These elements we recognize and give our thanks but ask to pass.  The elemental we call upon this night knows another name.  We call upon the element of Night.  Shadow, we name you.  Death’s messenger and black assassin, we claim for our own.  Accept our sacrifice and do our bidding ‘til the morning come. {\em Instinct} : To darken
\stopMonsterDescription
       
\startitemize[1,packed]
         
\item Snuff out light

         
\item Spawn another shadow from the dead

       
\stopitemize
       
\startMonsterName
Sigben	\CMTags{Horde, Large, Construct} 
\stopMonsterName
       

Tail whip (d6+1 damage)	11 HP	2 Armor

       

         \CMTags{Close, Reach}        

       
\startMonsterQualities
         {\bf Special Qualities:}  Vampire spawn
\stopMonsterQualities
       
\startMonsterDescription
Aswang-hound and hopping whip-tail!  Sent by vampires on their two, twisted legs, these ugly things look like the head of a rat or a crocodile, maybe, furry though and sharp of tooth.  They have withered wings, but cannot use them and long, whipping tails, spurred with poison tips.  Stupid, vengeful and mischievous they cause all kinds of chaos when let out of the strange clay jars in which they’re born.  Only a vampire could love such a wretched thing. {\em Instinct} : To disturb
\stopMonsterDescription
       
\startitemize[1,packed]
         
\item Poison them

         
\item Do a vampire's bidding

       
\stopitemize
       
\startMonsterName
Skeleton	\CMTags{Horde} 
\stopMonsterName
       

Slam (d6 damage)	7 HP	1 Armor

       

         \CMTags{Close}        

       
\startMonsterDescription
Dem bones, dem bones, dem dry bones. {\em Instinct} : To take the semblance of life
\stopMonsterDescription
       
\startitemize[1,packed]
         
\item Act out what it did in life

         
\item Reveal a fact from its life

       
\stopitemize
       
\startMonsterName
Spectre	\CMTags{Solitary, Hoarder} 
\stopMonsterName
       

Withering touch (d10 damage)	12 HP	0 Armor

       

         \CMTags{Close}        

       
\startMonsterQualities
         {\bf Special Qualities:}  Insubstantial
\stopMonsterQualities
       
\startMonsterDescription
For some folk, when they pass, Death himself cannot release their grip on the places they love most.  A priest whose devotion to the temple is greater than that of his god.  A banking guild official who cannot bear to part with his vault.  A drunk and his favourite tavern.  All make excellent spectres.  They act not out of the usual hunger that drive the undead, but jealousy.  Jealousy that anyone else might come to love their home as much as them and drive them out.  These places belong to them and these invisible spirits will kill before they’ll let anyone send them to their rest. {\em Instinct} : To drive life from a place
\stopMonsterDescription
       
\startitemize[1,packed]
         
\item Turn their haunt against a creature

         
\item Bring the environment to life

       
\stopitemize
       
\startMonsterName
Vampire	\CMTags{Group, Stealthy, Organized, Intelligent} 
\stopMonsterName
       

Supernatural force (d8+5 damage 1 piercing)	10 HP	2 Armor

       

         \CMTags{Close, Forceful}        

       
\startMonsterQualities
         {\bf Special Qualities:}  Changing form, ancient mind
\stopMonsterQualities
       
\startMonsterDescription
We fear them, because they call to us.  So much like us, or how we hope to be: beautiful, passionate, and powerful.  They are drawn to us for what they cannot be: warm, kind, and alive.  These tormented souls can only hope, at most, to pass their dreadful curse along.  Every time they feed they run the risk of passing along their torture to another and in each one lives the twisted seed of creator.  Vampires beget vampires.  Suffering begets suffering.  Do not be drawn in by their seduction or you may be given their gift—a crown of shadows and the chains eternal undying grief. {\em Instinct} : To manipulate
\stopMonsterDescription
       
\startitemize[1,packed]
         
\item Charm someone

         
\item Feed on their blood

         
\item Retreat to plan again

       
\stopitemize
       
\startMonsterName
Wight-Wolf	\CMTags{Horde, Organized, Intelligent} 
\stopMonsterName
       

Pounce (d6+1 damage 1 piercing)	7 HP	1 Armor

       

         \CMTags{Close}        

       
\startMonsterQualities
         {\bf Special Qualities:}  Shadow form
\stopMonsterQualities
       
\startMonsterDescription
Like the nightwing, the wight-wolf is a creature not spawned in our world.  Somehow slipping the seals of the Black Gates of Death, these spirits take the shape of massive hounds or shadowy wolves and hunt the living for sport.  They travel in packs, led by a mighty alpha, but bear a kind of intelligence unknown to true canines.  Their wild hunts draw the attention of intelligent undead—liches, vampires and the like—who will sometimes make pacts with the alpha and serve a grim purpose together.  Listen for the baying of the hounds of Death and pray that they do not howl for you. {\em Instinct} : To hunt
\stopMonsterDescription
       
\startitemize[1,packed]
         
\item Encircle prey

         
\item Summon the pack

       
\stopitemize
       
\startMonsterName
Zombie	\CMTags{Horde} 
\stopMonsterName
       

Bite (d6 damage)	11 HP	1 Armor

       

         \CMTags{Close}        

       
\startMonsterDescription
When there’s no more room in Hell… {\em Instinct} : Braaaaaains
\stopMonsterDescription
       
\startitemize[1,packed]
         
\item Attack with overwhelming numbers

         
\item Corner them

         
\item Gain strength from the dead, spawn more zombies

       
\stopitemize
                
\section{Gnarled Woods}  \index{Gnarled Woods} \index{Gnarled} \index{Woods}
            
\startMonsterName
Assassin Vine	\CMTags{Solitary, Stealthy, Gibbous} 
\stopMonsterName
       

Thorns (d10 damage 1 piercing)	15 HP	1 Armor

       

         \CMTags{Close, Reach, Messy}        

       
\startMonsterQualities
         {\bf Special Qualities:}  Plant
\stopMonsterQualities
       
\startMonsterDescription
Among the animals there exists a clear division ‘tween hunter and hunted. All it takes is a glance to know—by fangs and glowing eyes or claws or venomous sting—which of the creature of this world are meant to kill and which stand to be killed. Such a split, if you have the eyes to see it, cuts the world of leaves and flowers in twain, as well. Druids in their forest circles know it. Rangers, too, might spot such a plant before it’s too late. Lay folk, though, they wander where they oughtn’t—paths into the deep woods covered in creeping vines and with a snap, these hungry ropes snap tight, dragging their meaty prey into the underbrush. Mind your feet, traveller. {\em Instinct} : To grow
\stopMonsterDescription
       
\startitemize[1,packed]
         
\item Shoot forth new growth

         
\item Attack the unwary

       
\stopitemize
       
\startMonsterName
Blink Dog	\CMTags{Group, Small, Magical, Organized} 
\stopMonsterName
       

Bite (d8 damage)	6 HP	4 Armor

       

         \CMTags{Close}        

       
\startMonsterQualities
         {\bf Special Qualities:}  Illusion
\stopMonsterQualities
       
\startMonsterDescription
Now you see it, now you don’t. Hounds once owned by a sorcerer lord and imbued with a kind of illusory cloak, they escaped into the woods around his lair and began to breed with wolves and wild dogs of the forest. You can spot them, if you’re lucky, by the glittering silver of their coats and their strange, ululating howls. They have a remarkable talent for being not-quite where they appear to be and use it to take down prey much stronger than themselves. If you find yourself facing a pack of blink dogs you might well close your eyes and fight. You’ll have an easier time when not betrayed by your natural sight. By such sorceries are the natural places of the world polluted with unnatural things. {\em Instinct} : To hunt
\stopMonsterDescription
       
\startitemize[1,packed]
         
\item Give the appearance of being somewhere they're not

         
\item Summon the pack

         
\item Move with amazing speed

       
\stopitemize
       
\startMonsterName
Centaur	\CMTags{Horde, Large, Organized, Intelligent} 
\stopMonsterName
       

Bow (d6+2 damage 1 piercing)	11 HP	1 Armor

       

         \CMTags{Close, Reach, Near}        

       
\startMonsterQualities
         {\bf Special Qualities:}  Half-horse, Half-man
\stopMonsterQualities
       
\startMonsterDescription
It will be a gathering of clans unseen in this age. Call Stormhoof and Brightspear. Summon Whitemane and Ironflanks. Sound the horn and we shall begin our meeting—we shall speak the words and bind our people together. Too long have the men cut the ancient trees for their ships. The elves are weak and cowardly, friend to these mannish slime. It will be a cleansing fire from the darkest woods. Raise the red banner of war! Today we strike back against these apes and retake what is ours! {\em Instinct} : To rage
\stopMonsterDescription
       
\startitemize[1,packed]
         
\item Overrun them

         
\item Move with unrelenting speed

       
\stopitemize
       
\startMonsterName
Chaos Ooze	\CMTags{Solitary, Planar, Terrifying, Gibbous} 
\stopMonsterName
       

Warping touch (d10 damage ignores armor)	23 HP	1 Armor

       

         \CMTags{Close}        

       
\startMonsterQualities
         {\bf Special Qualities:}  Ooze, Fragments of other planes embedded in it
\stopMonsterQualities
       
\startMonsterDescription
The barrier between Dungeon World and the Elemental Planes is not, as you might hope, a wall of stone. It’s much more porous. Thin-like, with holes. Places where the civil races do not often tread can sometimes, how to put this, spring a leak. Like a dam come just a little loose. Bits and pieces of the chaos spill out. Sometimes, they’ll congeal like an egg on a pan—that’s where we get the material for many of the Guild’s magical trinkets. Useful, right? Sometimes, though, it squirms and squishes around a bit and stays that way, warping all it touches into some other, strange form. Chaos begets chaos, and it grows. {\em Instinct} : To change
\stopMonsterDescription
       
\startitemize[1,packed]
         
\item Cause a change in appearance or substance

         
\item Briefly bridge the planes

       
\stopitemize
       
\startMonsterName
Cockatrice	\CMTags{Group, Small, Hoarder} 
\stopMonsterName
       

Peck (d8 damage)	6 HP	1 Armor

       

         \CMTags{Close}        

       
\startMonsterQualities
         {\bf Special Qualities:}  Stone touch
\stopMonsterQualities
       
\startMonsterDescription
I ain’t ever seen such a thing, sir. Rodrick thought it a chicken, maybe. Poor Rodrick. I figured it to be a lizard of a sort, though he was right—it had a beak and grey feathers like a chicken. Right, well, see, we found it in the woods, in a nest at the foot of a tree while we were out with the sow. Looking for mushrooms, sir. I told Rodrick we were—yes, sir, right sir, the bird—see, it was glaring at Rodrick and he tried to scare it off with a stick to steal the eggs but the thing pecked his hand. Quick it was, too. I tried to get him away but he just got slower and slower and…yes, as you see him now, sir. All frozen up like when we left the dog out overnight in winter two years back. Poor, stupid Rodrick. Weren’t no bird nor lizard, were it, sir? {\em Instinct} : To defend the nest
\stopMonsterDescription
       
\startitemize[1,packed]
         
\item Start a slow transformation to stone

       
\stopitemize
       
\startMonsterName
Dryad	\CMTags{Solitary, Magical, Intelligent, Devious, Gibbous} 
\stopMonsterName
       

Crushing vines (2d10·w damage)	23 HP	5 Armor

       

         \CMTags{Close}        

       
\startMonsterQualities
         {\bf Special Qualities:}  Plant
\stopMonsterQualities
       
\startMonsterDescription
More beautiful by far than any man or woman born in the civil realms. To gaze upon one is to fall in love. Deep and punishing, too. Thing is, they don’t love. Not the fleshy folk who often find them, though. Their love is a primal thing, married to the woods—to a great oak that serves as home and mother and sacred place to them. It’s a curse to see one, too, they’ll never love you back. No matter what you do. No matter how you pledge yourself to them, they’ll always spurn you. If ever their oak comes to harm, you’ve not only the dryad’s wrath to contend with, but in every nearby village there’s a score of men with a secret longing in their heart, ready to murder you where you sleep for just a smile from such a creature. {\em Instinct} : To love nature passionately
\stopMonsterDescription
       
\startitemize[1,packed]
         
\item Entice a mortal

         
\item Merge into a tree

         
\item Turn nature against them

       
\stopitemize
       
\startMonsterName
Eagle Lord	\CMTags{Group, Large, Organized, Intelligent} 
\stopMonsterName
       

Claw (2d8·b+1 damage 1 piercing)	10 HP	1 Armor

       

         \CMTags{Close, Reach}        

       
\startMonsterQualities
         {\bf Special Qualities:}  Mighty wings
\stopMonsterQualities
       
\startMonsterDescription
Some the size of horses. Bigger, even—the Kings and Queens of the Eagles. Their cry pierces the mountain sky and woe to those who fall under the shadow of their mighty wings. The ancient wizards forged a pact with them in the primordial days. Men would take the plains and valleys and leave the mountaintops to the Eagle Lords. These sacred pacts should be honored, lest they set their talons into you. Lucky are the elves, for the makers of their bonds yet live and when danger comes to Elvish lands, the Eagle Lords often serve as spies and mounts for the elves. Long-lived and proud, some might be willing to trade their ancient secrets for the right price, too. {\em Instinct} : To rule the heights
\stopMonsterDescription
       
\startitemize[1,packed]
         
\item Attack from the sky

         
\item Pull someone into the air

         
\item Call on ancient oaths

       
\stopitemize
       
\startMonsterName
Elvish Warrior	\CMTags{Horde, Intelligent, Organized} 
\stopMonsterName
       

Sword (2d6·b damage)	3 HP	2 Armor

       

         \CMTags{Close}        

       
\startMonsterQualities
         {\bf Special Qualities:}  Sharp sense
\stopMonsterQualities
       
\startMonsterDescription
Like all the elves do, war is an art. I saw them fight, once. The Battle of Astrid’s Veil. Yes, I am that old, boy, now hush. She was clad in plate that shone like the winter sky. White hair streaming and a pennant of ocean blue tied to her spear. She seemed to glide across between the trees the way an angel might, striking out and bathing her blade in blood that steamed in the cold air. I never felt so small before. I trained with the master-at-arms of Battlemoore, you know. I’ve held a sword longer than you’ve been alive, boy, and in that one moment I knew that my skill meant nothing. Thank the gods the elves were with us then. A more beautiful and terrible thing I have not seen since. {\em Instinct} : To seek perfection
\stopMonsterDescription
       
\startitemize[1,packed]
         
\item Strike at a weak point

         
\item Set ancient plans in motion

         
\item Use the woods to advantage

       
\stopitemize
       
\startMonsterName
Elvish High Arcanist	\CMTags{Solitary, Magical, Intelligent, Organized} 
\stopMonsterName
       

Flame (d10 damage ignores armor)	12 HP	0 Armor

       

         \CMTags{Near, Far}        

       
\startMonsterQualities
         {\bf Special Qualities:}  Sharp senses
\stopMonsterQualities
       
\startMonsterDescription
True elvish magic isn’t like the spells of men. Mannish wizardry is all rotes and formulas. They cheat to find the arcane secrets that resound all around them. They are deaf to the arcane symphony that sings in the woods. Elvish magic is fine ear to hear it and the voice with which to sing. To harmonize with what is already resounding. Men bind the forces of magic to their will; Elves simply pluck the strings and hum along. The High Arcanists, in a way, have become more and less than any elf. The beat of their blood is the throbbing of all magic in this world. {\em Instinct} : To unleash power
\stopMonsterDescription
       
\startitemize[1,packed]
         
\item Work the magic that wants to be worked

         
\item Cast forth the elements

       
\stopitemize
       
\startMonsterName
Griffin	\CMTags{Group, Large, Organized} 
\stopMonsterName
       

Claw (d8+3 damage)	10 HP	1 Armor

       

         \CMTags{Close, Reach, Forceful}        

       
\startMonsterQualities
         {\bf Special Qualities:}  Wings
\stopMonsterQualities
       
\startMonsterDescription
On first glance, one might mistake the Griffin for another magical mistake like the Manticore or the Chimera. It looks the part, doesn’t it? These creatures have the regal haughtiness of a lion and the arrogant bearing of a eagle but temper it with the unshakeable loyalty of both. To earn the friendship of a Griffin is to have an ally all your living days. Truly a gift, that. If you’re ever lucky enough to meet one be respectful and deferential above all else. It may not seem it but they can tell and answer perceived slights with a sharp beak and talons. {\em Instinct} : To serve allies
\stopMonsterDescription
       
\startitemize[1,packed]
         
\item Carry an ally aloft

         
\item Strike from above

       
\stopitemize
       
\startMonsterName
Ogre	\CMTags{Group, Large, Intelligent} 
\stopMonsterName
       

Club (d8+5 damage)	10 HP	1 Armor

       

         \CMTags{Close, Reach, Forceful}        

       
\startMonsterDescription
A tale, then. Somewhere in the not-so-long history of the Mannish race there was a divide. In days when men were merely dwellers-in-the-mud with no magic to call their own, they split in two: one camp left their caves and the dark forests and built the First City to honor the gods. The others, a wild and savage lot, retreated into darkness. They grew, there. In the deep woods a grim loathing for their softer kin gave them strength. They found dark gods of their own, there in the woods and hills. Ages passed and they bred tall and strong and full of hate. We have forged steel and they match it with their savagery. We may have forgotten our common roots, but somewhere, deep down, the Ogres remember. {\em Instinct} : To return the world to darker days
\stopMonsterDescription
       
\startitemize[1,packed]
         
\item Destroy something

         
\item Topple trees

         
\item Bring down the roof

       
\stopitemize
       
\startMonsterName
Hill Giant	\CMTags{Group, Huge, Intelligent, Organized} 
\stopMonsterName
       

Rock (d8+3 damage)	10 HP	1 Armor

       

         \CMTags{Reach, Near, Far, Forceful}        

       
\startMonsterDescription
Ever seen an ogre before? Bigger than that. Dumber and meaner, too. Hope you like having cows thrown at you. {\em Instinct} : To hurl
\stopMonsterDescription
       
\startitemize[1,packed]
         
\item Throw something

         
\item Shake the earth

       
\stopitemize
       
\startMonsterName
Razor Boar	\CMTags{Solitary} 
\stopMonsterName
       

Bite (d10 damage 3 piercing)	16 HP	1 Armor

       

         \CMTags{Close, Messy}        

       
\startMonsterDescription
The tusks of the razor boar shred metal plate like so much tissue. Voracious, savage and unstoppable, they tower over their mundane kin. To kill one? A greater trophy of bravery and skill is hard to name, though I hear a razor boar killed the Drunkard King in a single thrust. You think you’re a better hunter than he? {\em Instinct} : To shred
\stopMonsterDescription
       
\startitemize[1,packed]
         
\item Rip them apart

         
\item Rend armor and weapons

       
\stopitemize
       
\startMonsterName
Sprite	\CMTags{Horde, Tiny, Stealthy, Magical, Devious, Intelligent} 
\stopMonsterName
       

Dagger (2d4·w damage)	3 HP	0 Armor

       

         \CMTags{Hand}        

       
\startMonsterQualities
         {\bf Special Qualities:}  Wings, Fey Magic
\stopMonsterQualities
       
\startMonsterDescription
I’d classify them elementals, except that “Being Annoying” isn’t an element. {\em Instinct} : To play tricks
\stopMonsterDescription
       
\startitemize[1,packed]
         
\item Play a trick to expose someone's true nature

         
\item Confuse their senses

         
\item Craft an illusion

       
\stopitemize
       
\startMonsterName
Treant	\CMTags{Group, Huge, Intelligent, Gibbous} 
\stopMonsterName
       

Wallop (d8+5 damage)	21 HP	4 Armor

       

         \CMTags{Reach, Forceful}        

       
\startMonsterQualities
         {\bf Special Qualities:}  Wooden
\stopMonsterQualities
       
\startMonsterDescription
Old and tall and thick of bark
\stopMonsterDescription
       

walk amidst the tree-lined dark

       

Strong and slow and forest-born

       

the treants anger quick, we warned

       

If to the woods with axe ye go

       

know the treants be thy foe

       

         {\em Instinct} : To protect nature

       
\startitemize[1,packed]
         
\item Move with implacable strength

         
\item Set down roots

         
\item Spread old magic

       
\stopitemize
       
\startMonsterName
Werewolf	\CMTags{Solitary, Intelligent} 
\stopMonsterName
       

Bite (d10+2 damage 1 piercing)	12 HP	1 Armor

       

         \CMTags{Close, Messy}        

       
\startMonsterQualities
         {\bf Special Qualities:}  Weak to silver
\stopMonsterQualities
       
\startMonsterDescription
Beautiful, isn’t it? The moon, I mean. She’s watching us, you know? Her pretty silver eyes watch us while we sleep. Mad, too—like all the most beautiful ones. If she were a woman, I’d bend my knee and make her my wife on the spot. No, I didn’t ask you here to speak about her, though. The chains? For your safety, not mine. I’m cursed, you see. You must have suspected. The sorcerer-kings called it “lycanthropy” in their day—passed on by a bite to make more of our kind. No, I could find no cure. Please, Don’t be scared. You have the arrows I gave you? Silver, yes. Ah, you begin to understand. Don’t cry, sister. You must do this for me. I cannot bear more blood on my hands. You must end this. For me. {\em Instinct} : To shed the appearance of civilization
\stopMonsterDescription
       
\startitemize[1,packed]
         
\item Transform to pass unnoticed as beast or man

         
\item Strike from within

         
\item Hunt like man and beast

       
\stopitemize
       
\startMonsterName
Worg	\CMTags{Horde, Organized} 
\stopMonsterName
       

Bite (d6 damage)	3 HP	1 Armor

       

         \CMTags{Close}        

       
\startMonsterDescription
As horses are to the civil races, so go the worg to the goblins. Mounts, fierce in battle, ridden by only the bravest and most dangerous, are found and bred in the forest primeval to serve the goblins in their wars on men. The only safe worg is a pup, separated from its mother. If you can find one of these, or make orphans of a litter with a sharp sword, you’ve got what could become a loyal protector or hunting hound in time. Train it well, mind you, for the worg are smart and never quite free of their primal urges. {\em Instinct} : To serve
\stopMonsterDescription
       
\startitemize[1,packed]
         
\item Carry a rider into battle

         
\item Give its rider an advantage

       
\stopitemize
       
\startMonsterName
Satyr	\CMTags{Group, Devious, Magical, Hoarder} 
\stopMonsterName
       

Charge (2d8·w damage)	10 HP	1 Armor

       

         \CMTags{Close}        

       
\startMonsterQualities
         {\bf Special Qualities:}  Enchantment
\stopMonsterQualities
       
\startMonsterDescription
One of only a very few creatures to be found in the old woods that don’t right out want to maim, kill, or eat us. They dwell in glades pierced by the sun, and dance on their funny goat-legs to enchanting music played on pipes made of bone and silver. They smile easily and, so long as you please them with jokes and sport, will treat our kind with friendliness. They’ve a mean streak, though, so if you cross them, make haste elsewhere; very few things hold a grudge like the stubborn Satyr. {\em Instinct} : To enjoy
\stopMonsterDescription
       
\startitemize[1,packed]
         
\item Pull others into revelry through magic

         
\item Force gifts upon them

         
\item Play jokes with illusions and tricks

       
\stopitemize
                
\section{Ravenous Hordes}  \index{Ravenous Hordes} \index{Ravenous} \index{Hordes}
            
\startMonsterName
Formian Drone	\CMTags{Horde, Organized, Cautious} 
\stopMonsterName
       

Bite (d6 damage)	7 HP	4 Armor

       

         \CMTags{Close}        

       
\startMonsterQualities
         {\bf Special Qualities:}  Hive connection, Insectoid
\stopMonsterQualities
       
\startMonsterDescription
With good cause, they say that these creatures (like all insects, really) are claimed by the powers of Law. They are order made flesh—a perfectly stratified society in which every larva, hatchling and adult knows their place in the great hive. The Formian is some strange intersection of men and ants (though there are winged tribes that look like ants out in the Western Desert, I’ve heard. And some with great sawtooth arms like mantids in the forests of the east). Tall, with a hard shell and a harder mind, these particular Formians are the bottom caste. They work the hills and honeycombs with single-minded joy that can be known only by such an alien mind. {\em Instinct} : To follow orders
\stopMonsterDescription
       
\startitemize[1,packed]
         
\item Raise the alarm

         
\item Create value for the hive

         
\item Assimilate

       
\stopitemize
       
\startMonsterName
Formian Taskmaster	\CMTags{Group, Organized, Intelligent} 
\stopMonsterName
       

Spiked whip (d8 damage)	6 HP	3 Armor

       

         \CMTags{Close, Reach}        

       
\startMonsterQualities
         {\bf Special Qualities:}  Hive connection, Insectoid
\stopMonsterQualities
       
\startMonsterDescription
It takes two hands to rule an empire: one to wield the scepter and one to crack the whip. These ant-folk are that whip. Lucky for them, with two extra arms, that’s a lot of whip to crack. They oversee the vast swarms of worker drones that set to build the mighty caverns and ziggurats that dot the places that formians can be found. One in a hundred, these brutes stand two or three feet taller than their pale, near-mindless kin and have a sharper, crueler wit to match. They’ll often ignore the soft races (as we’re known) if we don’t interfere in a project, but get in the way of The Great Work and expect nothing less than their full attention. You don’t want their full attention. {\em Instinct} : To command
\stopMonsterDescription
       
\startitemize[1,packed]
         
\item Order drones into battle

         
\item Set great numbers in motion

       
\stopitemize
       
\startMonsterName
Formian Centurion	\CMTags{Horde, Intelligent, Organized} 
\stopMonsterName
       

Barbed spear (2d6·b+2 damage)	7 HP	3 Armor

       

         \CMTags{Close, Reach}        

       
\startMonsterQualities
         {\bf Special Qualities:}  Hive connection, Insectoid, Wings
\stopMonsterQualities
       
\startMonsterDescription
Whether in the form of a legionnaire, part of the Formian standing army, or as a praetorian guard to the queen, every formian hive contain a great number of these most dangerous insectoids. Darker in carapace, often scarred with furrows and the ceremonial markings that set them apart from their drones, the formian centurions are their fighting force and rightly so. Born, bred and living only for the singular purpose to kill the enemies of their kind, they fight with one mind and a hundred swords. Thus far, the powers of Law have seen fit to spare mankind a great war with these creatures, but we’ve seen them in skirmish—descending sometimes on border towns with their wings flickering in the heat or spilling up from a sandy mound to wipe clean a newly-dug mine. Theirs is an orderly bloodshed, committed with no pleasure but the completion of a goal. {\em Instinct} : To fight as ordered
\stopMonsterDescription
       
\startitemize[1,packed]
         
\item Advance as one

         
\item Summon reinforcements

         
\item Give a life for the hive

       
\stopitemize
       
\startMonsterName
Formian Queen	\CMTags{Solitary, Huge, Organized, Intelligent, Hoarder} 
\stopMonsterName
       

Smash (d10+5 damage)	24 HP	3 Armor

       

         \CMTags{Reach, Forceful}        

       
\startMonsterQualities
         {\bf Special Qualities:}  Hive connection, Insectoid
\stopMonsterQualities
       
\startMonsterDescription
At the heart of every hive, no matter its size or kind, lives a queen. As large as any giant, she sits protected by her guard, served by every drone and taskmaster with her own, singular purpose: to spread her kind and grow the hive. To birth the eggs. To nurture. We do not understand the minds of these creatures but it is known they can communicate with their children, somehow, over vast distances and that they begin to teach them the ways of earth and stone and war while still pale and wriggling larva, without a word. To kill one is to set chaos on the hive; without their queen, the rest turn on one another in a mad, blind rage. {\em Instinct} : To spread formians
\stopMonsterDescription
       
\startitemize[1,packed]
         
\item Call every formian it spawned

         
\item Release a half-formed larval mutation

         
\item Set designs on a formian goal

       
\stopitemize
       
\startMonsterName
Gnoll Tracker	\CMTags{Group, Organized, Intelligent} 
\stopMonsterName
       

Bow (d8 damage)	6 HP	1 Armor

       

         \CMTags{Near, Far}        

       
\startMonsterQualities
         {\bf Special Qualities:}  Scent
\stopMonsterQualities
       
\startMonsterDescription
Once they scent your blood, you can’t escape. Not without intervention from the gods, or the duke’s rangers at least. The desert scrub is a dangerous place to go exploring on your own and if you fall and break your leg or eat the wrong cactus, well, you’ll be lucky if you die of thirst before the gnolls find you. They prefer their prey alive, see—cracking bones and the screams of the dying lend a sort of succulence to a meal. Sickening creatures, no? They’ll hunt you, slow and steady, as you die. If you hear laughter in the desert wind, well, best pray Death comes to take you before they do. {\em Instinct} : To prey on weakness
\stopMonsterDescription
       
\startitemize[1,packed]
         
\item Doggedly track prey

         
\item Strike at a moment of weakness

       
\stopitemize
       
\startMonsterName
Gnoll Emissary	\CMTags{Solitary, Divine, Intelligent, Organized} 
\stopMonsterName
       

Ceremonial dagger (d10+2 damage)	18 HP	1 Armor

       

         \CMTags{Close, Reach}        

       
\startMonsterQualities
         {\bf Special Qualities:}  Scent
\stopMonsterQualities
       
\startMonsterDescription
Oh, an emissary! How nice. I suspect you didn’t know the Gnolls had ambassadors, did you? Yes, even these mangy hyenas have to make nice sometimes. No, no, not with us. Nor the dwarves, neither. No, the Emissary is the one, among his packmates, who trucks directly with their dripping demon lord. Frightening? Too right. Every hound has a master with his hand on the chain. This gnoll hears his master’s voice. Hears it and obeys. {\em Instinct} : To share divine insight
\stopMonsterDescription
       
\startitemize[1,packed]
         
\item Pass on demonic influence

         
\item Drive the pack into a fervor

       
\stopitemize
       
\startMonsterName
Gnoll Alpha	\CMTags{Solitary, Intelligent, Organized} 
\stopMonsterName
       

Sword (d10·b damage 1 piercing)	12 HP	2 Armor

       

         \CMTags{Close}        

       
\startMonsterQualities
         {\bf Special Qualities:}  Scent
\stopMonsterQualities
       
\startMonsterDescription
Every pack has its top dog. Bigger, maybe—that’d be the simplest way. Often, though, with these lank and filthy mutts, it’s not about size or sharp teeth but about cruelty. About a willingness to kill your brothers and eat them while the pack watches. Willingness to desecrate the pack in a way that cows them to you. If they’re that awful to each other—to their living kin—think about how they must view us. It’s hard to be mere meat in a land of these kinds of predators. {\em Instinct} : To drive the pack
\stopMonsterDescription
       
\startitemize[1,packed]
         
\item Demand obedience

         
\item Send the pack to hunt

       
\stopitemize
       
\startMonsterName
Orc Bloodwarrior	\CMTags{Horde, Intelligent, Organized} 
\stopMonsterName
       

Jagged blade (d6+2 damage 1 piercing)	3 HP	0 Armor

       

         \CMTags{Close, Messy}        

       
\startMonsterDescription
The orcish horde is a savage, bloodthirsty, and hateful collection of tribes. There are myths and stories that tell of the origin of their rage—a demon curse, a homeland destroyed, elven magic gone wrong—but the truth has been lost to time. Every able orc, be it man or woman, child or elder, swears fealty to the warchief and their tribe and bears the jagged blade of a bloodwarrior. Men are trained to fight and kill—orcs are born to it. {\em Instinct} : To fight
\stopMonsterDescription
       
\startitemize[1,packed]
         
\item Fight with abandon

         
\item Revel in destruction

       
\stopitemize
       
\startMonsterName
Orc Berserker	\CMTags{Solitary, Large, Divine, Intelligent, Organized} 
\stopMonsterName
       

Cleaver (d10+5 damage)	20 HP	0 Armor

       

         \CMTags{Close, Reach}        

       
\startMonsterQualities
         {\bf Special Qualities:}  Mutations
\stopMonsterQualities
       
\startMonsterDescription
Stained in the unholy ritual of Anointing By The Night’s Blood, some warriors of the horde rise to a kind of twisted knighthood. They trade their sanity for this honor, stepping halfway into a world of swirling madness. This makes Berserkers the greatest of their tribe, though as time passes, the chaos spreads. The rare Berserker that lives more than a few years becomes horrible and twisted, growing horns or an extra arm with which to grasp the iron cleavers they favor in battle. {\em Instinct} : To rage
\stopMonsterDescription
       
\startitemize[1,packed]
         
\item Fly into a frenzy

         
\item Unleash chaos

       
\stopitemize
       
\startMonsterName
Orc Breaker	\CMTags{Solitary, Large} 
\stopMonsterName
       

Massive hammer (d10+3 damage ignores armor)	16 HP	0 Armor

       

         \CMTags{Close, Reach, Forceful}        

       
\startMonsterDescription
Before you set out across the hordeland, brave sir, hark a moment to the tale of Sir Regnus. Regnus was like you, sir—a Paladin of the Order, all a-shine in his armored plate and with a shield as tall as a man. Proud he was of it, too—Mirrorshield, he called himself. Tale goes that he’d set his eyes on rescuing some lost priest, a kidnap from the abbey on the borders. Regnus came across some orcs in his travels, a dozen or so, and thought, as one might, that they’d be no match. Battle was joined and all was well until one of them orcs emerged from the fray with a hammer bigger than any man ought to be able to wield. Built more like an ogre or a troll, they say it was, and with a single swing, it crushed Regnus to the ground, shield and all. It were no ordinary orc, they say. It were a breaker. They can’t make plate of their own, see, so maybe it’s jealousy drives these burly things to crush and shatter the way they do. Effective tactic, though. Careful out there. {\em Instinct} : To smash
\stopMonsterDescription
       
\startitemize[1,packed]
         
\item Destroy armor or protection

         
\item Lay low the mighty

       
\stopitemize
       
\startMonsterName
Orc One-Eye	\CMTags{Group, Divine, Magical, Intelligent, Organized} 
\stopMonsterName
       

Inflict Wounds (d8+2 damage ignores armor)	6 HP	0 Armor

       

         \CMTags{Close, Reach, Near, Far}        

       
\startMonsterQualities
         {\bf Special Qualities:}  One eye
\stopMonsterQualities
       
\startMonsterDescription
In the name of He Of Riven Sight and by the First Sacrifice of Elf-Flesh do we invoke the Old Powers. By the Second Sacrifice, I make my claim to what is mine—the dark magic of Night. In His image, I walk the path to Gor-sha-thak, the Iron Gallows! I call to the runes! I call to the clouded sky! Take this mortal organ, eat of the flesh of our enemy and give me what is mine! {\em Instinct} : To hate
\stopMonsterDescription
       
\startitemize[1,packed]
         
\item Rend flesh with divine magic

         
\item Take an eye

         
\item Make a sacrifice

       
\stopitemize
       
\startMonsterName
Orc Shaman	\CMTags{Solitary, Intelligent, Organized} 
\stopMonsterName
       

Flame (d10 damage ignores armor)	12 HP	0 Armor

       

         \CMTags{Close, Reach, Near, Far}        

       
\startMonsterQualities
         {\bf Special Qualities:}  Elemental power
\stopMonsterQualities
       
\startMonsterDescription
The orcs are as old a race as any. They cast bones in the dirt and called to the gods in the trees and stone as the elves built their first cities. They have waged wars, conquered kingdoms, and fallen into corruption in the aeons it took for men to crawl from their caves and dwarves to first see the light of the sun. Fitting, then, that the old ways still hold. They summon the powers of the world to work, to fight and to protect their people, as they have since the first nights. {\em Instinct} : To strengthen orc-kind
\stopMonsterDescription
       
\startitemize[1,packed]
         
\item Give protection of earth

         
\item Give power of fire

         
\item Give swiftness of water

         
\item Give clarity of air

       
\stopitemize
       
\startMonsterName
Orc Slaver	\CMTags{Horde, Stealthy, Intelligent, Organized} 
\stopMonsterName
       

Whip (d6 damage)	3 HP	0 Armor

       

         \CMTags{Close, Reach}        

       
\startMonsterDescription
Red sails fly in the southern sea. Red sails and ships of bone, old wood and iron. The warfleet of the horde. Orcs down that way have taken to the sea, harassing island towns and stealing away with fishermen and their kin. It’s said the custom is spreading north and the orcs learn the value of free work. Taken to it like a sacred task—especially if they can get their hands on elves. Hard to think of a grimmer fate than to live out your life under a lash in an orcish fist. {\em Instinct} : To take
\stopMonsterDescription
       
\startitemize[1,packed]
         
\item Take a captive

         
\item Pin someone under a net

         
\item Drug them

       
\stopitemize
       
\startMonsterName
Orc Shadowhunter	\CMTags{Solitary, Stealthy, Magical, Intelligent} 
\stopMonsterName
       

Poisoned dagger (d10 damage 1 piercing)	10 HP	0 Armor

       

         \CMTags{Close, Reach}        

       
\startMonsterQualities
         {\bf Special Qualities:}  Shadow cloak
\stopMonsterQualities
       
\startMonsterDescription
Not every attack by orcs is torches and screaming and enslavement. Amongst those who follow He Of Riven Sight, poison and murder-in-the-dark are considered sacred arts. Enter the shadowhunter. Orcs cloaked in Night’s magic who slip into camps, towns and temples and end the lives of those within. Do not be so distracted by the howling of the berserkers that you do not notice the knife at your back. {\em Instinct} : To kill in darkness
\stopMonsterDescription
       
\startitemize[1,packed]
         
\item Poison them

         
\item Melt into the shadows

         
\item Cloak them in darkness

       
\stopitemize
       
\startMonsterName
Orc Warchief	\CMTags{Solitary, Intelligent, Organized} 
\stopMonsterName
       

Iron Sword of Ages (2d10·b+2 damage)	16 HP	0 Armor

       

         \CMTags{Close, Reach}        

       
\startMonsterQualities
         {\bf Special Qualities:}  One-Eye blessings, Shaman blessings
\stopMonsterQualities
       
\startMonsterDescription
There are chiefs and there are leaders of the tribes among the orcs. There are those who rise to seize power and fall under the machinations of their foes. There is but one Warchief. One orc in all the horde who stands above the rest, bearing the blessings of the One-Eyes and the Shamans both. Who walks with the elements under Night. Who bears the Iron Sword of Ages and carries the ancient grudge against the civil races on his shoulders. The warchief is to be respected, to be obeyed and above all else, to be feared. All glory to the Warchief. {\em Instinct} : To lead
\stopMonsterDescription
       
\startitemize[1,packed]
         
\item Start a war

         
\item Make a show of power

         
\item Enrage the tribes

       
\stopitemize
       
\startMonsterName
Triton Spy	\CMTags{Solitary, Stealthy, Intelligent, Organized} 
\stopMonsterName
       

Trident (2d10·w damage)	12 HP	2 Armor

       

         \CMTags{Close, Near}        

       
\startMonsterQualities
         {\bf Special Qualities:}  Aquatic
\stopMonsterQualities
       
\startMonsterDescription
A fishing village caught one in their net, some time ago. Part a man and part some scaly sea creature, it spoke in a broken, spy-learned form of the common tongue before it suffocated in the open air. It told the fishermen of a coming tide, an inescapable swell of the power of some deep-sea god and that the triton empire would rise up and drag the land down into the ocean. The tale spread and now, when fishermen sail the choppy seas, they watch and worry that the dying triton’s tales were true. That there are powers deep below that watch and wait. They fear the tide is coming in. {\em Instinct} : To spy on the surface world
\stopMonsterDescription
       
\startitemize[1,packed]
         
\item Reveal their secrets

         
\item Strike at weakness

       
\stopitemize
       
\startMonsterName
Triton Tidecaller	\CMTags{Group, Divine, Magical, Intelligent} 
\stopMonsterName
       

Wave (d8+2 damage ignores armor)	6 HP	2 Armor

       

         \CMTags{Near, Far}        

       
\startMonsterQualities
         {\bf Special Qualities:}  Aquatic, Mutations
\stopMonsterQualities
       
\startMonsterDescription
Part priest, part outcast among their kind, the tidecaller speaks with the voice of the deeps. They can be known by their mutations—transparent skin, perhaps, or rows of teeth like a shark. Glowing eyes or fingertips, angler-lights in the darkness of their underwater kingdom. They speak in strange tongue that can call and command creatures of the sea. They ride wild hippocampi and cast strange spells that rot through the wooden decks of ships or encrust them with barnacles heavy enough to sink. It is the tidecallers who come, now, back to the cities of the Triton, bearing word that the prophecy is coming to pass. The world of men will drown in icy brine. The tidecallers speak and the Lords begin to listen. {\em Instinct} : To bring on The Flood
\stopMonsterDescription
       
\startitemize[1,packed]
         
\item Cast a spell of water and destruction

         
\item Command beasts of the sea

         
\item Reveal divine commands

       
\stopitemize
       
\startMonsterName
Triton Sub-Mariner	\CMTags{Group, Organized Intelligent} 
\stopMonsterName
       

Harpoon (2d8·b damage)	6 HP	3 Armor

       

         \CMTags{Close, Near, Far}        

       
\startMonsterQualities
         {\bf Special Qualities:}  Aquatic
\stopMonsterQualities
       
\startMonsterDescription
The Triton are not a militant race by nature. They shy away from battle except when the sahuagin attack, and only then do they defend themselves and retreat into the depths where their foes can’t follow. This trend begins to change. As the tidecallers come to rally their people, some Triton men and women take up arms. They call these generals “sub-mariners” and build for them armor of shells and hardened glass. They swim in formation, wielding pikes and harpoons and attack the crews of ships that wander too far from port. Watch for their pennants of kelp on the horizon and the conch-cry of a call to battle and keep, if you can, your boats near shore. {\em Instinct} : To wage war
\stopMonsterDescription
       
\startitemize[1,packed]
         
\item Lead Tritons to battle

         
\item Pull them beneath the waves

       
\stopitemize
       
\startMonsterName
Triton Noble	\CMTags{Group, Organized, Intelligent} 
\stopMonsterName
       

Trident (d8 damage)	6 HP	2 Armor

       

         \CMTags{Close, Near, Far}        

       
\startMonsterQualities
         {\bf Special Qualities:}  Aquatic
\stopMonsterQualities
       
\startMonsterDescription
The Triton Ruling Houses were chosen, they say, at the dawn of time. Granted lordship over all the races of the sea by some now-forgotten god. These bloodlines continue, passing rulership from father to daughter and mother to son through the ages. Each is allowed to rule their city in whatever way they choose—some alone or with their spouses, others in council of brothers and sisters. In ages past, they were known for their sagacity and bloodlines of even-temper were respected above all else. The tidecallers prophecy is changing that: Nobles are expected to be strong, not wise. The Nobles have begun to respond, and it is feared by some that the ancient blood is changing forever. It may be too late to turn back. Time and tide wait for none. {\em Instinct} : To lead
\stopMonsterDescription
       
\startitemize[1,packed]
         
\item Stir tritons to war

         
\item Call reinforcements

       
\stopitemize
                
\section{Twisted Experiments}  \index{Twisted Experiments} \index{Twisted} \index{Experiments}
            
\startMonsterName
Bulette	\CMTags{Solitary, Huge, Construct} 
\stopMonsterName
       

Bite (d10+5 damage 3 piercing)	20 HP	3 Armor

       

         \CMTags{Close, Forceful}        

       
\startMonsterQualities
         {\bf Special Qualities:}  Burrowing
\stopMonsterQualities
       
\startMonsterDescription
A seasoned caravan guard learns to listen for the calls of a scout or sentry with a keen ear. A few extra seconds after the alarm is raised can mean life or death. Different cries mean different responses, too—a call of “orcs!” means draw your sword and steady for blood but a call of “bandits!” says you might be able to bargain. One alarm from the scouts that always, always means it’s time to pack up, whip your horse and run for the hills? “LAND SHARK!” {\em Instinct} : To devour
\stopMonsterDescription
       
\startitemize[1,packed]
         
\item Drag prey into rough tunnels

         
\item Burst from the earth

         
\item Swallow whole

       
\stopitemize
       
\startMonsterName
Chimera	\CMTags{Solitary, Large, Construct} 
\stopMonsterName
       

Bite (d10+1 damage)	16 HP	1 Armor

       

         \CMTags{Reach}        

       
\startMonsterDescription
Well known and categorized, the chimera is a perfected creature. From the codices of the Mage’s Guild to the famous pages of Cullaina’s Creature Compendium, there’s no confusion about what Chimera means. Two parts lioness, one part serpent, head of a she-goat, and all the vicious magic one can muster. The actual ritual might vary, as might a detail or two—more creative sorcerers switch the flame breath for acid, perhaps. Used as a guardian, an assassin or merely an instrument of chaos unchained, it matters little. The chimera is the worst sort of abomination: an intentional affront to all natural life. {\em Instinct} : To do as commanded
\stopMonsterDescription
       
\startitemize[1,packed]
         
\item Belch forth flame

         
\item Run them over

         
\item Poison them

       
\stopitemize
       
\startMonsterName
Derro	\CMTags{Horde, Devious, Intelligent, Organized} 
\stopMonsterName
       

Pickaxe (2d6·w damage)	3 HP	2 Armor

       

         \CMTags{Close}        

       
\startMonsterQualities
         {\bf Special Qualities:}  Telepathy
\stopMonsterQualities
       
\startMonsterDescription
It’s typical to think that all the malignant arcane monsters made in this world are birthed by wizards, sorcerers, and their ilk. That the colleges and towers of Dungeon World are womb to every bleak experiment. There are mistakes made in the depths of the earth, too. These ones, the Derro, are the mistakes of a long-forgotten dwarven alchemist. The derro don’t forget, though. Twisted and hateful, the Derro can be spotted by their swollen skulls, brain-matter grown too large. They do not speak except in thoughts to one another and plot in the silent dark to extract sweetest revenge—that of the created on the creator. {\em Instinct} : To replace dwarves
\stopMonsterDescription
       
\startitemize[1,packed]
         
\item Fill a mind with foreign thoughts

         
\item Take control of a beast's mind

       
\stopitemize
       
\startMonsterName
Digester	\CMTags{Solitary, Large, Construct} 
\stopMonsterName
       

Acid (d10+1 damage ignores armor)	16 HP	1 Armor

       

         \CMTags{Close, Reach}        

       
\startMonsterQualities
         {\bf Special Qualities:}  Digest acid secretion
\stopMonsterQualities
       
\startMonsterDescription
It’s okay, magical experimentation is a messy science. For every beautiful pegasus there’s a half-done creature that wasn’t quite right. We understand. The goblin-elephant you thought was such a great idea. The Gelatinous Drake. Just examples. No judgement here. Anyway, we’ve got something for that. We call it the Digester. Yes, just like it sounds. Strange looking, I know, and the smell isn’t the best, but this thing—it’ll eat magic like Svenloff the Stout drinks ale. Next time one of these unfortunate accidents occurs, just point the Digester at it and all your troubles drain away. Just keep an eye on it. Damn thing ate my wand last week. {\em Instinct} : To digest
\stopMonsterDescription
       
\startitemize[1,packed]
         
\item Eat away at something

         
\item Draw sustenance

       
\stopitemize
       
\startMonsterName
Ethereal Filcher	\CMTags{Solitary, Devious, Planar} 
\stopMonsterName
       

Stolen dagger (2d8·w damage)	12 HP	1 Armor

       

         \CMTags{Close, Reach}        

       
\startMonsterQualities
         {\bf Special Qualities:}  Burrowing
\stopMonsterQualities
       
\startMonsterDescription
Things go missing. A sock, a silver spoon, your dead mother’s bones. We blame the maid, or bad luck, or just a moment of stupid forgetfulness and we move on. We never get to see the real cause of these problems. The spidery thing with human hands and eyes as blue as the deep ethereal from whence the creature comes. We never see the nest it makes of astral silver webbing and stolen objects arranged in some madness pattern. We never watch it assemble its collection of halfling finger-bones, stolen from the hands of the sleeping. We’re lucky, that way. {\em Instinct} : To steal
\stopMonsterDescription
       
\startitemize[1,packed]
         
\item Take something important to it's planar lair

         
\item Retreat from the Ethereal plane

         
\item Use an item from its lair

       
\stopitemize
       
\startMonsterName
Ettin	\CMTags{Solitary, Large, Construct} 
\stopMonsterName
       

Club (d10+3 damage)	16 HP	1 Armor

       

         \CMTags{Close, Reach, Forceful}        

       
\startMonsterQualities
         {\bf Special Qualities:}  Two heads
\stopMonsterQualities
       
\startMonsterDescription
What could possibly be better than an idiotic angry hill giant? One with two heads. Fantastic idea, really. Grade A stuff. {\em Instinct} : To smash
\stopMonsterDescription
       
\startitemize[1,packed]
         
\item Attack two enemies at once

         
\item Defend its creator

       
\stopitemize
       
\startMonsterName
Girallon	\CMTags{Solitary, Huge} 
\stopMonsterName
       

Rend (d10+5 damage)	20 HP	1 Armor

       

         \CMTags{Reach, Forceful}        

       
\startMonsterQualities
         {\bf Special Qualities:}  Many arms
\stopMonsterQualities
       
\startMonsterDescription
The pounding of the jungle drums calls to it. The slab of meat on the sacrificial stone to lure in the Great Ape. Girallon, they call it—a name from the long-forgotten tongue of the kings who bred the beast. Taller than a building, some say. Cloaked in ivory fur with tusks as long as scimitars. Four arms? Six? The rumors are hard to verify. Every year it is the same: some explorer visits the jungle villages seeking the Ape and return, never quite the same, never with a trophy. The pounding of the drums goes on. {\em Instinct} : To rule
\stopMonsterDescription
       
\startitemize[1,packed]
         
\item Answer the call of sacrifice

         
\item Drive them from the jungle

         
\item Throw someone

       
\stopitemize
       
\startMonsterName
Iron Golem	\CMTags{Group, Large, Contstruct} 
\stopMonsterName
       

Slam (d8+5 damage)	10 HP	3 Armor

       

         \CMTags{Close, Reach, Forceful}        

       
\startMonsterQualities
         {\bf Special Qualities:}  Metal
\stopMonsterQualities
       
\startMonsterDescription
A staple of the enchanters art. Every golemist and mechano-thaumaturge in the Kingdoms knows this. Iron is a misnomer, though. These guardians are crafted of any metal, really. Steel, copper, or even gold, in some small cases. As much an art as a science, the crafting of a fine golem is as respected in the Kingdoms as a bridge newly built or a castle erected in the mountains. Unceasing watchdogs, stalwart defenders, the iron golem lives to serve, following its orders eternally. Any enchanter worth his salt can craft one, if he can afford the materials. If not… {\em Instinct} : To serve
\stopMonsterDescription
       
\startitemize[1,packed]
         
\item Follow orders implacably

         
\item Use a special tool or adaptation built-in

       
\stopitemize
       
\startMonsterName
Flesh Golem	\CMTags{Horde} 
\stopMonsterName
       

Claw (d6+2 damage)	3 HP	0 Armor

       

         \CMTags{Close, Forceful}        

       
\startMonsterQualities
         {\bf Special Qualities:}  Many body parts
\stopMonsterQualities
       
\startMonsterDescription
Stolen bits and pieces in the night. Graveyards stealthily uprooted and maybe tonight an arm. A leg. Another head (the last one came apart too soon). Even the humblest hedge-enchanter can make due with what he can and, with a little creativity, well—it’s not only the College that can make life, hmm? We’ll show them. {\em Instinct} : To live
\stopMonsterDescription
       
\startitemize[1,packed]
         
\item Follow orders

         
\item Detach a body part

       
\stopitemize
       
\startMonsterName
Kraken	\CMTags{Solitary, Huge} 
\stopMonsterName
       

Slam (d10+5 damage)	20 HP	2 Armor

       

         \CMTags{Reach, Forceful}        

       
\startMonsterQualities
         {\bf Special Qualities:}  Aquatic
\stopMonsterQualities
       
\startMonsterDescription
A cephalo-what? No, boy. Not “a kraken” but “the kraken”. I don’t know what nonsense they taught you at that school you say you’re from, but here, we know to respect the Hungerer. Right, that’s what we call it, The Hungerer in the Deep to be more proper. Ain’t no god, though we’ve got those, too. It’s a squid! A mighty squid with tentacles thicker ‘round than a barrel and eyes the size of the full moon. Smart, too, the Hungerer. Knows just when to strike—when you’re all too drunk or too tired or run out of clean water, that’s when he gets you. No, I ain’t ever seen him. I’m alive, aren’t I? {\em Instinct} : To rule the ocean
\stopMonsterDescription
       
\startitemize[1,packed]
         
\item Drag a person or ship to a watery grave

         
\item Wrap them in tentacles

       
\stopitemize
       
\startMonsterName
Manticore	\CMTags{Solitary, Large, Construct} 
\stopMonsterName
       

Sting (d10+1 damage 1 piercing)	16 HP	3 Armor

       

         \CMTags{Close, Reach, Messy}        

       
\startMonsterQualities
         {\bf Special Qualities:}  Wings
\stopMonsterQualities
       
\startMonsterDescription
If the chimera is the first step down a dark path, the manticore is a door that can’t be closed once its been opened. A lion, a scorpion, the wings of a drake. All difficult to obtain but not impossible and just animals, anyway. The last component, the hissing hateful face of the beast, is the ingredient that makes a manticore so cruel. Young or old, man or woman, it matters not but that they are human, living and breathing, married to the creature with twisted magic. All sense of who they are is lost, and maybe that’s a blessing, but the beast is born from human suffering. No wonder, then, that they’re all so eager to kill. {\em Instinct} : To kill
\stopMonsterDescription
       
\startitemize[1,packed]
         
\item Poison them

         
\item Rip something apart

       
\stopitemize
       
\startMonsterName
Owlbear	\CMTags{Solitary, Construct} 
\stopMonsterName
       

Claw (d10 damage)	12 HP	2 Armor

       

         \CMTags{Close}        

       
\startMonsterDescription
Body of a bear. Feathers of an owl. Beak, claws, and excellent night vision. What’s not to love? {\em Instinct} : To hunt
\stopMonsterDescription
       
\startitemize[1,packed]
         
\item Strike from darkness

       
\stopitemize
       
\startMonsterName
Pegasus	\CMTags{Group, Construct} 
\stopMonsterName
       

Stomp (d8 damage)	10 HP	1 Armor

       

         \CMTags{Close}        

       
\startMonsterQualities
         {\bf Special Qualities:}  Wings
\stopMonsterQualities
       
\startMonsterDescription
Don’t go thinking that every creature not natural-born is a horrible abomination. Don’t imagine for a second that they’re all tentacles and screaming and blood or whatever. Take this noble beast, for example. Lovely thing, isn’t it? A fine white horse with the wings of a swan. Don’t look like it ought to be able to fly, but it does. The elves work miracles, in their own way. They breed true—that’s the purity of elf-magic at work. Hatching from little crystal eggs and bonded with their riders for life. There’s still some beauty in the world, mark my words. {\em Instinct} : To carry aloft
\stopMonsterDescription
       
\startitemize[1,packed]
         
\item Carry a rider into the air

         
\item Give their rider an advantage

       
\stopitemize
       
\startMonsterName
Rust Monster	\CMTags{Group, Construct} 
\stopMonsterName
       

Corrode (d8 damage ignores armor)	6 HP	3 Armor

       

         \CMTags{Close}        

       
\startMonsterQualities
         {\bf Special Qualities:}  Corrosive touch
\stopMonsterQualities
       
\startMonsterDescription
A very distinctive-looking creature. Something like a reddish cricket, I think. Long crickety legs, anyhow. Blind, too, as I understand it—they feel their way around with those long moth-looking tendrils. Feed that way, too. Sift through piles of metal for the choicest bits. That’s what they eat, don’t matter the type, neither. Their merest touch turns it all to rusted flakes. Magic lasts longer but under the scrutiny of a rust monster, it’s a foregone conclusion. Only the gods know where they came from, but they’re a curse if you value your belongings. {\em Instinct} : To decay
\stopMonsterDescription
       
\startitemize[1,packed]
         
\item Turn metal to rust

         
\item Gain strength from consuming metal

       
\stopitemize
       
\startMonsterName
Xorn	\CMTags{Solitary, Large, Construct} 
\stopMonsterName
       

Maw (d10 damage)	12 HP	2 Armor

       

         \CMTags{Close, Reach}        

       
\startMonsterQualities
         {\bf Special Qualities:}  Burrowing
\stopMonsterQualities
       
\startMonsterDescription
Dwarf-made elemental garbage muncher. Shaped like a trash bin with a radius of arms to feed excess rock and stone into its gaping maw. They eat stone and excrete light and heat. Perfect for operating a mine or digging out a quarry. Once one gets lost in the sewers below a city, though, or in the foundation of a castle? You’re in deep trouble. They’ll eat and eat until you’ve got nothing left but to collapse the place down on it and move somewhere else. Ask Burrin, Son of Fjornnvald, exile from his clan. I bet he could tell you a story about a Xorn. {\em Instinct} : To eat
\stopMonsterDescription
       
\startitemize[1,packed]
         
\item Consume stone

         
\item Give off a burst of light and heat

       
\stopitemize
                
\section{Lower Depths}  \index{Lower Depths} \index{Depths}
            
\startMonsterName
Aboleth	\CMTags{Group, Huge, Intelligent} 
\stopMonsterName
       

Tentacle (d8+3 damage)	18 HP	0 Armor

       

         \CMTags{Reach}        

       
\startMonsterQualities
         {\bf Special Qualities:}  Telepathy
\stopMonsterQualities
       
\startMonsterDescription
Deep below the surface of the world, in freshwater seas untouched by the sun dwell the aboleth. Fish the size of whales, with strange growths of gelatinous feelers used to probe the lightless shores. They’re served by slaves; blind albino victims of any race unfortunate enough to stumble on them, drained of thought and life by the powers of the aboleth’s alien mind. In the depths they bid and plot against each other, fishy cultists building and digging upward towards the surface until someday, they’ll breach it. For now, they sleep and dream and guide their pallid minions to do their bidding. {\em Instinct} : To command
\stopMonsterDescription
       
\startitemize[1,packed]
         
\item Invade a mind

         
\item Turn minions on them

         
\item Put a plan in motion

       
\stopitemize
       
\startMonsterName
Apocalypse Dragon	\CMTags{Solitary, Huge, Magical, Divine} 
\stopMonsterName
       

Bite (2d10·b+9 damage 4 piercing)	26 HP	5 Armor

       

         \CMTags{Reach, Forceful, Messy}        

       
\startMonsterQualities
         {\bf Special Qualities:}  Inch-thick metal hide, Supernatural knowledge, Wings
\stopMonsterQualities
       
\startMonsterDescription
The end of all things shall be a burning—of tree and earth and of the air itself. It shall come upon the plains and mountains not from beyond this world but from within it. Birthed from the womb of deepest earth shall come the Dragon that Will End the World. In its passing all will become ash and bile and the earth a dying thing will drift through planar space devoid of life. They say to worship it is to invite madness. They say to love it is to know oblivion. The awakening is coming. {\em Instinct} : To end the world
\stopMonsterDescription
       
\startitemize[1,packed]
         
\item Set a disaster in motion

         
\item Breath forth the elements

         
\item Act with perfect knowledge

       
\stopitemize
       
\startMonsterName
Chaos Spawn	\CMTags{Solitary, Gibbous} 
\stopMonsterName
       

Chaotic touch (d10 damage)	19 HP	1 Armor

       

         \CMTags{Close, Reach}        

       
\startMonsterQualities
         {\bf Special Qualities:}  Chaos form
\stopMonsterQualities
       
\startMonsterDescription
Driven from the city, a cultist finds solace in towns and villages. Discovered there, he flees to the hills and scratches his devotion on the cave walls. Found out, he is chased with knife and torch into the depths, crawling deeper and deeper until, in the deepest places, he loses his way. First, he forgets his name. Then he forgets his shape. His chaos gods, most beloved, bless him with a new one. {\em Instinct} : To undermine
\stopMonsterDescription
       
\startitemize[1,packed]
         
\item Rewrite reality

         
\item Unleash chaos from containment

       
\stopitemize
       
\startMonsterName
Chuul	\CMTags{Group, Large, Cautious} 
\stopMonsterName
       

Claws (d8+1 damage 3 piercing)	10 HP	4 Armor

       

         \CMTags{Close, Reach, Messy}        

       
\startMonsterQualities
         {\bf Special Qualities:}  Amphibious
\stopMonsterQualities
       
\startMonsterDescription
Let us, for a moment, consider the lobster. This one is your worst seafood nightmare come to life. A sort of vicious, half-man half-crawdad cursed with primal intelligence and blessed with a pair of razor-sharp claws. Strange things lurk in the stinking pools in caverns best forgotten and the Chuul is one of them. If you spot one, your best hope is a heavy mace to crack its shell and maybe a little garlic butter. Mmmm. {\em Instinct} : To split
\stopMonsterDescription
       
\startitemize[1,packed]
         
\item Split something in two with mighty claws

         
\item Retreat into water

       
\stopitemize
       
\startMonsterName
Deep Elf Assassin	\CMTags{Group, Intelligent, Organized} 
\stopMonsterName
       

Poisoned blade (d8 damage 1 piercing)	6 HP	1 Armor

       

         \CMTags{Close}        

       
\startMonsterDescription
It was not so simple a thing as a war over religion or territory. No disagreement of Queens led to the great sundering of the elves. It was sadness. It was the very diminishing of the world by the lesser races and the glory of all the elves had built was cracking and turning to glass. Some, then, chose to separate themselves from the world; wracked with tears they turned their backs on men and dwarves. Others, though, they were overcome with something new. A feeling no elf had felt before. Spite. Hatred filled these elves and twisted them and they turned on their weaker cousins. Some still remain after the great exodus below. Some hide amongst us with spider-poisoned blades, meting out that strangest of punishments: elven vengeance. {\em Instinct} : To spite
\stopMonsterDescription
       
\startitemize[1,packed]
         
\item Poison them

         
\item Unleash an ancient spell

         
\item Call reinforcements

       
\stopitemize
       
\startMonsterName
Deep Elf Swordmaster	\CMTags{Group, Intelligent, Organized} 
\stopMonsterName
       

Barbed blade (2d8·b+2 damage 1 piercing)	6 HP	2 Armor

       

         \CMTags{Close}        

       
\startMonsterDescription
The deep elves lost the sweetness and gentle peace of their bright cousins ages ago but did not abandon grace. They move with a swiftness and beauty that would bring a tear to any warrior’s eye. In the dark, they’ve practiced. A cruelty has infested their swordsmanship—a wickedness comes to the fore. Barbed blades and whips replace the shining pennant-spears of surface elven battles. The swordmasters of the deep elf clans do not merely seek to kill, but to punish with every stroke of their blades. Wickedness and pain are their currency. {\em Instinct} : To punish
\stopMonsterDescription
       
\startitemize[1,packed]
         
\item Inflict pain beyond measure

         
\item Use the dark to advantage

       
\stopitemize
       
\startMonsterName
Deep Elf Priest	\CMTags{Solitary, Divine, Intelligent, Organized} 
\stopMonsterName
       

Smite (d10+2 damage)	14 HP	0 Armor

       

         \CMTags{Close, Reach}        

       
\startMonsterQualities
         {\bf Special Qualities:}  Divine connection
\stopMonsterQualities
       
\startMonsterDescription
The spirits of the trees and the lady sunlight are far far from home in the depths where the deep elves dwell. New gods were found, there, waiting for their children to come home. Gods of the spiders, the fungal forests, and things that whisper in the forbidden caves. The deep elves, ever attuned to the world around them, listened with hateful intent to their new gods and found a source of power yet unrealized. Hate calls to hate and grim alliances were made. Even among these spiteful ranks, piety finds a way to express itself. {\em Instinct} : To pass on divine vengeance
\stopMonsterDescription
       
\startitemize[1,packed]
         
\item Weave spells of hatred and malice

         
\item Rally the deep elves

         
\item Pass on divine knowledge

       
\stopitemize
       
\startMonsterName
Dragon	\CMTags{Solitary, Huge, Terrifying, Cautious, Hoarder} 
\stopMonsterName
       

Bite (2d10·b+5 damage 4 piercing)	16 HP	5 Armor

       

         \CMTags{Reach, Messy}        

       
\startMonsterQualities
         {\bf Special Qualities:}  Elemental blood, Wings
\stopMonsterQualities
       
\startMonsterDescription
They are the greatest and most terrible things this world will ever have to offer. {\em Instinct} : To rule
\stopMonsterDescription
       
\startitemize[1,packed]
         
\item Bend an element to its will

         
\item Demand tribute

         
\item Act with disdain

       
\stopitemize
       
\startMonsterName
Gray Render	\CMTags{Solitary, Large} 
\stopMonsterName
       

Rend (d10+3 damage)	16 HP	1 Armor

       

         \CMTags{Close, Reach, Forceful}        

       
\startMonsterDescription
On its own, the render is a force of utter destruction. Huge and leathery, with a maw of unbreakable teeth and claws to match, the render seems to enjoy little more than tearing things apart. Stone, flesh, or steel, it matters little. However, the gray render is so rarely found alone. They bond with other creatures. Some at birth, others as fully-grown creatures, and will follow their bonded master wherever it goes, bringing them offerings of meat and protecting them while they sleep. Finding an un-bonded render means certain riches, if you survive to sell it. {\em Instinct} : To serve
\stopMonsterDescription
       
\startitemize[1,packed]
         
\item Tear something apart

       
\stopitemize
       
\startMonsterName
Magmin	\CMTags{Horde, Intelligent, Organized, Hoarder} 
\stopMonsterName
       

Flaming hammer (d6+2 damage)	7 HP	4 Armor

       

         \CMTags{Close, Reach}        

       
\startMonsterQualities
         {\bf Special Qualities:}  Firey blood
\stopMonsterQualities
       
\startMonsterDescription
Dwarf-shaped and industrious, the magmin are among the deepest-dwellers of Dungeon World. Found in cities of brass and obsidian built nearest the molten core of the planet, the magmin live a life devoted to craft—especially that of fire and magical items related to it. Surly and strange, they do not often deign to speak to petitioners who appear at their gates, even those who have somehow found a way to survive the hellish heat. Even so, they respect little more than a finely made item and to learn to forge from a magmin craftsman means unlocking secrets unknown to surface blacksmiths. Like so much else, visiting the magmin is a game of risk and reward. {\em Instinct} : To craft
\stopMonsterDescription
       
\startitemize[1,packed]
         
\item Offer a trade or deal

         
\item Strike with fire or magic

         
\item Provide just the right item, at a price

       
\stopitemize
       
\startMonsterName
Minotaur	\CMTags{Solitary, Large} 
\stopMonsterName
       

Axe (d10+1 damage)	16 HP	1 Armor

       

         \CMTags{Close, Reach}        

       
\startMonsterQualities
         {\bf Special Qualities:}  Unerring sense of direction
\stopMonsterQualities
       
\startMonsterDescription
Head of a man, body of a bull. No, wait, I’ve got that backwards. It’s the bull’s head and the man’s body. Hooves sometimes? Is that right? I remember the old King said something about a maze? Blast! You know I can’t think under this kind of pressure. What was that? Oh gods, I think it’s coming… {\em Instinct} : To contain
\stopMonsterDescription
       
\startitemize[1,packed]
         
\item Confuse them

         
\item Make them lost

       
\stopitemize
       
\startMonsterName
Naga	\CMTags{Solitary, Intelligent, Organized, Hoarder, Magical} 
\stopMonsterName
       

Bite (d10 damage)	12 HP	2 Armor

       

         \CMTags{Close, Reach}        

       
\startMonsterDescription
Ambitious and territorial above nearly all else, the naga are very rarely found without a well-formed and insidious cult of followers. You’ll see it in many mountain towns—a snake sigil scrawled on a tavern wall or a local church burned to the ground. People going missing into the mines. Men and women wearing the mark of the serpent. At the core of it all lies a naga; an old race now fallen into obscurity, still preening with the head of a man over its coiled, serpent body. Variations of these creatures exist depending on their bloodline and original purpose, but they are all master manipulators and magical forces to be reckoned with. {\em Instinct} : To lead
\stopMonsterDescription
       
\startitemize[1,packed]
         
\item Send a follower to their death

         
\item Use old magic

         
\item Offer a deal or bargain

       
\stopitemize
       
\startMonsterName
Salamander	\CMTags{Horde, Large, Intelligent, Organized, Planar} 
\stopMonsterName
       

Flaming spear (2d6·b+3 damage)	7 HP	3 Armor

       

         \CMTags{Close, Reach, Near}        

       
\startMonsterQualities
         {\bf Special Qualities:}  Burrowing
\stopMonsterQualities
       
\startMonsterDescription
The excavation uncovered a basalt gate, the reports called it. Black stone carved with molten runes. When they dug it up, the magi declared it inert but further evidence indicates that was an incorrect claim. The entire team went missing. When we arrived, the gate was glowing. Its light filled the whole cavern. We could see from the entrance that the area had become full of these creatures—like red and orange skinned men, tall as an ogre but with a snakes tail where there legs ought to be. They were clothed, too —some had black glass armor. They spoke to each other in a tongue that sounded like grease in a fire. I wanted to leave but the Sergeant wouldn’t listen. You’ve already read what happened next, sir. I know I’m the only one that got back, but what I said is true. The gate is open, now. This is just the beginning! {\em Instinct} : To consume in flame
\stopMonsterDescription
       
\startitemize[1,packed]
         
\item Summon elemental fire

         
\item Melt away deception

       
\stopitemize
                
\section{Planar Powers}  \index{Planar Powers} \index{Planar} \index{Powers}
            
\startMonsterName
Angel	\CMTags{Solitary, Terrifying, Divine, Intelligent, Organized} 
\stopMonsterName
       

Flame sword (2d10·b+4 damage ignores armor)	18 HP	4 Armor

       

         \CMTags{Close, Forceful}        

       
\startMonsterQualities
         {\bf Special Qualities:}  Wings
\stopMonsterQualities
       
\startMonsterDescription
So was it written that the heavens opened up to Avra’hal and did an angel from the clouds emerge to speak unto her and so did it appear to her as her firstborn daughter—beautiful, of ebon skin and golden eyes—and did Avra’hal cry tears to see it. “Be not afraid” it commanded her “go to the villages I have shown you in your dreams and unto them show the word I have written on your soul.” Avra’hal wept and wept and did agree to do this and did take up her sword and tome and did into the villages go, a great thirst for blood on her lips for the word the angel wrote upon the soul of Avra’hal was “kill”. {\em Instinct} : To share divine will
\stopMonsterDescription
       
\startitemize[1,packed]
         
\item Deliver visions and prophecy

         
\item Stir mortals to action

         
\item Expose sin and injustice

       
\stopitemize
       
\startMonsterName
Barbed Devil	\CMTags{Solitary, Large, Planar, Terrifying} 
\stopMonsterName
       

Spine (d10+3 damage 3 piercing)	16 HP	3 Armor

       

         \CMTags{Close, Reach, Messy}        

       
\startMonsterQualities
         {\bf Special Qualities:}  Spines
\stopMonsterQualities
       
\startMonsterDescription
There are a thousand forms of devil, maybe more. Some common and some unique. Each time the Inquisitors discover a new one they write it into the codex and the knowledge is shared among the abbeys in the hope that the atrocities of that particular sort won’t find their way into the world again. The barbed devil has long been known to the brothers and sisters of the inquisition. A literal thing, it appears only at a site of great violence or when called by a wayward summoner. Covered in sharp quills, this particular demon revels in the spilling of blood, being specifically fond of impaling victims piecemeal or in whole upon its thorns and letting them die there. Cruel but not particularly effective beyond slaughter. A low inquisitorial priority. {\em Instinct} : To bloody
\stopMonsterDescription
       
\startitemize[1,packed]
         
\item Impale someone

         
\item Kill randomly

       
\stopitemize
       
\startMonsterName
Chain Devil	\CMTags{Solitary, Planar} 
\stopMonsterName
       

Crush (d10 damage ignores armor)	12 HP	3 Armor

       

         \CMTags{Close, Reach}        

       
\startMonsterDescription
Do you think the phrase “drag him to hell” means nothing? It is unfortunately literal, in the case of the chain devil. Appearing differently to each victim, this summoned creature has but a single purpose: to wrap its victim up in binding coils and take it away to a place of torment. Sometimes it will come as a man-shaped mass of rusting iron, hooks and coils of mismatched links. Other times, a roiling tangle of rope or kelp or twisted bloody bedsheets. The results are always the same. {\em Instinct} : To capture
\stopMonsterDescription
       
\startitemize[1,packed]
         
\item Take a captive

         
\item Return to whence it came

         
\item Torture with glee

       
\stopitemize
       
\startMonsterName
Concept Elemental	\CMTags{Solitary, Devious, Planar, Gibbous} 
\stopMonsterName
       
\startMonsterQualities
         {\bf Special Qualities:}  Ideal form
\stopMonsterQualities
       
\startMonsterDescription
The planes are not as literal as our world. Clothed in the elemental chaos are places of stranger stuff than air and water. Here, rivers of time crash upon shores of crystal fear. Bleak storms of nightmare roil and churn in a laughter-bright sky. Sometimes, the spirits of these places can be lured into our world, though they are infinitely more unpredictable and strange than mere fire or earth might be. Easier to make mistakes, too—one might try calling up a Wealth Elemental and be surprised to find a Murder Elemental instead. {\em Instinct} : To perfect its concept
\stopMonsterDescription
       
\startitemize[1,packed]
         
\item Show its concept in its purest form

       
\stopitemize
       
\startMonsterName
Corrupter	\CMTags{Solitary, Devious, Planar, Hoarder} 
\stopMonsterName
       

Secret dagger (2d8·w damage)	12 HP	0 Armor

       

         \CMTags{Close}        

       
\startMonsterDescription
Surely, my good man, you must know why I am here. Must know who I am. You said the words. You spilled the blood and followed the instructions almost to the letter. Your pronunciation was a bit off but that’s to be expected. I’ve come to give you what you’ve always wanted, friend. Glory, love, money? Paltry things when you’ve the vaults of hell to plumb. Don’t look so shocked, you knew what this was. You have but one thing we desire. Promise it to us, and the world shall be yours for the taking. Trust me. {\em Instinct} : To bargain
\stopMonsterDescription
       
\startitemize[1,packed]
         
\item Offer a deal with horrible consequences

         
\item Plumb the vaults of hell for a bargaining chip

         
\item Make a show of power

       
\stopitemize
       
\startMonsterName
Djinn	\CMTags{Group, Large, Magical} 
\stopMonsterName
       

Flame (d8+1 damage ignores armor)	14 HP	4 Armor

       

         \CMTags{Close, Reach}        

       
\startMonsterQualities
         {\bf Special Qualities:}  Made of flame
\stopMonsterQualities
       
\startMonsterDescription
Stop rubbing that lamp, you idiot. I do not care what you have read, it will not grant you wishes. I brought you here to show you something real, something true. See this mural? It shows the ancient city. The true city that came before. They called it Majilis and it was made of brass by the spirits. They had golem servants and human lovers and, in that day, it was said you could trade them a year of your life for a favor. We are not here to gather treasure this night, fool, we are here to learn. The djinn still sometimes come to these places, and you must understand their history if you are to know how to behave. They are powerful and wicked and proud and you must know them if you hope to survive a summoning. Now, bring the lamp here and we will light it, it grows dark and these ruins are dangerous at night. {\em Instinct} : To burn eternally
\stopMonsterDescription
       
\startitemize[1,packed]
         
\item Grant power for a price

         
\item Summon the forces of the City of Brass

       
\stopitemize
       
\startMonsterName
Hell Hound	\CMTags{Group, Planar, Organized} 
\stopMonsterName
       

Fiery Bite (d8 damage)	10 HP	1 Armor

       

         \CMTags{Close}        

       
\startMonsterQualities
         {\bf Special Qualities:}  Hide of shadow
\stopMonsterQualities
       
\startMonsterDescription
When one reneges on a deal, does not the debtor come for payment? Does the owed party not send someone to collect what is due? So too with the Powers Below. They only want what is theirs. A howling pack of shadows, flame and jagged bone, driven by the hunting horn. They will not cease, they cannot be evaded. {\em Instinct} : To pursue
\stopMonsterDescription
       
\startitemize[1,packed]
         
\item Follow despite all obstacles

         
\item Spew fire

         
\item Summon the forces of hell on their target

       
\stopitemize
       
\startMonsterName
Imp	\CMTags{Horde, Planar, Intelligent, Organized} 
\stopMonsterName
       

Flame gout (d6 damage ignores armor)	7 HP	1 Armor

       

         \CMTags{Close, Near, Far}        

       
\startMonsterDescription
These tiny observer-demons often act as a first-time binding subject by neonate warlocks. They can be found infesting arcane cabals, drinking potions when no-one watches, and chasing pets and servants with tiny pitchforks. A caricature of true demonhood, these little creatures are, thankfully, not too difficult to bind or extinguish. {\em Instinct} : To harass
\stopMonsterDescription
       
\startitemize[1,packed]
         
\item Send information back to hell

         
\item Cause mischief

       
\stopitemize
       
\startMonsterName
Inevitable	\CMTags{Group, Large, Magical, Cautious, Gibbous, Planar} 
\stopMonsterName
       

Hammer (d8+1 damage)	21 HP	5 Armor

       

         \CMTags{Close, Reach}        

       
\startMonsterQualities
         {\bf Special Qualities:}  Made of Order
\stopMonsterQualities
       
\startMonsterDescription
All things come to an end. Entropy bleeds reality slowly out. At the edge of time itself stand the Inevitable. Massive, powerful and seemingly carved from star-stuff themselves, the Inevitable intervene only where magic or calamity have undone the skein of fate. Where the arrogant and powerful boil the substance of destiny away and seek to undermine the very laws of reality, the Inevitable arrive to guide things back to the proper order. Unshakable, seemingly immune to mortal harm and utterly enigmatic, it is said that the Inevitable are all that will remain when time’s long thread has run out. {\em Instinct} : To preserve order
\stopMonsterDescription
       
\startitemize[1,packed]
         
\item End a spell or effect

         
\item Enforce a law of nature or man

         
\item Give a glimpse of destiny

       
\stopitemize
       
\startMonsterName
Larvae	\CMTags{Horde, Devious, Planar, Intelligent} 
\stopMonsterName
       

Slime (2d4·w damage)	10 HP	0 Armor

       

         \CMTags{Close}        

       
\startMonsterDescription
Those who have seen visions of the Planes Below, and survived with their sanity intact, speak of masses of these writhing wretches. Maggots with the face of men and women, crying out for salvation in a nest of flames. Sometimes, they can be goaded out through a rip in the planar caul and emerge, wriggling and in torment, into our world. Once here, they spread misery and sickness during their mayfly lives before expiring into a slurry of gore. All in all, an inspiration towards good deeds in life. {\em Instinct} : To suffer
\stopMonsterDescription
       
\startitemize[1,packed]
         
\item Fill them with despair

         
\item Beg for mercy

         
\item Draw evil attention

       
\stopitemize
       
\startMonsterName
Nightmare	\CMTags{Horde, Large, Magical, Terrifying, Planar} 
\stopMonsterName
       

Trample (d6+1 damage)	7 HP	4 Armor

       

         \CMTags{Close, Reach}        

       
\startMonsterQualities
         {\bf Special Qualities:}  Flame and shadow
\stopMonsterQualities
       
\startMonsterDescription
The herd came from a pact made in the days when folk still inhabited the Blasted Steppes. Horselords, they were, who travelled those lands. Born in the saddle, it was said. One of theirs, in a bid to dominate his peers, made a black pact with some fell power and traded away his finest horses. He had some power, sure—but what’s a thousand year dynasty when a life is so short? Now the fiends of the pit ride on the finest horses ever seen. Coats of shining oil and manes of tormented flame: these are steeds of hell’s cavalry. {\em Instinct} : To ride rampant
\stopMonsterDescription
       
\startitemize[1,packed]
         
\item Sheath a rider in hellish flame

         
\item Drive them away

       
\stopitemize
       
\startMonsterName
Quasit	\CMTags{Horde, Planar} 
\stopMonsterName
       

Hellish weaponry (d6 damage)	7 HP	2 Armor

       

         \CMTags{Close}        

       
\startMonsterQualities
         {\bf Special Qualities:}  Adaptable form
\stopMonsterQualities
       
\startMonsterDescription
An imp with some ambition. A quasit is a kind of foot soldier in the demon realm. A commoner, armed with fangs or claws or wings or some other thing to give it just a little edge over its hellish peers. Commonly bound by warlocks to carry heavy loads or build bridges or guard their twisted towers, a quasit can take many forms, none of them pleasant. {\em Instinct} : To serve
\stopMonsterDescription
       
\startitemize[1,packed]
         
\item Attack with abandon

         
\item Inflict pain

       
\stopitemize
       
\startMonsterName
The Tarrasque	\CMTags{Solitary, Huge, Planar} 
\stopMonsterName
       
\startMonsterQualities
         {\bf Special Qualities:}  Impervious
\stopMonsterQualities
       
\startMonsterDescription
The Tarrasque. Legendary unstoppable juggernaut—eater of cities and swallower of ships, horses, and knights. A creature unseen in an age but about whom all kinds of stories are told. One thread of truth weaves through these stories. It cannot be killed. No blade can pierce its stony shell nor spell penetrate the shield it somehow bears. Stories say, though, that the will of one pure soul can send it to slumber, though what that means and, by the gods, where such a thing might be found, pray we do not ever need to learn. It slumbers. Somewhere in the periphery of the planar edge, it sleeps for now. {\em Instinct} : To consume
\stopMonsterDescription
       
\startitemize[1,packed]
         
\item Swallow a person, group, or place whole

         
\item Release a remnant of a long-eaten place from its gullet

       
\stopitemize
       
\startMonsterName
Word Daemon	\CMTags{Solitary, Planar, Magical} 
\stopMonsterName
       
\startMonsterDescription
All of mortal magic is just words. Spells are prayers, rote formula, runes cast, or songs sung. Letters, words, sentences, and syntax strung together in a language that the whole world itself might understand. By way of words we can make our fellows cry or exult, can paint pictures and whisper desire to the gods. No little wonder, then, that in all that power is intent. That every word we utter, if repeated and meaning or emotion given to it, can spark a kind of unintentional summoning. Word daemons are called by accident, appear at random and are often short-lived, but come to attend a particular word. Capricious, unpredictable and dangerous, yes—but possibly useful, depending on the word. {\em Instinct} : To further their word
\stopMonsterDescription
       
\startitemize[1,packed]
         
\item Cast a spell related to their word

         
\item bring their word into abundance

       
\stopitemize
                
\section{Folk of the Realm}  \index{Folk of the Realm} \index{Folk} \index{Realm}
            
\startMonsterName
Acolyte	
\stopMonsterName
       
\startMonsterDescription
Can’t all be the High Priest, they said. Can’t all wield the White Spire, they said. Scrub the floor, they told me. The Cthonic Overgod don’t want a messy floor, do he? They said it’d be enlightenment and magic. Feh. It’s bruised knees and dish-pan-hands. If only I’d been a cleric, instead. {\em Instinct} : To serve dutifully
\stopMonsterDescription
       
\startitemize[1,packed]
         
\item Follow dogma

         
\item Offer eternal reward for mortal deeds

       
\stopitemize
       
\startMonsterName
Adventurer	\CMTags{Horde, Intelligent} 
\stopMonsterName
       

Sword (d6 damage)	3 HP	1 Armor

       

         \CMTags{Close}        

       
\startMonsterQualities
         {\bf Special Qualities:}  Endless enthusiasm
\stopMonsterQualities
       
\startMonsterDescription
Scum of the earth, they are. A troupe of armored men and women come sauntering into town, brandishing what, for all intents and purposes, is enough magical and mundane power to level the whole place. Bringing with them bags and bags of loot, still dripping blood from whatever poor sod they had to kill to get it. An economical fiasco waiting to happen, if you ask me. The whole system becomes completely uprooted. Dangerous, unpredictable murder-hobos. Oh, wait, you’re an adventurer? I take it all back. {\em Instinct} : To adventure or die trying
\stopMonsterDescription
       
\startitemize[1,packed]
         
\item Go on a fool's errand

         
\item Act impulsively

         
\item Share tales of past exploits

       
\stopitemize
       
\startMonsterName
Bandit	\CMTags{Horde, Intelligent, Organized} 
\stopMonsterName
       

Dirk (d6 damage)	3 HP	1 Armor

       

         \CMTags{Close}        

       
\startMonsterDescription
Desperation is the watchword of banditry. When times are tough, what else is there to do but scavenge a weapon and take up with a clan of nasty men and women? Highway robbery, poaching, scams and cons and murder most foul but we’ve all got to eat so who can blame them? Then again, there’s evil in the hearts of some and who’s to say that desperation isn’t a want to sate one’s baser lusts? Anyway—it’s this or starve, sometimes. {\em Instinct} : To rob
\stopMonsterDescription
       
\startitemize[1,packed]
         
\item Steal something

         
\item Demand tribute

       
\stopitemize
       
\startMonsterName
Bandit King	\CMTags{Solitary, Intelligent, Organized} 
\stopMonsterName
       

Trusty knife (2d10·b damage)	12 HP	1 Armor

       

         \CMTags{Close}        

       
\startMonsterDescription
Better to rule in hell than serve in heaven. {\em Instinct} : To lead
\stopMonsterDescription
       
\startitemize[1,packed]
         
\item Make a demand

         
\item Extort

         
\item Topple power

       
\stopitemize
       
\startMonsterName
Fool	
\stopMonsterName
       
\startMonsterDescription
There’s not but one person in all the King’s court allowed to speak the truth. The real, straight-and-honest truth about anything. The Fool couches it all in bells and prancing and chalky face-paint, but who else gets to tell the King what’s what? You can trust a Fool, they say, especially when he’s made you red-faced and you’d just as soon drown him in a cesspit. {\em Instinct} : To mock
\stopMonsterDescription
       
\startitemize[1,packed]
         
\item Expose injustice

         
\item Play a trick

       
\stopitemize
       
\startMonsterName
Guardsman	\CMTags{Group, Intelligent, Organized} 
\stopMonsterName
       

Spear (d8 damage)	6 HP	1 Armor

       

         \CMTags{Close, Reach}        

       
\startMonsterDescription
Noble protector or merely drunken lout, it often makes no difference to these sorts. Falling just shy of a noble Knight, the proud town guard is an ancient profession none-the-less. These folks of the constabulary often dress in the colors of their lord (when you can see it under the mud) and, depending on the richness of that lord, might even have a decent weapon and some armor that fits. Those are the lucky ones. Even so, someone has to be there keep an eye on the gate when the Black Riders have been spotted in the woods. Too many of us owe our lives to these souls—remember that the next time one is drunkenly insulting your mother, hmm? {\em Instinct} : To do as ordered
\stopMonsterDescription
       
\startitemize[1,packed]
         
\item Uphold the law at all costs

         
\item Make a profit

       
\stopitemize
       
\startMonsterName
Halfling Thief	\CMTags{Solitary, Small, Intelligent, Stealthy, Devious} 
\stopMonsterName
       

Dagger (2d8·w damage)	12 HP	1 Armor

       

         \CMTags{Close}        

       
\startMonsterDescription
It would be foolish, now, to draw conclusions about folks just because they happen to be good at one thing or another. Then again, a spade’s a spade, isn’t it. Or maybe just the goodly, soft-and-sweet type of Halfling have the mind to stay in their grassy-hill homes and aren’t the type you find in the slums and taverns of the mannish world . Perhaps they’re there to cut your purse for calling them “halfing” in the first place. Not all take so kindly to the title. Or they’re playing a game, pretending to be a child in need of alms—and your arrogant eyes can’t even see the difference until too late. Well, it matters little. They’re gone with your coin before you even realize you deserved it. {\em Instinct} : To live a life of stolen luxury
\stopMonsterDescription
       
\startitemize[1,packed]
         
\item Steal

         
\item Put on the appearance of friendship

       
\stopitemize
       
\startMonsterName
Hedge Wizard	\CMTags{Magical} 
\stopMonsterName
       
\startMonsterDescription
Not all those who wield the arcane arts are adventuring Wizards. Nor necromancers in mausoleums or sorcerers of ancient bloodline. Some are just old men and women, smart enough to have discovered a trick or two. It might make them a bit batty to come by that knowledge, but if you’ve a curse to break or a love to prove, might be that a hedge wizard will help you, if you can find his rotten hut in the swamp and pay the price he asks. {\em Instinct} : To learn
\stopMonsterDescription
       
\startitemize[1,packed]
         
\item Cast almost the right spell (for a price)

         
\item Make deals beyond their ken

       
\stopitemize
       
\startMonsterName
High Priest	
\stopMonsterName
       
\startMonsterDescription
Respected by all who gaze upon them, the high priests and abbesses of Dungeon World are treated with a sort of reverence. Whether they pay homage to Ur-thuu-hak, God of Swords, or whisper quiet prayers to Namiah, precious daughter of peace, they know a thing or two that you and I, we won’t ever know. The gods speak to them as a hawker-of-wares might speak to us in the marketplace. For this, for the bearing-of-secrets and the knowing-of-things, we give them a wide berth as they pass in their shining robes. {\em Instinct} : To lead
\stopMonsterDescription
       
\startitemize[1,packed]
         
\item Set down divine law

         
\item Reveal divine secrets

         
\item Commission divine undertakings

       
\stopitemize
       
\startMonsterName
Hunter	\CMTags{Group, Intelligent} 
\stopMonsterName
       

Ragged bow (d6 damage)	6 HP	1 Armor

       

         \CMTags{Near, Far}        

       
\startMonsterDescription
The wilds are home to more than just beasts of horn and scale. There are men and women out there, too—those who smell blood on the wind and stalk the plains in the skins of their prey. Whether with a trusty longbow bought on a rare trip into the city or with a knife of bone and sinew-made, these folk have more in common with the things they track and make their meals than with their own kind. Solemn, somber and quiet, they find a sort of peace in the wild. {\em Instinct} : To survive
\stopMonsterDescription
       
\startitemize[1,packed]
         
\item Bring back news from the wilds

         
\item Slay a beast

       
\stopitemize
       
\startMonsterName
Knight	\CMTags{Solitary, Intelligent, Organized, Cautious} 
\stopMonsterName
       

Sword (2d10·b damage)	12 HP	4 Armor

       

         \CMTags{Close}        

       
\startMonsterDescription
What youngster doesn’t cling to the rail at the mighty joust, blinded by the sun on their glittering armor, wishing they could be the one adorned in steel and riding to please the King and Queen? What peasant youth with naught but a loaf of bread and a lame sow doesn’t wish to trade it all in for the lance and the bright pennant? A Knight is many things—a holy warrior, a sworn sword, a villain sometimes, too, but a Knight cannot help but be a symbol to all who see her. A Knight means something. {\em Instinct} : To live by a code
\stopMonsterDescription
       
\startitemize[1,packed]
         
\item Make a moral stand

         
\item Lead peasants into battle

       
\stopitemize
       
\startMonsterName
Merchant	
\stopMonsterName
       
\startMonsterDescription
Ten foot poles. Get your ten foot poles, here. Torches, bright and hot. Mules, too - stubborn but immaculately bred. Need a linen sack, do you? Right over here! Come and get your ten foot poles! {\em Instinct} : To profit
\stopMonsterDescription
       
\startitemize[1,packed]
         
\item Propose a business venture

         
\item Offer a 'deal'

       
\stopitemize
       
\startMonsterName
Noble	
\stopMonsterName
       
\startMonsterDescription
Are they granted their place by the gods, perhaps? Is that why they’re able to pass their riches and power down by birth? Some trick or enchantment of the blood, maybe. The peasant bends his knee and scrapes and toils and the noble wears the finery of his place and, they say, we all have our burdens to bear. Seems to me that some of us have burdens of stone and some carry their weight in gold. It’s a tough life. {\em Instinct} : To rule
\stopMonsterDescription
       
\startitemize[1,packed]
         
\item Issue an order

         
\item Offer a reward

       
\stopitemize
       
\startMonsterName
Peasant	
\stopMonsterName
       
\startMonsterDescription
Covered in muck, downtrodden at the bottom of the great chain of being, we all stand on the backs of those that grow our food on their farms. Some peasants do better than others, but none will ever see a coin of gold in their day. They’ll dream at night of how someday, somehow, they’ll fight a dragon and save a princess. Don’t act like you weren’t one before you lost what little sense you had, adventurer. {\em Instinct} : To get by
\stopMonsterDescription
       
\startitemize[1,packed]
         
\item Plead for help

         
\item Offer a simple reward and gratitude

       
\stopitemize
       
\startMonsterName
Rebel	\CMTags{Horde, Intelligent, Organized} 
\stopMonsterName
       

Axe (d6 damage)	3 HP	1 Armor

       

         \CMTags{Close}        

       
\startMonsterDescription
In the countryside they’d be called outlaw and driven off or killed. The city, though, is full of places to hide. Damp basements to pore over maps and to plan and plot against a corrupt system. Like rats, they gnaw away at order, either to supplant it anew or just erode the whole thing. The line between change and chaos is a fine one—some rebels walk that thin line and others just want to see it all torched. Disguise, a knife in the dark or a thrown torch at the right moment are all tools of the rebel. The burning brand of anarchy is a common fear amongst the nobles of Dungeon World. These men and women are why. {\em Instinct} : To upset order
\stopMonsterDescription
       
\startitemize[1,packed]
         
\item Die for a cause

         
\item Inspire others

       
\stopitemize
       
\startMonsterName
Soldier	\CMTags{Horde, Intelligent, Organized} 
\stopMonsterName
       

Spear (d6 damage)	3 HP	1 Armor

       

         \CMTags{Close, Reach}        

       
\startMonsterDescription
For a commoner with a strong arm, sometimes it’s this or be a bandit. It’s wear the colors and don ill-fitting armor and march into the unknown with a thousand other scared men and women conscripted to fight the wars of our time. They could be hiding out in the woods instead, living off poached elk and dodging the king’s guard. Better to risk ones life in service to a cause. To bravely toss one’s lot in with their fellows and hope to come out the other side still in one piece. Besides, the nobles need strong men and women. What is it they say? A handful of soldiers beats a mouthful of arguments. {\em Instinct} : To fight
\stopMonsterDescription
       
\startitemize[1,packed]
         
\item March into battle

         
\item Fight as one

       
\stopitemize
       
\startMonsterName
Spy	
\stopMonsterName
       
\startMonsterDescription
Beloved of Kings but never truly trusted. Mysterious, secretive and alluring, the life of a spy is, if you ask a commoner, full of romance and intrigue. They’re a knife in the dark and a pair of watchful eyes. A spy can be your best friend, your lover or that old man you see in the market every day. One never knows. Hells, maybe you’re a spy—they say there’s magic that can turn folks minds without them ever knowing it. How can we trust you? {\em Instinct} : To infiltrate
\stopMonsterDescription
       
\startitemize[1,packed]
         
\item Report the truth

         
\item Double cross

       
\stopitemize
       
\startMonsterName
Tinkerer	
\stopMonsterName
       
\startMonsterDescription
It’s said that if you see a tinker on the road and you don’t offer him a swig of ale or some of your food that he’ll leave a curse of bad luck behind. A tinker is a funny thing. These strange folk often travel the roads between towns with their oddment carts and favorite mules. With a ratty dog and always a story to tell. Sometimes the mail, too, if you’re lucky and live in a place where Queen’s Post won’t go. If you’re kind, maybe they’ll sell you a rose that never wilts or a clock that chimes with the sound of faerie laughter. Or maybe they’re just antisocial peddlers. You never know, right? {\em Instinct} : To create
\stopMonsterDescription
       
\startitemize[1,packed]
         
\item Offer an oddity at a price

         
\item Spin tales of great danger and reward in far-off lands

       
\stopitemize
                
