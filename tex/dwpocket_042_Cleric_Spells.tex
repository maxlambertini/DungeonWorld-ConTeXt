\chapter{Cleric Spells}
 \index{Cleric Spells} \index{Cleric} \index{Spells}
 
\section{Guidance}  \index{Guidance} \index{Guidance}
 

It's up the the creativity of your deity (and the GM) to communicate as much as possible through the motions and gestures of you deity's symbol. You don't get visions or a voice from heaven, just some visual cue of what your deity would have you do (even if it's not in your best interest).

 
\section{Magic Weapon}  \index{Magic Weapon} \index{Magic} \index{Weapon}
 

Casting Magic Weapon on the same weapon again has no effect. No matter how many times you cast it on the same weapon it's still just magic +1d4 damage.

 

Magic though is nothing to be scoffed at. Having a magic weapon may give you an advantage against some of the stranger beasts of Dungeon World, ghosts and the sort. The exact effects depend on the monster and circumstances, so make the most of it.

 
\section{Animate Dead}  \index{Animate Dead} \index{Animate} \index{Dead}
 

Treating the zombie as your character means you make moves with it's ability scores based on the fiction, just like always. Unless it's brain is functioning on its own the zombie can't do much besides follow the last order it was given, so you'd better stay close. Even if its brain works it's still bound to follow your orders.

 
